\section{Tutorial Kennismaking met de Computer}

\subsection{Inleiding}
\subsubsection{Dit hoofdstuk is onderhevig aan veranderingen}

Dit hoofdstuk beschrijft de omgang met de X--terminals zoals die gebruikt
worden bij het tweedejaars practicum.
Het gebruik van de X--terminals is in hoge mate configureerbaar.
Het gaat hier om instellingen als kleur, lettertypes, de gebruikte
window manager en nog enkele andere zaken.
Voor het practicum is gekozen voor een relatief eenvoudige set
van instellingen.
Doordat inzichten in het gebruik van X--terminals veranderen, kan het
gebeuren dat kleine veranderingen in de configuratie en dus
het gebruik worden doorgevoerd.
Dit kan dan ook enkele verschillen t.o.v.\ dit hoofdstuk met zich meebrengen.

\subsubsection{Hoe moet dit hoofdstuk worden gelezen}

Dit hoofdstuk is bedoeld om de gebruiker snel en eenvoudig de
basis van het gebruik van X--terminals en UNIX bij te brengen.
De snelste manier om dit gebruik onder de knie te krijgen is door
achter een scherm plaats te nemen, in te loggen en ''het'' te doen.
Ter ondersteuning hiervan is paragraaf~\ref{Een voorbeeld login sessie}
opgenomen. Deze paragraaf is een voorbeeldsessie.
Wij raden u aan om paragraaf~\ref{Een voorbeeld login sessie}
te lezen voordat u achter het scherm gaat zitten.
\newline
De meest elementaire zaken zoals inloggen, uitloggen, tekst invoeren,
manual pages lezen, en de basis UNIX commando's komen aan bod.
Paragrafen~\ref{Login},~\ref{Logout} gaan iets dieper in op het in- en
uitlog proces en het uiterlijk van het scherm voor en na inloggen.
Voor meer achtergrond-informatie over UNIX en X-windows wordt u
verwezen naar appendix~\ref{UNIX en X-windows}.
Paragraaf~\ref{Objecten op het scherm} behandelt alle objecten die op
het scherm zichtbaar zijn na login.
Paragraaf~\ref{UNIX} gaat in op het gebruik van het operating systeem UNIX.
Hieronder valt ook het gebruik van de c-shell, de basis UNIX commando's,
teksteditors en tekstverwerkers, elektronische post en het printen en plotten
van files.

\subsection{Een voorbeeld login sessie}
\label{Een voorbeeld login sessie}
\subsubsection{Login}
\label{Login}

Het loginscherm wordt gevormd door een donkere achtergrond met in het
midden een zogenaamd identificatie ''widget'' (window met invul-form).
Hierin bevindt zich het loginveld en het passwordveld.
Alle aanslagen op het toetsenbord worden doorgegeven aan dit loginprocess.
U hoeft de pointer dus niet op het loginveld te zetten om uw
login-account in te tikken.
Een login-account naam en password moet gevolgd worden door een {\em Return}.
Het intikken van passwords wordt niet op het scherm ge\"echoed.
\newline
Als het systeem uw account heeft gecontroleerd,
wordt voor u een sessie opgestart.
Het eerste wat verschijnt is de ''{\em message of the day}'',
als er tenminste iets zinnigs te melden valt.\\
Ook kunnen er andere meldingen verschijnen,
b.v.\ dat er {\em mail} of ongelezen {\em news} is.\\
U kunt nu commando's gaan intikken achter de prompt in een {\em xterm}-window.

\subsubsection{Vituele schermen}
\label{Vituele schermen}

De beeldbuis waar u achter werkt lijkt redelijk groot maar toch is
deze vaak niet groot genoeg.
Vaak komt het voor dat u enkele tientallen windows tegelijk open hebt
en dan is het scherm niet groot genoeg om het netjes te houden.
Om hier wat aan te doen zijn 4 virtuele schermen naast elkaar geplaatst.
Linksonderaan op het scherm ziet u een vak met
de naam {\em Misc} en daaronder een vlak in 4 stukken verdeeld.
Deze stukken zijn een kleine versie van ieder virtueel scherm.
\newline
U kunt uzelf verplaatsen van het ene naar het andere virtuele scherm
door met de muis tegen de rand van het scherm aan te lopen.
U zult zien dat het scherm plotseling wordt vervangen door een ander scherm.
Ook kunt u uzelf verplaatsen middels toetscombinaties.
Dit werkt op den duur sneller.
Gebruik de {\em Control} toets en de pijltjes toetsen.
{\em Ctrl-$<$pijltje naar beneden$>$} betekent uiteraard ga naar
het virtuele scherm onder het huidige scherm.

\subsubsection{De pointer}
\label{De pointer}

De pointer kunt u met de muis over het scherm bewegen.
Wanneer u hierbij de pointer binnen een window brengt zult u zien dat
de pointer van vorm verandert.
Op het donkere achtergrondscherm heeft de pointer de vorm van een hoofdletter X.
In een {\em xterm}-window verandert de pointer in een vorm die lijkt op \index{xterm}
een hoofdletter I.
Merk op dat wanneer de pointer binnen een {\em xterm}-window of een ander
window komt waar iets kan worden gedaan, dat dan de kleur van de rand
om dit window aangeeft dat het actief is.
Standaard heeft een actief
window een roze rand en een niet-actief window een paarse rand.
In de meeste andere windows zal de pointer een schuin pijltje zijn.
Wanneer het systeem van u verlangt dat u even wacht zal de pointer veranderen
in een zandloper.
Deze zandloper heeft u b.v.\ kunnen zien op het moment dat u inlogde.

Wanneer u de pointer voorzichtig langs de rand van b.v.\ een {\em xterm}
beweegt, kunt u de pointer  van vorm zien veranderen.
De pointer geeft dan ongeveer aan wat u met de rand kunt doen.
Ga b.v.\ naar de onderrand van de {\em xterm}. De pointer
verandert in een horizontaal streepje met daarop een pijl die naar
beneden wijst. Druk de linker-muisknop in en houd hem ingedrukt,
trek nu de rand iets naar beneden. Wanneer u de muisknop loslaat zal
de {\em xterm} de grootte aannemen die u heeft aangegeven.
Alle operaties die het kader van een window mogelijk maakt,
zijn met de linker-muisknop te activeren.

\subsubsection{Windows en iconen} \index{icoon|(bold}
\label{Windows en iconen}

Rechtsboven in de hoek van het kader van de {\em xterm} bevinden \index{xterm}
zich 2 vakjes.
\begin{figure}[bth]
\centerline{\callpsfig{kader_rb.ps}{width=0.2\textwidth}}
\caption{Deel van de Motif kaders
\label{kader_rb}}
\end{figure}
Het eerste vakje bevat een driehoekje dat naar beneden wijst en
tweede vakje bevat een driehoekje dat naar boven wijst.
Met deze knoppen kan een
window worden ge{\ii}conificeerd of maximaal worden vergroot.
Klik \'e\'enmaal op het driehoekje naar beneden van de {\em xterm}.
Er verschijnt nu een icoon.
Het hangt een beetje van de applicatie af wat er gebeurd bij maximaal vergroten.
Het betekent niet dat windows automatisch het hele scherm gaan innemen.

\begin{figure}[bth]
\centerline{\callpsfig{xterm_icon.ps}{width=0.2\textwidth}}
\caption{Een standaard icoon
\label{icon}}
\end{figure}
Klik nu tweemaal snel achter elkaar met de linker-muisknop
op het icoon.
Het icoon verdwijnt en de originele {\em xterm} komt weer tevoorschijn.
We kunnen een window verplaatsen door de balk met de titel van een
window met de linker-muisknop te klikken en vast te houden.
Er verschijnt nu een dun kader wat verplaatst kan worden met de muis.
Verplaats de {\em xterm} iets en laat dan los.

\subsubsection{Een tweede xterm}
\label{Een tweede xterm}
We cre\"eren nu een tweede {\em xterm}.
Verplaats de pointer naar het achtergrondscherm.\\
De pointer is nu in een kruis veranderd.
Klik op de linker-muisknop en houd hem ingedrukt.
Er verschijn nu een popup-menu.\\
Trek de muis iets naar beneden zodat de knop met {\tt idaard} erop
iets geaccentueerd wordt.
Laat nu de muisknop los.
Na enige tijd verschijnt een kader op het scherm
ter grootte van een {\em xterm}
dat meteen met de beweging van de muis meeloopt.
Plaats het kader op een acceptabele plaats en
klik met de linker-muisknop.
Het nieuwe window wordt nu volledig geplaatst.
We hebben slechts \'e\'en window nodig
dus kan \'e\'en van de twee {\em xterm}'s
worden ge{\ii}conificeerd.
Druk met de linker-muisknop op het kleine driehoekje uit
figuur~\ref{kader_rb}\index{icoon|)} dat naar beneden wijst.

\subsubsection{Het gebruik van de manual pages}\index{manual page}
\label{Het gebruik van de manual pages}

Onder {\em UNIX} worden manual pages van oudsher gelezen met het commando {\em man}.
Voor bijna alle commando's onder UNIX bestaat een {\em manual page}.
De manual van het commando {\em ls} wordt
b.v.\ opgeroepen middels het commando {\em man ls}.

{\em Man} is tekst ge\"orienteerd.
Er zijn ook grafische versies van {\em man}.
De bekendste hiervan heet (geheel in UNIX stijl) {\em xman}.
In veel gevallen is deze {\em manual-browser} handiger dan {\em man}.\\
U kunt hem opstarten door op het {\em manual}-icoon onderaan te klikken.

\subsubsection{Het veranderen van de achtergrondkleur}
\label{Het veranderen van de achtergrondkleur}

We gaan nu de achtergrondkleur proberen te veranderen van
de standaard login kleur in een mooie kleur groen.
We gebruiken hiervoor het commando \tool{xsetroot}.
U moet het commando \tool{xsetroot} aanroepen met een optie
die we nu eerst met de manual page gaan opzoeken.
Klik op het {\em manual}-icoon om de manual-browser {\em xman} op te starten.
Ga naar het {\em Sections} pulldown-menu van het {\em Manual Page}
window en kies voor de {\em User Commands} optie.
Scroll met de balk naar beneden totdat u \tool{xsetroot} hebt gevonden.
Klik nu op \tool{xsetroot} en de manual page van \tool{xsetroot} komt
tevoorschijn.
Ga nu op zoek naar de optie die gebruikt moet worden om
de achtergrondkleur te veranderen.
Als u die gevonden heeft tikt u in:
\begin{verbatim}
        [idaard:op5/op5u9] xsetroot <de juiste optie> seagreen
\end{verbatim}
Als het u pijn aan uw ogen doet,
dan kunt u \tool{xsetroot} zonder optie intikken
om een standaard grijze achtergrond terug te krijgen.
Om de paarse achtergrond terug te krijgen moet u intikken:
\begin{verbatim}
        [idaard:op5/op5u9] xsetroot <de juiste optie> #554055
\end{verbatim}

\subsubsection{Tekst editen}
\label{Tekst editen}
We gaan nu een stukje tekst genereren \index{jet|(bold}
m.b.v.\ de teksteditor {\em jet}.
Een teksteditor moet niet worden verward met een tekstprocessor.
Met een teksteditor bewerkt (edit) u een stukje ASCII tekst,
maar een tekstprocessor verwerkt een stuk tekst.
Ga met de muis (pointer) naar een {\em xterm} en
tik achter de prompt het commando {\em jet} in
gevolgd door een {\em Return}.\\
U kunt ook gewoon op het {\em editor}-icoon klikken
die onderaan het scherm te vinden is,
maar dan wordt {\em jet} in uw home directory opgestart.
Om {\em jet} als achtergrond-proces op te starten en
het window van figuur~\ref{jet_1} te krijgen tikt u:
\begin{verbatim}
        [idaard:op5/op5u9] jet &
\end{verbatim}

\begin{figure}[bth]
\centerline{\callpsfig{jet.ps}{width=0.5\textwidth}}
\caption{Het {\em jet} window
\label{jet_1}}
\end{figure}

Voordat u iets kunt intikken moet u een file openen of
een nieuwe file opgeven.
Ga met de muis naar de {\em File} pulldown-menu knop rechtsboven,
klik hem aan en houdt hem vast.
Ga nu met de muis naar beneden totdat u bij {\em New} komt en
laat de muisknop dan los.
Bij een nieuwe file vraagt {\em jet} om een naam.
Tik hier b.v.\ in {\tt my\_demo\_file}.

Ga nu naar het grote tekstveld binnen het {\em jet} window en
tik een stukje tekst in.
Dit kan b.v.\ een klein programma zijn of een sls-beschrijving.
Fouten kunt u met de {\em Backspace} toets weghalen.
U kunt de tekst opslaan door te klikken op {\em File} waarna
u in het ontstane pulldown-menu de optie {\em Save} selecteert.
We bevelen van harte aan om met deze editor te experimenteren.

Wanneer u i.p.v.\ {\em New} voor {\em Open} kiest verschijnt er een
window waarmee u kunt bladeren (browsen) in de files die u heeft.
Alle files die beginnen met een {\em punt} zijn uw
systeem-omgeving-configuratie files,
deze kunt u beter niet veranderen.
\index{jet|)}

\subsubsection{Het gebruikt van de c-shell}
\label{Het gebruikt van de c-shell}
Nu u een kleine file heeft,
kunnen we enkele leuke dingen doen met de {\em c-shell}.
Dit is het programma dat in de {\em xterm} draait en
tegen het operating systeem UNIX praat.
Ga met de pointer naar de {\em xterm} en volg de volgende sessie.
\index{ls|(bold}
Bekijk eerst de inhoud van uw directory met het commando {\em ls} (list short):
\begin{verbatim}
   [idaard:op5/op5u9] ls
   my_demo_file
\end{verbatim}

Dit is echter niet de enige file in uw home directory.
Files die beginnen met een ''.'' worden door {\em ls} standaard niet getoond.
Vaak zijn files die beginnen met een ''.'' speciale files.\\
U kunt ze toch zien door de optie ''-a'' (all) aan {\em ls} mee te geven.
Tik in:
\begin{verbatim}
   [idaard:op5/op5u9] ls -a
   ./            .alias*        .cshrc*     .login*        .simeyerc*
   ../           .alias.hp700*  .emacs*     .motifbind*    .terminalrc*
   .Xauthority   .alias.hp800*  .fvwm2rc    .prompt*       my_demo_file
\end{verbatim}

U ziet dat er naast ''{\tt my\_demo\_file}'' nog heel wat andere files in uw
directory staan.
Files die eindigen met een ''{\tt $\ast$}'' zijn
executeerbare files, althans dat denkt UNIX.
Files die eindigen met een ''{\tt /}'' zijn directories.
''{\tt ./}'' wijst naar de huidige directory en
''{\tt ../}'' wijst naar de directory \'e\'en hoger in de hierarchie.
Een aantal van de files die met een punt beginnen zullen binnenkort verdwijnen.
\newline
U kunt nog veel meer over de files in uw directory te weten komen.
Hiervoor gebruikt u het commando {\em ll} \index{ll}(long listing)
of {\em ls -al} (list -all -long).
Bijvoorbeeld:

\begin{verbatim}
  [idaard:op5/op5u9] ls -al
  total 66
  drwxr-xr-x   2 op5u9    op5         2048 Jul  2 11:22 ./
  drwxr-xr-x  20 root     op5         1024 Dec  8  1994 ../
  -rw-------   1 op5u9    op5           49 Jul  2 11:48 .Xauthority
  -r-xr-xr-x   1 op5u9    sys          841 Jul  2 10:09 .alias*
  -r-xr-xr-x   1 op5u9    sys          127 Jul  2 10:09 .alias.hp700*
  -r-xr-xr-x   1 op5u9    sys          126 Jul  2 10:09 .alias.hp800*
  -r-xr-xr-x   1 op5u9    sys         1586 Jul  2 10:09 .cshrc*
  -r-xr-xr-x   1 op5u9    sys        15537 Jul  2 10:09 .emacs*
  -rw-r--r--   1 op5u9    sys        10554 Jul  2 10:09 .fvwm2rc
  -r-xr-xr-x   1 op5u9    sys          744 Jul  2 10:09 .login*
  -r-xr-xr-x   1 op5u9    sys          572 Jul  2 10:09 .motifbind*
  -r-xr-xr-x   1 op5u9    sys         2558 Jul  2 10:09 .prompt*
  -r-xr-xr-x   1 op5u9    sys          118 Jul  2 10:09 .simeyerc*
  -r-xr-xr-x   1 op5u9    sys          461 Jul  2 10:09 .terminalrc*
  -rw-r--r--   1 op5u9    op5          113 Jul  2 12:05 my_demo_file
\end{verbatim}

De meest interessante informatie
die u uit de bovenstaande listing kunt halen
is wie de eigenaar van een file is,
hoe groot de file is en wat de protecties zijn.
Voor de file {\tt my\_demo\_file} in uw directory kunt u zien dat de eigenaar
de gebruiker ''{\tt op5u9}'' is,
en dat deze persoon deel uitmaakt van de groep ''{\tt op5}''.
Bij u zal het iets zijn als gebruiker ''{\tt opXuY}'' en groep ''{\tt opX}''.
Verder kunt u de lengte van b.v.\ ''{\tt my\_demo\_file}'' zien.
Deze is 113 karakters (bytes).
De tijdsaanduiding geeft aan wanneer er voor het laatst iets aan een file
is veranderd.
De bovenste 2 files zijn directories,
dit kunt u zien aan de ''{\tt d}'' waarmee het protectieveld begint.
De rest van het protectieveld laten we nu \index{ls|)} onbesproken.
''{\tt my\_demo\_file}'' is de file die u net heeft gemaakt met {\em jet}.
U kunt de inhoud van deze file ook
bekijken zonder dat u een editor hoeft op te starten.
Hiervoor gebruikt u het commando {\em cat}
(afkorting van con{\em cat}enate):\index{cat|bold}

\begin{verbatim}
   [idaard:op5/op5u9] cat my_demo_file

          This is my demo file.
          It doesn't tell you very much!
          In fact, it tells you nothing at all.
\end{verbatim}
\index{more|(bold}Als een file zo lang is dat hij niet in z'n geheel in een window past,
heeft het zin om i.p.v.\ {\em cat} het commando {\em more} te gebruiken.
De file wordt dan in stukken op het scherm getoond.
Steeds als er een nieuw scherm vol is wacht {\em more}
tot u een karakter intikt.
Als u een {\em Spatie} intikt (op de Space-bar klopt)
verschijnt het volgende scherm,
als u een {\em Return} intikt verschijnt alleen de volgende regel.
Een lange file op het systeem is
b.v.\ de file ''.cshrc'' in uw home directory.
Let u niet op de inhoud.
Probeer het volgende commando, en
bedenk dat het intikken van een {\em q} in plaats van een {\em Spatie}
het programma stopt:
\begin{verbatim}
   [idaard:op5/op5u9] more .cshrc
   if ( ! ${?LOGNAME} ) setenv LOGNAME `whoami`
   if ( ! ${?HOSTNAME} ) setenv HOSTNAME `uname -n`
   if ( ! ${?CACD} ) setenv CACD ~cacd     # default release/3 directory
   ...
                ~ocean/bin/$OCEAN_BIN \
   --More--(30%)
\end{verbatim}
\index{more|)}
De naam van een file kan vrij lang zijn.
Ook de naam ''{\tt my\_demo\_file}'' is al lang
en veel tikwerk begint snel te vervelen
(en geeft sneller aanleiding tot het maken van fouten).
Als u weet dat de naam van een file met ''{\tt my}'' begint,
dan kunt u de computer ({\em c-shell}) zelf de naam laten afmaken.
Dit doet u door na het intikken van ''{\tt my}''
op de {\em Tab} toets te drukken.
Automatisch zal nu de volle naam verschijnen.
Als er meerdere files zijn die met ''{\tt my}'' beginnen zal
er niets gebeuren en moet u eerst nog verder tikken.\\
Om te zien welke files dat zijn
kunt u ook {\em Ctrl-D} intikken.
\index{cat}
\begin{verbatim}
   [idaard:op5/op5u9] cat my<TAB>
\end{verbatim}

wordt

\begin{verbatim}
   [idaard:op5/op5u9] cat my_demo_file
\end{verbatim}

U kunt ook gebruik maken van X-windows om sneller commando's te genereren
dan met normaal tikken mogelijk zou zijn.
Zo kunt u b.v.\ de string ''{\tt my\_demo\_file}'' in de muisbuffer zetten,
en deze op een andere plaats weer op het scherm zetten.
Ga b.v.\ met de pointer van de muis naar
de ''{\tt m}'' van ''{\tt my\_demo\_file}'' in uw vorige commando.
Klik met de linker-muisknop en houd hem vast.
Trek de pointer naar rechts totdat de
hele string ''{\tt my\_demo\_file}'' in inverse video is verschenen
en laat de muisknop weer los (snel dubbel klikken had ook gekund).
De tekst staat nu in de buffer van de muis.
Tik nu {\em cat} en een spatie in en
druk daarna op de middelste-muisknop (deze leegt het muisbuffer).
U ziet dat de tekst ''{\tt my\_demo\_file}'' verschijnt
waar u bent opgehouden met tikken.
Pas op,
alles wat u met de linker-muisknop toevallig in inverse video zet,
staat in het muisbuffer.
Leeg het muisbuffer daarom niet zomaar!

Als u een fout maakt tijdens het intikken,
kunt u met de {\em Backspace} toets op de regel terug gaan.
De laatste karakters worden dan gewist.
Als u het commando helemaal niet meer ziet zitten
kunt u {\em Ctrl-U} intikken.
Dan wordt de hele regel gewist.
Zet nu ''{\tt cat my\_demo\_file}'' met de muis opnieuw achter de prompt:
\begin{verbatim}
   [idaard:op5/op5u9] cat my_demo_file
\end{verbatim}

Toets nu 7 keer het {\em Backspace} karakter in
en geef dan een {\em Ctrl-U}
omdat u de inhoud van die file nu al zo vaak heeft gezien.
Met het commando {\em cp}\index{cp|bold} (copy) kunt u files kopi\"eren.
Bijvoorbeeld:
\begin{verbatim}
   [idaard:op5/op5u9] cp my_demo_file new_file
\end{verbatim}

U heeft nu twee identieke files.
U wilt nu beide files bekijken met het commando {\em ls -l},
maar u wilt niet de ander files in uw directory zien.
Dit kan,
door aan {\em ls} op te geven welke files u precies wilt zien:
\begin{verbatim}
   [idaard:op5/op5u9] ls -l my_demo_file new_file
\end{verbatim}

Maar dit kan ook handiger.
B.v.\ door het gebruik van een zogenaamde wildcard
(zie paragraaf~\ref{File namen en wildcards}).
Dit is een constructie waarbij de \tool{c-shell} voor u uitzoekt
welke files aan de door u gestelde criteria voldoen.
Een ''{\tt $\ast$}'' betekent dat de \tool{c-shell}
op die plaats zelf alles mag invullen.
Een ''{\tt $\ast$}'' staat voor nul of meer karakters.
Daarom is de volgende constructie ook mogelijk:
\begin{verbatim}
   [idaard:op5/op5u9] ls -l *file
\end{verbatim}

Dit wordt door de \tool{c-shell} vertaald naar het vorige commando
en dan pas uitgevoerd.
Overigens zou ''{\tt ls -l $\ast$e}'' ook al gewerkt hebben
aangezien er maar 2 files in uw directory zijn
die met een ''{\tt e}'' eindigen.
Files die met een punt beginnen worden met ''{\tt $\ast$}'' niet geselecteerd.
\newline
Gooi nu de laatste file weg:
\begin{verbatim}
   [idaard:op5/op5u9] rm new_file
\end{verbatim}

U begrijpt dat u vroeg of laat \index{rm|bold}
in de problemen kunt komen als u een dergelijke wildcard gebruikt tezamen
met het {\em rm} (remove) commando.
Daarom is voor uw veiligheid
in het practicum {\em rm} een alias voor {\em rm -i} (inquery).
\newline
Om overzicht te kunnen houden over uw files is het handig om files die bij elkaar horen in \'e\'en subdirectory te zetten. Het commando om een subdirectory te maken is \tool{mkdir} (make directory).
Maak nu onder uw home directory een subdirectory ''{\tt tekst}'':
\begin{verbatim}
   [idaard:op5/op5u9] mkdir tekst
\end{verbatim}

Bekijk met \tool{ls} of \tool{ll} of het gelukt is.\\
Files kunnen van de ene directory naar de andere verplaatst worden met \tool{mv} (move):\index{mv|bold}
\begin{verbatim}
   [idaard:op5/op5u9] mv my_demo_file tekst
\end{verbatim}

Controleer het resultaat weer met \tool{ls} of \tool{ll}.
Of doe eens ''{\tt ll tekst}''.

We gaan nu een \smc{nelsis} project cre\"eren.
Een \smc{nelsis} project \index{project}
is een database waarin alle gegevens m.b.t.\ het ontwerp van een IC (integrated circuit) staan.
Voor het ontwerppracticum wordt een speciale database gemaakt waar u de beschikking heeft over een aantal bibliotheken.
Het commando hiervoor is \tool{mkopr} (make ontwerppracticum project).\index{mkopr|(bold}
Tik in:
\begin{verbatim}
   [idaard:op5/op5u9] mkopr test
\end{verbatim}

In feite is \tool{mkopr} een afgeleide van het commando \tool{mkpr}.
Met dit laatste commando wordt een nieuwe \smc{nelsis} database gecre\"eerd.
Er wordt van u een process-type verwacht en ook een zogenaamde lambda.
Deze lambda is de steek in micrometers waarmee u gaat ontwerpen.
Verder wordt voor het ontwerppracticum gebruik gemaakt van een cellen-bibliotheek.
Deze zult u bekend moeten maken aan de database middels enkele mistige
commando's.
Omdat deze acties voor iedereen in het ontwerppracticum hetzelfde zijn
is \tool{mkopr} bedacht.
Eigenlijk doet het niet meer dan het kopi\"eren van
een template-project naar uw directory.
Het enige wat u hoeft op te
geven is de naam van het project.\index{mkopr|)}
\index{cd|(bold}
Om gebruik te kunnen maken van de nieuwe database,
moet u 'naar het project' gaan.
Het project is echter niets anders dan een nieuwe subdirectory
en dus kunt
u met het commando {\em cd} (change directory) naar het project gaan.
Als volgt:
\begin{verbatim}
   [idaard:op5/op5u9] cd test
   [idaard:op5u9/test]
\end{verbatim}

U ziet dat de prompt veranderd is.
De prompt geeft het laatste stuk van uw huidige werkdirectory weer.
Zo heeft u altijd een redelijk idee waar u zich
bevindt binnen het systeem.
In het laatste geval bent u van directory veranderd.
U kunt nu kijken waar u zich in totaliteit bevindt
met het commando {\em pwd} (print working directory):
\begin{verbatim}
   [idaard:op5u9/test] pwd
   /usrI1/op/op5/op5u9/test
\end{verbatim}

De volgende drie {\em cd} commando's brengen u allemaal weer \'e\'en directory hoger
naar uw zogenaamde home directory:
\begin{verbatim}
   [idaard:op5u9/test] cd
\end{verbatim}

of

\begin{verbatim}
   [idaard:op5u9/test] cd ..
\end{verbatim}

of

\begin{verbatim}
   [idaard:op5u9/test] cd /usrI1/op/op5/op5u9
\end{verbatim}

''{\tt cd}'' zonder opties betekent: ga terug naar de home directory.
''{\tt cd ..}'' betekent: ga naar de directory die boven de huidige ligt.
Dit verklaart
meteen het nut van een file in iedere directory die wijst naar de directory
daarboven.
''{\tt cd /usrI1/op/op5/op5u9}'' geeft een specifiek pad aan waarnaar
verhuisd moet worden.
Uiteraard verandert de prompt steeds weer mee.
We gaan toch nog even terug naar het project ''{\tt test}''
en kijken eens in het rond.
Tik de commando's achter de prompt in en vergelijk het resultaat op het scherm met hieronder:
\begin{verbatim}
   [idaard:op5/op5u9] cd test
   [idaard:op5u9/test] ls
   circuit/         layout/          sls_prototypes/
   floorplan/       projlist
   [idaard:op5u9/test] ls -l
   total 10
   drwxr-xr-x   2 op5u9    op5        1024 Jul  3 11:19 circuit/
   drwxr-xr-x   2 op5u9    op5        1024 Jul  3 11:19 floorplan/
   drwxr-xr-x   2 op5u9    op5        1024 Jul  3 11:19 layout/
   -rw-r--r--   1 op5u9    op5         263 Jul  3 11:19 projlist
   drwxr-xr-x   2 op5u9    op5        1024 Jul  3 11:19 sls_prototypes/
\end{verbatim}

\subsubsection{Het printen van een file}\index{printen|bold}
\label{Het printen van een file}
U wilt nog even de inhoud van een file printen,
b.v.\ de file ''{\tt .cshrc}'' in uw home directory.
Dat kunt u als volgt doen:
\index{lp}\index{printen!lp}
\begin{verbatim}
   [idaard:op5u9/test] cd
   [idaard:op5/op5u9] lp .cshrc
   Converting your ASCII file to Postscript.
   request id is midlum-8585 (1 file)
\end{verbatim}
\index{cd|)}

De printer zal vrijwel meteen beginnen met ratelen.
U kunt kijken hoe het gesteld is met alle print jobs.
Dit is handig als uw print-job niet meteen aan de beurt is.
Geef het volgende commando:
\index{lpq}
\begin{verbatim}
   [idaard:op5/op5u9] lpq
   midlum-8585     op5u9   caslp.6198      13601   bytes
\end{verbatim}

Als de printer om de een of andere reden niet wil printen,
b.v.\ als hij off-line staat,
dan komt er mogelijk een andere melding.

Als u de print-job alsnog niet wilt laten printen,
kunt u deze (als u snel genoeg bent)
als volgt uit de print-queue verwijderen:
\index{cancel}
\begin{verbatim}
   [idaard:op5/op5u9] cancel midlum-8585
   request "midlum-8585" cancelled
\end{verbatim}

Het bovenstaande geldt voor het printen van zogenaamde ASCII files.
Dit zijn files die u kunt editen, cat-en en more-en.
Maar hetzelfde geldt ook voor Postscript files.
Soms moet of wilt u echter bij het commando \tool{lp}
nog andere opties opgeven (b.v.\ voor HPGL files).\\
Zie voor meer informatie paragraaf~\ref{Printen en Plotten}.

\subsubsection{Electronic mail}
\label{Electronic mail} \index{pine|(bold}

Alle e-mail van de OP accounts zijn gekoppeld met de persoonlijk
studmail adressen van de faculteit. Dit betekent dat u wel e-mail
onder uw OP account kunt versturen maar niets kunt ontvangen. E-mail
die u verstuurd wordt zodanig bewerkt dat het lijkt alsof ze van uw
studmail account komen. E-mail die u verstuurd aan een OP account
wordt geforward naar het corresponderende studmail account.
Dit betekent dat u uw e-mail op {\tt elektron} moet lezen.
Zie ook paragraaf~\ref{domeinnamen}.
\index{pine|)}

\subsubsection{Logout}
\label{Logout}
Voordat u uitlogt wilt u nog even weten hoe laat het is,
en of er nog meer mensen op het systeem werken.
Hiervoor heeft u de commando's {\em date} en {\em who}\/:
\begin{verbatim}
   [idaard:op5/op5u9] date
   Thu Jul  3 12:45:24 METDST 1997
   [idaard:op5/op5u9] who
   antoon     pty/ttyv0    Jun 26 17:15
   op5u9      ttyp7        Jul  3 11:18
\end{verbatim}

Na deze {\em c-shell sessie} kunt u proberen uit te loggen.
Klik met de linker-muisknop op de paarse achtergrond.
Het {\em Utilities} popup-menu verschijnt.
Onderaan bevindt zich een een {\em Exit} submenu.
Hierin kun u kiezen voor ''{\em Yes, Realy Quit}\/''.
Als u hier de muisknop weer los laat wordt u uitgelogd.

\subsection{Belangrijke commando's om te onthouden}

Hieronder volgt een opsomming van de belangrijkste commando's die in het voorgaande aan bod zijn geweest.
\begin{itemize}
\item
\tool{ls, ls -a, ll} : listen van files.
\item
\tool{cat, more} : op het scherm laten zien van ASCII files.
\item
\tool{cp} : kopi\"eren van files en directories.
\item
\tool{rm} : verwijderen van files en directories.
\item
\tool{mv} : verplaatsen van files en directories.
\item
\tool{cd} : veranderen van directory.
\item
\tool{mkdir} : Maken van een subdirectory.
\item
\tool{lp} : printen van files op line printer.
\item
\tool{mkopr} : maken van een database voor het ontwerppracticum.
\end{itemize}

\cleardoublepage
