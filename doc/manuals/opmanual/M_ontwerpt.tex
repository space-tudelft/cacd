\section{Het IC Ontwerptrajekt}

\subsection{Inleiding}

Vanaf het moment dat iemand een idee krijgt tot het moment dat het produkt 
ook echt in de winkel verkrijgbaar is, 
moeten heel wat ontwerpstappen doorlopen worden. 
Dit ontwerptrajekt lijkt voor heel veel produkten sterk op elkaar. 
Zo'n algemeen geldend trajekt is systematisch afgebeeld in figuur~\ref{ontw}.

%figuur ontwerpen.eps
\begin{figure}[bth]
\centerline{\callpsfig{ontwerpen.eps}{width=.90\textwidth}}
\caption{Algemeen geldend ontwerptrajekt}
\label{ontw}
\end{figure}

Ook voor het ontwerp van een ASIC (Application Specific Integrated Circuit) geldt ongeveer dit trajekt.
Na het ontstaan van een idee moeten er specificaties opgesteld worden waaraan de schakeling moet voldoen 
en moeten er ook randvoorwaarden vastgesteld worden.
Hierna wordt de schake\-ling ontworpen. Eerst op circuitniveau en vervolgens de layout.
Deze layout wordt geprocessed en vervolgens wordt de ontstane chip getest.
Als alles goed is, wordt het sein gegeven om de chip in massaproductie te nemen.
Met deze laatste stap zijn we bij het uiteindelijke produkt gekomen. \\
In het ontwerppracticum doorlopen we dit gehele trajekt vanaf de specificaties tot aan het testen.
De nadruk zal echter vooral liggen op het ontwerpen van een IC,
dus het verkrijgen van een circuit en layout uit de specificaties en de randvoorwaarden.\\
Bij het ontwerpen van een IC staan ons computers, software en programmeerbare
hardware (voor ''rapid prototyping'') terzijde.
Deze hulpmiddelen zijn in onze tijd onmisbaar 
vanwege de zeer grote en nog steeds groeiende complexiteit van het hele IC-gebeuren. 
Ze zijn echter nutteloos als ze niet met verstand worden gebruikt. 
Bovendien blijft het echte ontwerpen nog steeds het werk van de ingenieur. \\
Om het ontwerpproces te stroomlijnen is het noodzakelijk enige afspraken te maken met betrekking 
tot de ontwerpstrategie en het gebruik van de hulpmiddelen. 
In dit hoofdstuk zullen daarom allereerst enige (ontwerp)regels gegeven worden, 
waarna vervolgens de tot uw beschikking staande hulpmiddelen aan bod komen. 

\subsection{Ontwerpregels}

\paragraph{Ontwerp hi\"erarchisch}
Hi\"erarchie is absoluut noodzakelijk om een goed overzicht over een ingewikkelde schakeling te kunnen behouden. 
Als vuistregel kan men stellen dat voor een goed overzicht niet meer dan 5 \`a 10 deelblokken 
op \'e\'en hi\"erarchisch niveau gedefinieerd moeten worden.
Ook de software is gebaseerd op een hi\"erarchisch ontwerp.
Bovendien hebben programma's als de automatische plaatser en bedrader de grootste moeite met grote "plat'' ontworpen circuits.\\
Een uitzondering op deze regel vormen de circuits die met "logische synthese'' gemaakt zijn.
Hiervoor is het argument van "het overzicht'' niet zo belangrijk. De begrenzing ligt dan bij de software.
Als vuistregel moet u zich dan houden aan niet meer dan 100 \`a 200 instances per hi\"erarchisch niveau.

\paragraph{Verricht eerst een goede verificatie van het volledige circuit m.b.v. simulatie en m.b.v. rapid prototyping,
voordat u begint met het maken van de layout}
In het ont\-werptrajekt zijn er talrijke analyse-stappen.
E\'en van de belangrijkste contr\^oles die u vaak zult doen, is (computer)simulatie van het ingevoerde circuit.
Een andere manier is rapid prototyping, waarbij 
speciale hardware (een FPGA) zodanig wordt geprogrammeerd dat deze zich gedraagt 
overeenkomstig de gedragsbeschrijving van het ontwerp.
Verricht deze vormen van verificatie voordat u begint met het maken van de layout.
Beide manieren vullen elkaar aan en maken het mogelijk fouten in
het ontwerp vroegtijdig op te merken en te herstellen.
De voordelen zijn kortere iteratie-lussen, wat inhoudt dat men minder ver terug moet in het 
ontwerptrajekt als iets niet goed blijkt te zijn. 
Hoewel een grondige verificatie vaak tijdrovend lijkt, 
blijkt in de praktijk dat deze tijdsinvestering zich ruimschoots terugverdient.

\paragraph{Hou rekening met de hardware bij maken van VHDL-beschrijvingen}
Hou bij het maken van een \smc{vhdl}-beschrijving er altijd rekening mee dat
de beschrijving een digitaal circuit voorstelt.
Hoewel het voordelig is om hoog-niveau \smc{vhdl}-beschrijvingen te maken 
(bijvoorbeeld c $<$= a + b voor de optelling van 2 integer getallen)
en deze door de logic synthesizer te laten vertalen 
naar een netwerk met logische poorten en flipflops,
moet altijd in ogenschouw worden genomen dat de \smc{vhdl}-beschrijving
uiteiendelijk wordt afgebeeld op hetzij een combinatorische schakeling, 
hetzij een sequenti\"ele schakeling volgens het Moore-model.
Wanneer de gemaakte VHDL beschrijving niet duidelijk genoeg is m.b.t.
het gedrag van de schakeling dan zal de logic synthesizer niet in staat
zijn de VHDL gedragsbeschrijving te vertalen naar een structuurbeschrijving,
of het gegenereerde circuit zal ongewenst/onvoorspelbaar gedrag vertonen. 
Meer informatie hierover is te vinden in
appendix~\ref{good_vhdl}.

\paragraph{Ontwerp besturingen volgens het Moore-model}
Besturingen of sequenti\"ele machines kunnen ontworpen worden volgens het Moore- of het Mealy-model. 
\index{Mealy}
Het verschil tussen deze twee ontwerpmethoden is dat bij het Moore-model de uitgangssignalen slechts 
afhankelijk zijn van de toestand van de schakeling, terwijl bij het Mealy-model de uitgangssignalen 
ook afhankelijk (kunnen) zijn van de ingangssignalen (zie voor meer uitleg appendix \ref{intro_seq_machine}).\\
In de praktijk heeft dit tot gevolg dat "Moore-besturingen'' minder storingsgevoelig zijn 
en gemakkelijker correct te ontwerpen zijn.
Een nadeel is de grotere hoeveelheid toestanden en dientengevolge
een groter beslag op het silicium-oppervlak.
Dit nadeel weegt echter (zeker tijdens het practicum) niet op tegen de voordelen.

\paragraph{Ontwerp synchroon}
\index{synchroon|(bold}
Synchroon ontwerpen betekent dat alle signalen in een schakeling na een tijd\-je stopgezet worden om 
vervolgens op een bepaald tijdstip weer verder te gaan. 
Dit is noodzakelijk om het overzicht te kunnen houden in de mengelmoes van signalen en hun vertragingstijden 
die anders zou ontstaan. 
Het (centrale) signaal dat aangeeft wanneer de schakeling weer verder mag, noemen we de klok.
Een ander voordeel van deze synchrone of "geklokte'' systemen is dat de problemen die altijd 
aanwezige spikes kunnen opleveren, zo goed als verwaarloosd kunnen worden: 
Op het moment dat de flipflops hun nieuwe waarde inlezen, 
zal hun ingangssignaal altijd stabiel zijn (mits de klokfrequentie laag genoeg gekozen wordt).\\
\index{synchroon|)}

\paragraph{Gebruik als geheugenelement alleen edge-triggered flipflops}
\index{flipflop!edge-triggered|bold}
Geheugenelementen kunnen hun nieuwe waarde aannemen tijdens een klokflank (edge)
of gedurende de hele periode dat het kloksignaal "hoog'' is (pulse).
De storingsgevoeligheid van zgn.\ edge-triggered flipflops is veel lager dan die van zijn pulse-triggered broertje.
Om deze reden wordt afgesproken dat tijdens het practicum alleen edge-triggered flipflops zijn toegestaan.
De {\tt oplib} (ontwerppracticum-library) bevat trouwens alleen flipflops van dit type.

\paragraph{Ga zorgvuldig om met het kloksignaal}
\index{flipflop!klokflank}
Het kloksignaal is het hart van de schakeling. 
Het is heel belangrijk dat bij alle flipflops de klokflank gelijktijdig aankomt, 
omdat dat het moment is dat de flipflops inlezen. 
In de praktijk zal dit echter nooit helemaal lukken.
Dit verschijnsel, dat ook wel "clock-skew'' \index{clock-skew} wordt genoemd, 
kan ervoor zorgen dat uw schakeling niet werkt, terwijl hij logisch gezien wel klopt. 
Ook simulatoren merken clock-skew vaak niet op. 
Het is dus zaak dat u bij het ontwerpen al met dit verschijnsel rekening houdt en zodanig te werk 
gaat dat er zo min mogelijk clock-skew ontstaat. 
Hiervoor zijn in de praktijk een paar regeltjes:
\begin {itemize}
\item
Gebruik geen logica in uw kloklijn.
\item
Buffer uw kloksignaal voldoende en gebruik in elke "tak'' van uw klok"boom'' eenzelfde aantal buffers.
Dit betekent dat tussen elke flipflop in uw circuit en het oorspronkelijke kloksignaal hetzelfde aantal buffers staat. \index{buffer}
\item
Belast elke buffer ongeveer even zwaar. 
\end{itemize}

\paragraph{Gebruik geen dynamische (basis)schakelingen}
Dynamische schakelingen gebruiken de altijd aanwezige capaciteit om tijdelijk informatie op te slaan.
Het bekendste voorbeeld van een dynamische schakeling is het dynamische RAM (Random Access Memory),
of kortweg DRAM dat tegenwoordig in de meeste computers gebruikt wordt.
Het voordeel van dynamische boven gewone "statische'' schakelingen is dat ze minder silicium-oppervlak in beslag nemen,
wat vooral bij grote repeterende structuren (bijvoorbeeld geheugens) belangrijk is.
Nadelen zijn de verplichte aanwezigheid van een meerfase-klok en de slechte testbaarheid bij lage frequenties.
Vanwege deze nadelen spreken we in het practicum af dat we geen dynamische geheugencellen of logica zullen gebruiken.

\paragraph{Gebruik alleen positieve logica}
Signalen kunnen positief en negatief gedefinieerd worden.
Voorbeelden van positieve definities zijn "lamp aan'' of "persoon gedetecteerd''.
Een veel gebruikt voorbeeld van een negatieve definitie is "niet-reset'' of $\overline{reset}$.
Dit laatste wordt vaak gedaan uit oogpunt van besparing: Het scheelt soms een invertor.
Het nadeel van negatieve logica is dat het niet intu{\ii}tief is.
In gedachten moet steeds weer ge{\ii}nverteerd worden.\\
In het practicum spreken we af dat we daarom bij voorkeur positieve logica zullen gebruiken. 

\paragraph{Plaats een buffer als een uitgang te zwaar belast wordt}
\index{buffer|bold}
Wanneer een uitgang belast wordt, wordt de "flank'' van het uitgaande signaal minder steil.
Enerzijds wordt uw schakeling hierdoor trager, anderzijds wordt er naar verhouding meer vermogen gedissipeerd.
In uw kloklijn heeft u daarnaast nog het probleem van vergroting van de clock-skew.\index{clock-skew}
Om deze effecten binnen de perken te houden, is het belangrijk dat u op tijd buffert.
Hiervoor worden bij de bibliotheekcellen de waarden van de ingangscapaciteit en de fanout vermeldt.
Als de belasting de fanout-waarde overschrijdt is het verstandig een buffer (b.v.\ buf40) te plaatsen.

\subsection{Hulpmiddelen}
\subsubsection{IC ontwerpsoftware}
In dit hoofdstuk komen 
de verschillende computerprogramma's voor het ontwerpen
en controleren van uw schakeling aan bod. 
Deze programma's
vormen het gereedschap (engels: \tool{tools}), net als een timmerman een
hamer en een boormachine gebruikt. Het is niet al te ingewikkeld om met
de computergereedschappen om te gaan. U zult er dan ook heel snel
mee vertrouwd raken. 
Het probleem van IC ontwerp is echter dat we
met heel veel componenten te maken hebben, die allemaal foutloos op
de chip gelegd moeten worden. De computer helpt u hierbij, maar 
de programma's zijn vaak niet zo slim en bevatten soms ook fouten.
Daardoor kan het voorkomen dat de computer hele rare en onverwachte 
dingen blijkt te doen. Hiervan moet u niet in paniek raken, want het is
heel normaal.

De computerhulpmiddelen die U gaat gebruiken om Uw schakeling te maken,
komen van verschillende bronnen. Gedeeltelijk zijn het commercieel
verkrijgbare programma's, zoals het door u in het eerste jaar al gebruikte
\smc{vhdl}-simulatie programma ModelSim,
alsmede het synthese programma van de firma Synopsys.
Het ontwerpsysteem \smc{ocean/nelsis} voor het maken van de
layouts is een hier op de universiteit zelf door studenten en promovendi
ontwikkeld systeem.\\
In dit hoofdstuk zullen we eerst gaan kijken welke stappen er nodig zijn om
een schakeling op Sea-of-Gates te implementeren.
Deze stappen zijn aangegeven in figuur \ref{desflow-window}.

\begin{figure}[htb]
  \centerline{\callpsfig{desflow.ps}{width=12cm}}
  \caption{De design-flow voor het IC ontwerptraject}
  \label{desflow-window}
\end{figure}

We kunnen hierin de volgende stappen onderscheiden:
\begin{itemize}
\item Allereerst moeten uiteraard de \smc{vhdl}-files worden gemaakt.
      \\
      Voor een bepaald onderdeel uit het ontwerp (een ''entity'') kan
      daarbij een gedrags (''behaviour'') beschrijving of een 
      structurele (''structural'') beschrijving gemaakt worden. 
      Aanwijzingen hiervoor kunnen verder worden gevonden in appendix~\ref{good_vhdl}.
      Een speciale file hierbij is verder de file met de \smc{vhdl}-code voor 
      een testbench. Met behulp van deze testbench kan de schakeling worden
      getest. De testbench bestaat in principe uit een instantiatie van
      de schakeling die moet worden getest, met testsignalen die op de
      ingangen worden aangesloten en uitgangssignalen, waaraan men de
      correctheid van de schakeling kan aflezen.
\item De gemaakte beschrijvingen moeten vervolgens op hun correctheid
      worden getest m.b.v. een \smc{vhdl}-simulator. Afhankelijk van de resultaten
      van deze simulatie kan worden verder gegaan met de volgende stap
      of, bij een niet correcte werking moeten de \smc{vhdl}-beschrijvingen
      worden aangepast. De verbeterde beschrijvingen moeten dan weer worden
      gesimuleerd enz. tot de simulaties het gewenste resultaat opleveren.
\item De volgende stap is het synthetiseren van de gemaakte gedragsbeschrijvingen.
      Hiervoor is, naast de \smc{vhdl}-beschrijvingen, ook een bibliotheek nodig
      van basis-circuits waaruit de te ont\-werpen schakeling kan worden
      opgebouwd.\\
      Bij de synthese kunnen de volgende stappen worden onderscheiden:
      \begin{itemize}
      \item De \smc{vhdl}-beschrijvingen worden omgezet in logische formules.
      \item Deze formules worden geoptimaliseerd.
      \item De geoptimaliseerde formules worden gerealiseerd m.b.v. de
            cellen uit de bibliotheek.
      \item Van de gesynthetiseerde schakeling wordt een structurele
            \smc{vhdl}-code gegenereerd, die dus uitsluitend bestaat uit
            bibliotheek-cellen en hun verbindingen.\\
            Tevens wordt een \smc{sls}-file gemaakt waarin het circuit wordt
            beschreven. Deze zal worden gebruikt om de layout te maken. 
      \end{itemize}
\item De gesynthetiseerde schakeling moet hierna weer worden gesimuleerd,
      Dit kan m.b.v. dezelfde testbench als die ook bij de simulaties
      van de oorspronkelijke \smc{vhdl}-code is gebruikt. Alleen moet
      via een configuratie statement worden aangegeven dat nu de
      gesynthetiseerde versie van het circuit moet worden gebruikt.\\
      Wanneer de simulaties uitwijzen, dat het gesynthetiseerde circuit
      ook de juiste werking vertoont, kan worden verder gegaan met de
      volgende stap; zo niet, dan moeten de oorspronkelijke \smc{vhdl}-beschrijvingen
      zodanig worden aangepast, dat wel het gewenste gesynthetiseerde
      circuit ontstaat.
\item De volgende stap is het vanuit de \smc{sls}-beschrijvingen genereren van de
      layout. Hiertoe kunnen aan de hand van de \smc{sls}-beschrijvingen de
      gebruikte cellen automatisch geplaatst worden door een placement-programma
      (hiervoor kan het programma \tool{madonna} of de \tool{row placer}. 
      Daarna kunnen de verbindingen tussen de cellen
      worden verzorgd door het routing-programma \tool{trout}.
\item Uit de layout van de schakeling kan weer
      \smc{vhdl}-code worden gegenereerd, die nu zal bestaan uit
      alle basis-cellen van de ontworpen schakeling en hun verbindingen.
      Deze circuit-extractie gebeurt m.b.v. het programma \tool{space}.
\item Ook deze beschrijving moet weer worden gesimuleerd m.b.v. de
      testbench, waarbij nu, via een configuration statement,
      moet worden aangegeven dat de geextraheerde versie van de scha\-ke\-ling
      moet worden getest. Verloopt deze simulatie succesvol, dan kan de volgende
      stap worden uitgevoerd. Is de simulatie niet succesvol, dan moet uit een
      nadere analyse van de resultaten van de simulatie worden afgeleid wat
      er fout is, en wat er aan het ontwerp moet worden veranderd.
\item Als voorlaatste stap kan er nog een simulatie worden uitgevoerd op
      het transistor niveau van de schakeling.
      Deze stap bestaat uit de volgende sub-stappen:
      \begin{itemize}
      \item Het transistor-circuit wordt uit de layout geextraheerd en de
           losse transistoren worden hieruit verwijderd.
     \item Vanuit de resultaten van een \smc{vhdl}-simulatie van het oorspronkelijke
           circuit worden een command-file en een referentie-file gemaakt.
           De command-file bevat de ingangssignalen die aan de schakeling
           moeten worden toegevoegd, en de referentie-file bevat de
           resultaten van de simulatie.
     \item De simulatie wordt nu uitgevoerd m.b.v. de command-file.
     \item Het resultaat van de simulatie, een result-file moet nu worden
           vergeleken met de referentie-file van de simulatie.
           Komen de twee overeen, dan is het ontwerp klaar en kan worden
           overgegaan tot de laatste stap. Komen de files niet overeen, 
           dan moet uit een nadere analyse van de verschillen worden
           afgeleid wat er fout is aan de schakeling en wat hieraan kan worden
           gedaan.
     \end{itemize}
\item Als laatste stap moet er nog een file worden gemaakt waarin de
      layout van de schakeling staat beschreven (de ldm-file).
\end{itemize}      
\subsubsection{Rapid prototyping}
Naast de IC ontwerp software zal er ook programmeerbare hardware beschikbaar
zijn, in de vorm van een Altera FPGA bord, waarmee het ontwerp ''run-time''
geverifieerd kan worden.
Deze vorm van verificatie maakt het
mogelijk om fouten uit het ontwerp te halen die met simulatie alleen
veel moeilijker te achterhalen zijn.
Terwijl bij simulatie de gebruiker op logisch niveau alle ingangsignalen moet 
specificeren en de uitgangsignalen moet interpreteren, 
zorgen bij rapid prototping de randapparaten ervoor dat de gebruiker 
op ''hoog niveau'' de invoer bepaalt en de uitvoer kan interpreteren.
Het is aan te raden verificatie m.b.v. rapid prototyping te verrichten
nadat de simulatie van de gedragsbeschrijving van het ontwerp succesvol is 
afgerond, en voordat met het maken van de layout van het ontwerp
wordt begonnen.

De volgende stappen zullen hierbij worden uitgevoerd:
\begin{itemize}
\item
Na een eerste verificatie met behulp van de \smc{vhdl}-simulator
wordt de \smc{vhdl} beschrijving van (een gedeelte van)
het ontwerp gesynthetiseerd voor de FPGA die aanwezig is op het Altera bord.
Dit gebeurt met het programma \tool{quartus}.
Gebruik hiervoor alleen behaviour en structural/circuit beschrijvingen,
en geen voor de Sea-of-Gates cellen bibliotheek gesynthetiseerde 
\smc{vhdl}-beschrijvingen.
Op Blackboard staat een template voor een \tool{quartus} project.
\item
De d.m.v. synthese met \tool{quartus} verkregen configuratie 
(bitstream) file wordt, eveneens met \tool{quartus},
gebruikt om de FPGA te configureren.
\item
Op het Altera bord is al aardig wat randapparatuur
(schakelaars, LEDs, LCD display, etc.) aanwezig 
welke gebruikt kan worden tezamen met de FPGA.
Eventuele andere randapparaten
(luidspreker, monitor, etc.) kunnen ook
met het bord worden verbonden.
\item
Het geprogrammeerde bord wordt ''run-time'' op verschillende manieren
uitgeprobeerd.
\item
Wanneer het ontwerp niet aan de eisen voldoet
wordt de oorspronkelijke \smc{vhdl} beschrijving aangepast
en wordt weer bij de eerste stap begonnen.
\end{itemize}
Merk op dat een werkend prototype op het FPGA bord niet automatisch betekend
dat de \smc{vhdl} beschrijving helemaal ''af'' is.
De ''hardware'' van de Sea-of-Gates chip zal zich over het algemeen anders gedragen
dan de hardware van het FPGA bord en een goede werking op het bord hoeft geen garantie
te zijn voor een goede werking van de chip.
Een simulatie van de voor de chip gesynthetiseerde \smc{vhdl} beschrijving
of (nog beter) een simulatie van de layout van de chip, zal 
hierover meer uitsluitsel geven.

\cleardoublepage

