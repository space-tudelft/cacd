\subsection{Testbaar ontwerpen}
\label{testbaar}
Nadat de schakelingen gefabriceerd zijn zullen zij door de groep getest worden. 
Het testen van de ontworpen schakeling na fabricage heeft een tweeledig doel: 
\begin{itemize}
\item
Nagaan of de schakeling aan de specificaties voldoet.
\item
Indien de schakeling niet aan de specificaties voldoet: 
proberen zo nauwkeurig mogelijk te weten te komen wat de reden is waardoor 
de schakeling niet voldoet,
zodat tijdens een eventueel herontwerp de schakeling wel zal kunnen werken.
\end{itemize}
Het is van groot belang dat al in een vroeg stadium tijdens het ontwerp
rekening wordt gehouden met het feit dat de schakeling later getest zal
worden.
Dit omdat in de eindfase van het ontwerp het meestal lastig is om de 
schakeling hiervoor nog aan te passen.
Daarom wordt in deze sectie besproken hoe het testen zal plaatsvinden
en aan welke eisen de schakeling (en de simulaties) moeten voldoen
om later op goede wijze te kunnen testen.

\subsubsection{De manier van testen bij het OP}
Het testen van de schakeling na fabricage wordt op de volgende 2 manieren uitgevoerd.  
In beide gevallen wordt gemeten aan 2 a 3 stuks afgemonteerde schakelingen in een DIL40 behuizing. 
\begin{itemize}
\item
De schakeling wordt aangesloten op een logic analyzer.
Via de logic analyzer worden logische signalen naar de ingangen van de
schakeling gestuurd en de logische signalen aan de uitgangen worden geobserveerd. 
Dit is vergelijkbaar met het simuleren van de schakeling tijdens het ontwerp,
maar nu met de echte hardware i.p.v. het (\smc{vhdl}) simulatie model.
Om te kunnen beoordelen of de schakeling werkt worden testvectors van 
een eerdere \smc{vhdl} simulatie gebruikt en wordt gekeken of de uitgangen 
van de werkelijke schakeling dezelfde resultaten geven als tijdens simulatie 
(aannemende dat de eerder \smc{vhdl} simulatie goede resultaten gaf).
Voor meer informatie over deze vorm van testen, zie appendix D van de OP handleiding.
De gehele schakeling dient op deze manier te worden getest, maar ook
de deelschakelingen dienen op deze
manier te worden getest.
\item
Aan de schakeling worden
de externe componenten aangesloten 
(schakelaars, oscillator kristal voor de klok, displays, luidsprekers etc) 
en de schakeling wordt real-time uitgeprobeerd.
\end{itemize}
\subsubsection{Overwegingen tijdens het ontwerp}
Vanwege de testbaarheid van het ontwerp en het eventuele localiseren van problemen, 
dient aan de volgende punten aandacht te worden geschonken tijdens het ontwerp:
\begin{itemize}
\item
Zorg dat de uitgangen van de schakeling, bij een gegeven set van ingangsignalen
en na de reset, altijd een bekende, vaste, reeks van waarden gaan doorlopen. 
Alleen dan namelijk kan de logic analyzer een vergelijking maken met de 
referentie set van uitgangsignalen zoals eerder verkregen via simulatie.
\item
Tijdens het testen met de logic analyzer (en ook tijdens simulatie !) 
kan slechts over een beperkt aantal klokcycli gesimuleerd worden: enkele 
miljoenen/honderdduizenden cycli tijdens simulatie en 65000 cycli tijdens 
testen met de logic analyzer. 
Dat wil dus zeggen dat wanneer een kloksignaal van 6.144 Mhz gebruikt wordt 
en men het gedrag van de schakeling na 1 minuut (real-time) wil bekijken 
er 60 x 6144000 is ongeveer 360 miljoen klokcycli nodig zouden zijn, wat dus veel te veel is.
Om dit probleem op te lossen kan men bijvoorbeeld oplossingen bedenken in de volgende richtingen:
(1) de schakeling een bepaalde mode geven waarin hij met een veel langzamere klok 
toch de gewenste toestanden doorloopt of 
(2) het mogelijk maken dat de schakeling van een bepaalde begintoestand start 
waarna de schakeling veel sneller de gewenste eindtoestand bereikt.
\item
Wanneer een schakeling niet goed werkt is aan alleen de uitgangssignalen 
vaak niet altijd te zien waar het probleem zit.  
Probeer daarom zoveel mogelijk ``interessante'' interne signalen op de pinnen van het 
IC aan te sluiten wanneer er nog pinnen beschikbaar zijn. 
Slecht een beperkt aantal signalen kan tijdens het testen bereikt worden worden: 
naast voedingsaansluitingen zijn er slechts 32 pinnen beschikbaar per schakeling voor in 
en uitgangen. 
Om te analyzeren wat er wel werkt en niet werkt en waar het probleem dus zit kan hier 
gedacht worden aan het demultiplexen en multiplexen van in en uitgangssignalen 
zodat via 1 pin meerdere ingangssignalen worden aangeboden of uitgangen worden gelezen. 
Dit kost echter weer wel extra pinnen voor het aansturen van de (de)multiplexers.
\end{itemize}
\subsubsection{Overwegingen tijdens het maken van testbenches}
Voor zowel de gehele schakeling als de verschillende deelschakelingen zal
minstens \'e\'en van de testbenches die gebruikt is tijdens simulatie, tevens 
(na conversie) gebruikt worden voor tijdens het testen met de logic analyzer.
Deze testbench zal aan de volgende eisen moeten voldoen:
\begin{itemize}
\item
Gebruik een simulatie die niet langer duurt dan 65000 klokcycli.
\item
Een ingangsignaal moet gedurende de hele simulatie ook ingangsignaal blijven. 
D.w.z. het is niet toegestaan dat voor bepaalde tijd een logische waarde opdrukt aan
een pin en de rest van de simulatie wordt het signaal ``losgelaten''.
\item
Bij de gespecificeerde ingangsignalen moeten na afloop van de reset altijd 
de gespecificeerde uitgangsignalen optreden 
(de uitgangsignalen mogen na de reset dus niet afhankelijk zijn van een 
toevallige willekeurige begintoestand waarin de schakeling zich bevindt).
\end{itemize}
