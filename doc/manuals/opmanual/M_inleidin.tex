\section{Inleiding}
In het eerste jaar van de studie Elektrotechniek krijgt de student allerlei
basisvaardigheden te leren waarvan het hem of haar niet altijd even duidelijk
 zal zijn wat zij of hij er in de latere beroepspraktijk mee zal kunnen doen.
In het ontwerppracticum krijgt de student de gelegenheid om, gebruikmakend van de in het eerste jaar verworven kennis iets 'concreets' te maken.
In het ontwerppracticum zal zij/hij ondervinden dat zij/hij ondanks haar/zijn beperkte kennis in staat is, in samenwerking met medestudenten, een complex elektrotechnisch produkt te kunnen
specificeren, ontwerpen en testen. Verder zal zij/hij in het practicum
 geconfronteerd worden met het feit dat een elektrotechnisch ingenieur
 niet louter individueel bezig is maar dat het kunnen samenwerken in teamverband essentieel is.
Hierdoor zal zij/hij een beter beeld gaan krijgen wat er na het afstuderen van haar of hem verwacht gaat worden. De verwachting is dat hierdoor het ontwerppracticum motiverend zal werken om de studie Elektrotechniek tot een goed einde te volbrengen.\\

Het ontwerppracticum is verdeeld over twee kwartalen in het tweede studiejaar
(1e + 3e of 2e + 4e). 
In het eerste c.q.\ tweede kwartaal wordt het produkt gespecificeerd en ontworpen.
In het derde c.q.\ vierde kwartaal wordt het produkt getest. Het tussenliggende  kwartaal wordt
gebruikt om het ontworpen produkt te laten maken.\\

In hoofdstuk 2 worden de doelstellingen van het ontwerppracticum geformuleerd.
Hoofdstuk 3 vermeldt een aantal regelingen en plichten waaraan men moet voldoen.
Hoofdstuk 4 geeft
 een gedetailleerde beschrijving wat er op de verschillende middagen wordt gedaan. Hierbij is aangegeven welke voorbereidingen gevraagd worden, welke werkvormen gebruikt worden. Tevens zijn hier de begeleidingsrollen, de beoordelingscriteria en de beschikbare infrastructuur beschreven.
Hoofdstuk 5 be\-schrijft een aantal aspecten van het werken in teamverband.
Hoofdstuk 6 schetst het beeld van het IC-ontwerptrajekt zoals dat gebruikt
 wordt in het ontwerpppracticum.
In hoofdstuk 7 zijn de inwerkopdrachten te vinden en is een voorbeeld
gegeven van hoe een inwerkopdracht wordt uitgewerkt, inclusief het
gebruik van de ontwerpomgeving.
Hoofdstuk 8 bevat de mogelijke groepsopdrachten. 
Het beschrijft het te ontwerpen produkt, de specificaties en de randvoorwaarden.
In de appendices vindt u o.a. een gedetailleerde beschrijving van de sea-of-gates 
chip met diverse belangrijke parameters 
en een beschrijving van de beschikbare cellenbibliotheek.

De hoofdstukken 1 t/m 7 dienen gelezen te worden voordat aan het
ontwerppracticum wordt begonnen.
Hoofdstukken 8 moet bestudeerd zijn voordat aan de groepsopdracht begonnen 
wordt.
De appendices zullen daar waar nodig gelezen moeten worden.

\cleardoublepage
