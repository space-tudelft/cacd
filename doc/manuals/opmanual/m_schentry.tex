\selectlanguage{dutch}
\title{Handleiding Schentry}
\maketitle
\index{schentry|bold}
\label{app_schentry}

\section{Inleiding}
In deze handleiding zal een beschrijving worden gegeven van de schematic-entry tool \tool{schentry},
waarvan bij het ontwerppracticum gebruik kan worden gemaakt.
De tool kan ook worden opgestart door een knop in de \tool{GoWithTheFflow} interface.
De \tool{schentry} tool kan bijvoorbeeld een hulpmiddel zijn bij het opdelen van de totale schakeling
in subschakelingen. De verbindingen tussen de blokken kunnen dan al worden
gemaakt, zodat kan worden voorkomen, dat de ene groep een signaal
veronderstelt, dat een andere groep niet levert. De architecturen van de
entities kunnen dan later worden toegevoegd.
De tool is in staat om \smc{vhdl}-code weg te schrijven naar de VHDL-directory.
Tevens kan men met de tool PostScript-files genereren voor documentatie doeleinden.

\section{Beschrijving van de interface}
De interface van de schematic-editor ziet er uit als getoond in figuur \ref{schen-window}
en bestaat uit 5 delen:
\begin{figure}[htb]
  \centerline{\callpsfig{schen_wdw.ps}{width=12cm}}
  \caption{De schematic-entry interface.}
  \label{schen-window}
\end{figure}
\begin{itemize}
\item Een menu-balk met de menu's '{\bf File}' en '{\bf Show}'.
In het '{\bf File}' menu bevinden zich commando's die eventueel ook direct met
een zogenaamde hotkey kunnen worden uitgevoerd.\\
Het '{\bf File}' menu bevat de volgende commando's:
   \begin{itemize}
   \item {\bf New}: Voor het maken van een nieuw circuit. Hierbij moet de naam
               van het nieuwe circuit worden opgegeven, waarna een leeg
               circuit wordt getoond in de edit-area van de interface.
   \item {\bf Read}: Voor het invoeren van een reeds eerder gemaakt circuit.
               Hierbij wordt een lijst getoond van reeds gemaakte circuits
               waaruit een keuze moet worden gemaakt door hierop te klikken.
   \item {\bf Save}: Voor het wegschrijven van het gemaakte circuit. De naam
                van de file, waarin het circuit wordt geschreven is de
                circuit-name met de extensie .cir. Dus een circuit
                met de naam 'adder' wordt in de circuit-directory geplaatst
                onder de naam 'adder.cir'.
   \item {\bf SaveAs}: Voor het wegschrijven van het circuit onder een andere naam.
   \item {\bf Quit}: Voor het beeindigen van het schematic-entry programma.
   \end{itemize}
In het '{\bf Show}' menu bevinden zich commando's die meer voor debugging zijn.
Daarmee kan men in de vorm van een lijst informatie verkrijgen over onderdelen van het circuit.
\item Een commando-balk waarop zich aan de linkerzijde een aantal knoppen
bevinden, waarvan er maar \'e\'en kan worden gekozen,
die bepaalt wat voor commando er kan worden uitgevoerd.
Deze commando's zijn:
\begin{itemize}
\item {\bf addCOMP}: Voor het toevoegen van een component aan de schakeling.
\item {\bf addPORT}: Voor het toevoegen van een terminal aan de schakeling.
\item {\bf addSPLIT}: Voor het toevoegen van een 'splitter' aan de schakeling.
\item {\bf addLABEL}: Voor het toevoegen van een label aan de schakeling.
\item {\bf cpyLABEL}: Voor het copieren van een label in de schakeling.
\item {\bf Connect}: Voor het maken van een verbinding in de schakeling.
\item {\bf Move}: Voor het verplaatsen van een item in de schakeling.
\item {\bf Delete}: Voor het verwijderen van een item in de schakeling.
\item {\bf Rename}: Voor het herbenoemen van een onderdeel van de schakeling.
\item {\bf Resize}: Voor het veranderen van de totale grootte van de schakeling.
\item {\bf addVDD}: Voor het toevoegen van een 'vdd' label aan de schakeling.
\item {\bf addVSS}: Voor het toevoegen van een 'vss' label aan de schakeling.
\end{itemize}
Deze commando's zullen nog nader worden uitgelegd in volgende paragrafen.\\
Aan de rechterzijde bevinden zich ook commando's van voornamelijk maak-knoppen.\\
Deze commando's zijn:
\begin{itemize}
\item {\bf zoomOUT}: Voor het uitzoomen van de schakeling.
\item {\bf zoomIN}: Voor het inzoomen op de schakeling.
\item {\bf makeCOMP}: Voor het genereren van een component van de gemaakte schakeling.
\item {\bf makeSLS}: Voor het genereren en parsen van een \smc{sls}-file van de gemaakte schakeling.
\item {\bf makeVHDL}: Voor het genereren en compileren van een \smc{vhdl}-file van de gemaakte schakeling.
\item {\bf makePS}: Voor het genereren van een PostScript-file van de gemaakte schakeling.
\item {\bf unRoute}: Voor het tekenen van de verbindingen in zijn oorspronkelijke vorm.
\item {\bf Route}: Voor het routen van de schakeling, d.w.z. het op een nettere manier tekenen van de verbindingen.
\end{itemize}
\item Een list-box aan de rechterzijde van de interface, waarin de componenten
      staan vermeld die bij het samenstellen van het circuit kunnen worden gebruikt.
\item Een info-window aan de onderzijde van de interface, waarin het
      programma mededelingen omtrend het verloop van het maken van het
      circuit kan schrijven.
\item Het midden-gedeelte wat wordt gebruikt om de tekening van het
      gemaakte circuit te tonen.
\end{itemize}

\section{Het initieren van een nieuw circuit}
Om een nieuw circuit te kunnen maken moeten de volgende stappen
worden uitgevoerd:
\begin{itemize}
\item Kies uit het '{\bf File}' menu de entry '{\bf New}'. Er verschijnt een window
      waarin de naam van het te maken circuit moet worden ingevuld.
\item Klik op \button{OK}. In het middengedeelte van \tool{schentry} verschijnt nu
      een gebied, waarin het circuit kan worden samengesteld.
\end{itemize}

\section{Het inlezen van een reeds bestaand circuit}
Om een reeds bestaand circuit in de editor in te lezen moet het volgende
worden gedaan:
\begin{itemize}
\item Kies uit het '{\bf File}' menu de entry '{\bf Read}'. Er verschijnt nu een lijst
      met reeds gemaakte circuits die in de directory 'circuits' zijn opgeslagen.
\item Klik op het circuit dat moet worden ingevoerd en op \button{OK} (of \button{Enter}).
    (Dubbel klikken kan ook.)
    Het circuit zal nu in het middengedeelte van de editor verschijnen.
\end{itemize}

\section{Het toevoegen van een component aan een circuit}
Om een component aan een te maken circuit toe te voegen moet het volgende
worden gedaan:
\begin{itemize}
\item Kies het commando \button{addCOMP}, door hierop te klikken.
\item Kies de gewenste component die moet worden toegevoegd uit de lijst
      met componenten aan de rechterzijde van \tool{schentry}, door hierop met de
      linker muisknop te klikken.
\item Druk nu de linker-muisknop in ergens binnen het circuit. Er verschijnt
      nu een omhullende van de toe te voegen component.
\item Beweeg deze omhullende naar de plaats waar de component moet worden
      geplaatst door het bewegen van de muis {\it met ingedrukte linker-muisknop}.
\item Laat de linker-muisknop los wanneer de omhullende op de juiste plaats staat.
	De component zal dan op deze plaats in de schakeling worden aangebracht.
\item Door opnieuw de linker-muisknop in te drukken wordt de component
      nogmaals toegevoegd, totdat een andere component of een ander
      commando wordt gekozen.
\end{itemize}

\section{Het toevoegen van een terminal aan een circuit}
Om een terminal (port) aan een te maken circuit toe te voegen moet het volgende
worden gedaan:
\begin{itemize}
\item Kies het commando \button{addPORT}, door hierop te klikken.
\item Klik de linker-muisknop binnen het circuit op de plaats waar de terminal
      moet komen. Wordt geklikt in de buurt van de linker-grens van het circuit,
      dan wordt de terminal een input-terminal en zal op de linker-grens
      worden geplaatst. Wordt geklikt in de buurt van de rechter-grens van het
      circuit, dan wordt de terminal een output-terminal en zal op de
      rechter-grens worden geplaatst.
      Zolang de muisknop ingedrukt wordt gehouden kan men de ligging nog bepalen.
\item Na het loslaten van de muisknop verschijnt een window waarin de volgende gegevens moeten worden
      opgegeven:
      \begin{itemize}
      \item De naam van de terminal.
      \item Het aantal bits dat de terminal breed is (default: 1).
      \item Het bidirectioneel zijn van de terminal, d.w.z. dat hij zowel als
            ingang als als uitgang kan worden gebruikt (default: niet bidirectioneel).
            Bidirectionele terminals mogen in het ontwerppracticum alleen
            worden gebruikt voor 'analoge'-signalen.
      \end{itemize}
\item Door nu op \button{OK} te klikken wordt de terminal geplaatst.
\end{itemize}

\section{Het toevoegen van een splitter aan een circuit}
Een splitter moet worden toegepast om een aantal bits uit een grotere
vector te selecteren.
Om een splitter aan een te maken circuit toe te voegen moet het volgende
worden gedaan:
\begin{itemize}
\item Kies het commando \button{addSPLIT}, door hierop te klikken.
\item Druk nu de linker-muisknop in ergens binnen het circuit. Er verschijnt
      nu een omhullende van de toe te voegen splitter.
\item Beweeg deze omhullende naar de plaats waar de splitter moet worden
      geplaatst door het bewegen van de muis {\it met ingedrukte linker-muisknop}.
\item Laat de linker-muisknop los wanneer de omhullende op de juiste plaats staat.
      Er verschijnt nu een window, waarin de volgende gegevens moeten
      worden gespecificeerd.
      \begin{itemize}
      \item De hoogste index uit de vector, die moet worden afgescheiden.
      \item De laagste index uit de vector, die moet worden afgescheiden.
      \item Het type: split op de ingang, d.w.z. dat de geselecteerde bits
            dienen als input voor de splitter of split op de uitgang, d.w.z.
            dat de geselecteerde bits de uitgang van de splitter vormen.
      \end{itemize}
\item Door nu op \button{OK} te klikken wordt de splitter in de schakeling
      aangebracht op de aangegeven plaats.
\end{itemize}

\section{Het toevoegen van een label aan een circuit}
Labels in het circuit zijn een alternatieve manier om verbindingen
te maken. Alle terminals in een circuit, die zijn voorzien van een label
met dezelfde naam worden beschouwd als aan elkaar verbonden.
Dit is bijvoorbeeld handig voor signalen die op vele plaatsen worden gebruikt
zoals kloksignalen.\\
Er zijn 3 speciale labels:
\begin{itemize}
\item Labels met de naam 'vdd' worden beschouwd als te zijn verbonden met een '1'.
\item Labels met de naam 'vss' worden beschouwd als te zijn verbonden met een '0'.
\item Labels met de naam 'gnd' worden beschouwd als te zijn verbonden met een '0'.
\end{itemize}
Om een label aan een te maken circuit toe te voegen moet het volgende
worden gedaan:
\begin{itemize}
\item Kies het commando \button{addLABEL}, door hierop te klikken.
\item Klik nu op een terminal van het circuit of op een terminal van een
      component. Er verschijnt nu een window waarin de naam van de label
      moet worden opgegeven.
\item Door nu op \button{OK} te klikken wordt de label aan de gekozen terminal
      aangebracht.
\end{itemize}
Met de knoppen \button{addVDD} en \button{addVSS} kan men snel speciale labels toevoegen.

\section{Het copieren van een label in een circuit}
Om een label in een te maken circuit te copieren moet het volgende
worden gedaan:
\begin{itemize}
\item Kies het commando \button{cpyLABEL}, door hierop te klikken.
\item Druk de linker-muisknop in boven de label die moet worden gecopieerd.
      Er verschijnt nu een omhullende van de label.
\item Verplaats deze omhullende door de muis te bewegen {\it met ingedrukte
      linker-muisknop} tot de cursor boven de terminal staat waarnaar de
      label moet worden gecopieerd.
\item Laat nu de linker-muisknop los. De label zal nu worden gecopieerd.
\end{itemize}

\section{Het maken van een verbinding in een circuit}
Om een verbinding te maken in een circuit moet het volgende
worden gedaan:
\begin{itemize}
\item Kies het commando \button{Connect}, door hierop te klikken.
\item Druk de linker-muisknop in boven de terminal die moet worden verbonden.
\item Beweeg de muis {\it met ingedrukte linker-muisknop}. Er verschijnt nu een
      lijn, waarvan het uiteinde met de muis meebeweegt.
\item Laat de linker-muisknop los boven de terminal waar de verbinding naar toe moet.
	De verbin\-ding zal nu worden gemaakt.
\end{itemize}

Voor het succesvol maken van een verbinding moet aan de volgende regels
worden voldaan:
\begin{itemize}
\item De te verbinden terminals moeten hetzelfde aantal bits hebben.
\item De verbinding moet beginnen bij een signaalbron (bijvoorbeeld input-terminal
      of output-terminal van een component) en mag niet op een bron eindigen.
      Voor inout terminals geldt, dat als ze eenmaal als bron zijn gebruikt
      ze verder als bron zullen worden behandeld. Zijn ze eenmaal als
      niet-bron gebruikt, dan kunnen ze niet meer als bron worden gebruikt.
	Bidirectionele terminals mogen alleen met bidirectionele terminals worden verbonden.
\end{itemize}

\section{Het verplaatsen van onderdelen in een circuit}
Om onderdelen in een circuit te verplaatsen moet het volgende
worden gedaan:
\begin{itemize}
\item Kies het commando \button{Move}, door hierop te klikken.
\item Druk de linker-muisknop in boven het onderdeel dat moet worden verplaatst.
      Er verschijnt nu een omhullende van het onderdeel dat zal worden
      verplaatst.
\item Beweeg de muis {\it met ingedrukte linker-muisknop} tot de omhullende de
      gewenste plaats heeft bereikt.
\item Laat de linker-muisknop los. Het onderdeel zal nu op de gekozen plaats
      worden neergezet.
\end{itemize}
Wordt de linker-muisknop ingedrukt buiten een circuit onderdeel, dan zal het
gehele circuit worden gekozen om verplaatst te worden.

\section{Het verwijderen van onderdelen uit een circuit}
Om onderdelen uit een circuit te verwijderen moet het volgende
worden gedaan:
\begin{itemize}
\item Kies het commando \button{Delete}, door hierop te klikken.
\item Klik met de linker-muisknop op het onderdeel dat moet worden verwijderd.
      Zolang de muisknop ingedrukt wordt gehouden kan men nog kiezen.
      Een geselecteerd onderdeel licht op in blauw.
      Dit onderdeel zal uit de schakeling verdwijnen, wanneer de knop wordt losgelaten.
      Hierbij gelden de volgende opmerkingen:
      \begin{itemize}
      \item Verbindingen kunnen zo alleen worden verwijderd, in niet geroute
            vorm. Dus geroute circuits moeten eerst ge-unrouted worden voor
            een verbinding kan worden verwijderd.
      \item  Onderdelen die niet zonder het te verwijderen onderdeel kunnen,
             zoals verbindingen naar een component of labels aan een terminal
             worden eveneens verwijderd.
      \end{itemize}
\end{itemize}

\section{Het veranderen van de afmetingen van een circuit}
Om de afmetingen van een circuit te veranderen moet het volgende
worden gedaan:
\begin{itemize}
\item Kies het commando \button{Resize}, door hierop te klikken.
\item Druk de linker-muisknop in er verschijnt nu en nieuwe omhullende van het
      totale circuit.
\item Beweeg de muis {\it met ingedrukte linker-muisknop}. De omhullende van
      het circuit beweegt mee met de muis.
\item Laat de linker-muisknop los wanneer het circuit de juiste afmetingen
      heeft verkregen. Dit zal dan de nieuwe afmeting van het circuit worden.\\
      Opmerkingen:
      \begin{itemize}
      \item Met de \button{zoomOUT} en de \button{zoomIN} knoppen kan de schaal nog worden
            veranderd voor het veranderen van de circuit-afmetingen.
      \item De afmetingen kunnen niet zodanig worden gemaakt, dat er onderdelen
            buiten het circuit vallen.
      \item De terminals aan de rechterzijde van det circuit worden meeverplaatst.
      \item Het verdient aanbeveling het circuit niet onnodig groot te maken,
            daar de tijd die het kost om het circuit te routen sterk
            afhankelijk is van de afmetingen hiervan.
      \end{itemize}
\end{itemize}
De afmeting van een component kan ook veranderd worden,
maar dit werkt door in het gebruik van dit component op alle plaatsen en alle circuits.

\section{Het routen van een circuit}
Om de verbindingen van een circuit wat beter en overzichtelijker te tekenen
kan het circuit gerouted worden. \\
Dit gebeurt door op de route-knop \button{Route} te klikken.
De verbindingen worden dan opnieuw gete\-kend, maar nu op een orthogonale manier.
Kan een bepaalde verbinding op deze manier niet worden getekend,
dan zal de oorspronkelijke verbinding worden getekend.
Soms kan dit worden verholpen door onderdelen te verplaatsen.\\
Overigens maakt het voor de werking van het circuit, of de vertaling naar
\smc{vhdl} niet uit of het circuit al dan niet goed gerouted is.

\section{Het un-routen van een circuit}
Om de verbindingen van een gerouted circuit weer op de oorspronkelijk ingevoerde
wijze te tekenen, om bijvoorbeeld een verbinding te verwijderen, kan het circuit weer
ge-unrouted worden.\\
Klik hiertoe op het \button{unRoute} commando.

\section{Het maken van een PostScript file van het circuit}
Door op het commando \button{makePS} te klikken wordt een PostScript-beschrijving van het
circuit gemaakt. De naam van de gemaakte file is 'circuit-naam'.ps.
Heet het circuit dus 'adder' dan zal een file 'adder.ps' worden gemaakt.

\section{Het maken van een component van het circuit}
Door op het commando \button{makeCOMP} te klikken wordt een component beschrijving toegevoegd
in component-directory. Hierdoor kan het circuit weer als component worden
gebruikt voor een grotere schakeling.

\section{Het maken van een \smc{sls}-beschrijving van het circuit}
Door op het commando \button{makeSLS} te klikken wordt een \smc{sls}-beschrijving van het circuit
gemaakt.\\
Deze wordt geplaatst in de directory SLS onder de naam gelijk aan de naam van
het circuit, met extensie .sls. Dus heet het circuit bijvoorbeeld 'adder' dan zal in
de subdirectory SLS een file met de naam 'adder.sls' worden geplaatst.\\
Opmerking: Bij het practicum zal de \smc{sls}-file meestal via de synthese van de
\smc{vhdl}-beschrijving worden gemaakt.

\section{Het maken van een \smc{vhdl}-beschrijving van het circuit}
Door op het commando \button{makeVHDL} te klikken wordt een \smc{vhdl}-beschrijving van het circuit
gemaakt.\\
Er verschijnt een window, waarin de naam voor een architecture moet worden
opgegeven (default is circuit).\\
Door hierna op \button{OK} te klikken worden twee \smc{vhdl}-files in de VHDL-directory
geschreven:\\
Een file, waarin de {\bf entity} staat beschreven, met een
file-naam gelijk aan de circuit-name, met de extensie .vhd
en een file, waarin de {\bf architecture} staat beschreven met de
naam 'circuit-naam'-'architecture-naam'.vhd.
Dus heet bijvoorbeeld het circuit 'adder' en is voor de architecture
de naam 'struct' gekozen, dan worden de file-namen dus:
'adder.vhd' en 'adder-struct.vhd'.
\cleardoublepage
