\section{Het schrijven van ''synthetiseerbare'' VHDL code}
\label{syn_vhdl}

\subsection{Inleiding}

Tijdens het ontwerpproces is het belangrijk om VHDL code te schrijven
die niet alleen de gewenste simulatieresultaten vertoont, maar die
ook geschikt om door de logic synthesiser te worden omgezet in
een schakeling die realiseerbaar is, en die daarna tijdens
simulatie nog steeds het gewenste gedrag vertoont.
In dit gedeelte zullen enkele richtlijnen worden gegeven
om dit zo succesvol mogelijk te laten verlopen.
Belangrijk hierbij is dat men zich tijdens het ontwikkelen
van de VHDL code realiseert hoe de structuur van de schakeling 
zal zijn waarvoor men het gedrag beschrijft.
Dit in ogenschouw nemende zullen we de volgende typen van VHDL 
code onderscheiden en apart behandelen.
\begin{itemize}
\item
Gedragsbeschrijving combinatorische schakeling
\item
Gedragsbeschrijving sequenti\"ele schakeling
\item
Structuurbeschrijving van een schakeling
\end{itemize}

\subsection{Gedragsbeschrijving combinatorische schakelingen}

Een gedragsbeschrijving van een combinatorische schakeling bestaat
uit een architecture body met daarin concurrent statements zoals
process statements en dataflow statements. 
Bij combinatorische schakelingen is het belangrijk om te zorgen
dat elk uitgangsignaal een waarde krijgt voor elke combinatie
van ingangsignalen.

Hieronder volgt een voorbeeld van een combinatorische schakeling 
met een process statement.
\begin{verbatim}
library IEEE;
use IEEE.std_logic_1164.ALL;

architecture behaviour of combinatorial is
begin
   -- a, b and c are input signals, y is output signal
   lbl1: process (a, b, c)
   -- All input signals must be in the sensitivity list
   begin
      -- Compute output value(s) based on input values
      -- Important: for every input combination, each 
      -- output should receive a value !
      if ((a = '1' or b = '1') and c = '0') then
         y <= '1';
      else
         y <= '0';
      end if;
   end process;
end behaviour;
\end{verbatim}

Hier is dezelfde combinatorische schakeling
op meer compacte wijze beschreven m.b.v. een dataflow statement:
\begin{verbatim}
architecture behaviour of combinatorial is
begin
   y <= (a or b) and (not c);
end behaviour;
\end{verbatim}

\subsection{Gedragsbeschrijving sequenti\"ele schakeling}

We gaan er van uit dat elke sequenti\"ele schakeling die we
ontwerpen beschreven wordt door een Moore machine 
zoals in figuur \ref{figuur:vhdlfsm}.
Zie ook figuur \ref{figuur:moore_mod}, waarbij $\delta$ en $\gamma$
zijn samen genomen in een ''combinatorial'' blok, en $Q$ gelijk is
aan het ''statereg'' blok.
\begin{figure}[bth]
\centerline{\callpsfig{vhdlfsm.eps}{width=0.60\textwidth}}
\caption{Blokschema voor het Moore model.}
\label{figuur:vhdlfsm}
\end{figure}
Bij het beschrijven van een sequenti\"ele schakeling maken we daarom 
onderscheid tussen een register gedeelte dat 
de toestand onthoudt (''statereg'') en een combinatorisch gedeelte dat 
de nieuwe toestand en de waarden van de uitgangsignalen berekent aan de hand
van de ingangsignalen en de huidige state (''combinatorial'').
Hoe de combinatorische schakeling ''combinatorial'' beschreven kan worden 
hebben we gezien in de voorafgaande sectie. 
Bij de beschrijving van het gedeelte dat de toestand onthoudt
moeten we ons realiseren dat we in de Sea-of-Gates cellenbibliotheek
de beschikking hebben over positive-edge triggered flipflops.
Deze flipflops zijn uitgevoerd zonder reset, met een synchrone
reset of met een asynchrone reset.
Ook op het Altera bord bevinden zich flipflops van deze types.
Wanneer ''state'' de huidige toestand weergeeft, ''new\_state''
de volgende toestand, ''RESET\_STATE'' de reset toestand, ''clk''
het klok signaal en ''res'' het reset signaal, en wanneer we
uitgaan van een synchrone reset, dan kunnen
we het gedeelte voor het opslaan van de toestand beschrijven met
\begin{verbatim}
   statereg: process (clk)
   begin
      if (clk'event and clk = '1') then
         if res = '1' then
            state <= RESET_STATE;
         else
            state <= new_state;
         end if;
      end if;
   end process;
\end{verbatim}

Een compleet voorbeeld voor een beschrijving voor een 
sequenti\"ele schakeling wordt dan:
\begin{verbatim}
library IEEE;
use IEEE.std_logic_1164.ALL;

architecture behaviour of fsm is
   type state_type is (STATE1, STATE2);
   signal state, new_state: state_type;
begin     
   statereg: process (clk)
   begin
      -- Because we use positive-edge triggered memory elements 
      -- (flipflops) with synchronous reset, we assign a new 
      -- state when clock goes to '1' 
      if (clk'event and clk = '1') then
         if res = '1' then
            -- assign reset state
            state <= STATE1;
         else
            state <= new_state;
         end if;
      end if;
   end process;
   combinatorial: process (state, input1)
   begin
      -- Assign output values based on state (for Moore machine
      -- the outputs are independent from the input signals) and
      -- compute new_state based on current state and input signals.
      -- Important: for all possible states, every output should 
      -- receive a value and new_state should be defined.
      case state is
         when STATE1 =>
            output1 <= '0';
            if (input1 = '0') then
               new_state <= STATE2;
            else  
               new_state <= STATE1;
            end if;
         when STATE2 =>
            output1 <= '1';
            new_state <= STATE1;
      end case;
   end process;
end behaviour;
\end{verbatim}

Wanneer we zouden kiezen voor een asynchrone reset dan 
zou het eerste process er als volgt uitzien:
\end{verbatim}
      if res = '1' then
         -- assign reset state
         state <= STATE1;
      elsif (clk'event and clk = '1') then
         state <= new_state;
      end if;
\end{verbatim}

Een ander voorbeeld van een 
sequenti\"ele schakeling die op bovenstaande wijze beschreven is,
is de hotel schakeling die als voorbeeld wordt gegeven
bij de inwerkopdrachten.
Nog een ander voorbeeld is een 4 bits teller zoals die hieronder is gegeven.
Merk op dat in dit geval de toestand wordt gevormd door 
de 4 bits van de std\_logic\_vector count
en dat de uitgangen gevormd worden door de 4 bits van
de vector a die hier rechtstreeks van worden afgeleid.
\begin{verbatim}
library IEEE;
use IEEE.std_logic_1164.ALL;
use IEEE.std_logic_arith.ALL;
use IEEE.std_logic_unsigned.ALL;

architecture behaviour of counter is
   signal count, new_count : std_logic_vector(3 downto 0);
begin
   statereg: process (clk)
   begin
      if (clk'event and clk = '1') then
         if reset = '1' then
            count <= (others => '0');
         else
            count <= new_count;
         end if;
      end if;
   end process;
   combinatorial: process (count, enable)
   begin
      a <= std_logic_vector(count);
      if (enable = '1') then
         new_count <= count + 1;
      else
         new_count <= count;
      end if;
   end process;
end behaviour;
\end{verbatim}

\subsection{Structuurbeschrijving van een schakeling}

Een structuurbeschrijving kan gezien worden als een
fysiek netwerk/circuit van componenten en de verbindingen
daartussen.
Afhankelijk van welke componenten er in zijn opgenomen
zal de structuurbeschrijving een combinatorische
of een sequenti\"ele schakeling beschrijven.
De componenten zijn representaties voor entities en
architecturen van andere delen
uit het ontwerp.
Eventueel kunnen kunnen de componenten ook
cellen uit de Sea-of-Gates cellenbibliotheek zijn,
wanneer de beschrijving alleen voor de Sea-of-Gates chip 
bedoeld is en niet op het Altera bord wordt afgebeeld.
Het is niet nodig om een structuurbeschrijving te synthetiseren.
Wanneer ook alle componenten op structuurniveau bekend zijn, 
dan kan direkt vanuit de structuurbeschrijving een layout 
aangemaakt worden.

Hieronder volgt een voorbeeld van een structuurbeschrijving.
Naast het tekstueel invoeren van de VHDL code kan een
VHDL structuurbeschrijving ook op eenvoudige wijze worden
aangemaakt met een schematic editor zoals Schentry.
\begin{verbatim}
library IEEE;
use IEEE.std_logic_1164.ALL;

architecture circuit of network is
   -- First the declaration of the components used
   component comp1
      port (x: in std_logic; y: out std_logic);
   end component;
   component comp2
      port (p, q: in std_logic; r: out std_logic);
   end component;
   -- second the declaration of the nets (internal nodes)
   signal net1, net2, net3 : std_logic;
begin
   -- connections between nets and pins
   output1 <= net2;
   -- instances of components and their connections 
   comp1_1: comp1 port map (x=>net1, y=>net3);
   comp1_2: comp1 port map (x=>net2, y=>net1);
   comp2_1: comp2 port map (p=>net3, q=>input1, r=>net2);
end circuit;
\end{verbatim}
\cleardoublepage
