\selectlanguage{dutch} 
\title{Handleiding Design\_flow}
\maketitle

\section{Inleiding}
Om de verschillende programma's die nodig zijn om de \smc{vhdl}-beschrijvingen te
genereren, te simule\-ren, te synthetiseren en om hieruit uiteindelijk de
layout te maken te laten samenwerken is het programma \tool{design\_flow}
gemaakt, van waaruit de verschillende programma's hiervoor kunnen
worden aangeroepen. De aangeroepen programma's
zoals \tool{seadali}, \tool{madonna} etc. worden in aparte handleidingen besproken.

Het programma \tool{design\_flow} wordt gestart door het typen van het commando:
\begin{verbatim}
      design_flow &
\end{verbatim}
Wanneer zich nog geen database bevindt in de directory waar \tool{design\_flow}
wordt opgestart, zal het programma eerst vragen of zo'n database moet worden
aangemaakt.

\section{De gebruikte directories}
Het programma maakt eventueel de volgende database directories aan, wanneer deze
nog niet aanwezig zijn:
\begin{itemize}
\item {\bf VHDL}: waarin alle \smc{vhdl}-files worden geplaatst.
\item {\bf SLS}: waarin alle tijdens het ontwerp gegenereerde \smc{sls}-files, die
            nodig zijn voor het \smc{nelsis} systeem worden geplaatst.
\item {\bf ADB}: waarin alle .db-files worden geplaatst, die tijdens de
            synthese van de circuits worden gegenereerd door de synthesize
            software, en die deze software zelf ook weer gebruikt.
\item {\bf components}: waarin de netwerkcomponenten worden geplaatst, die door
                  het schematic-entry programma kunnen worden gebruikt en ook
                  door het programma \tool{design\_flow} om op een gemak\-kelijke
                  wijze components aan de architecture van een circuit toe
                  te voegen.
\item {\bf circuits}: waarin de circuit beschrijvingen worden geplaatst die worden
                gemaakt door het schematic-entry programma \tool{schentry}.
\item {\bf work}: een working directory voor de \smc{vhdl} simulatie-software, waarin
            deze software de gecompileerde versies van de gemaakte
            \smc{vhdl}-beschrijvingen kan plaatsen.
\item {\bf test}: een working directory voor de backend software \smc{ocean/nelsis}, waarin
            deze software de circuit en layout beschrijvingen kan plaatsen.
\item {\bf syn\_work}: de working directory voor de synthese-software, waarin
            deze software de gecompileerde versies van de gemaakte
            \smc{vhdl}-beschrijvingen kan plaatsen.
\end{itemize}

\section{Het main-window}
Nadat het programma \tool{design\_flow}
de controle van de bovenstaande directories heeft afgerond,
verschijnt een window als aangegeven in figuur \ref{main-window}.
De figuur toont in het grafische midden-gedeelte links de hierarchische
samenhang tussen de verschillende entities die in het ontwerp aanwezig zijn
en rechts de verschillend de representaties die er voor alle entities
aanwezig zijn.
Maar deze delen kunnen ook leeg zijn als er niet eerder is gewerkt op deze plaats.

\begin{figure}[htb]
  \centerline{\callpsfig{design_flow_opstart.eps}{width=12cm}}
  \caption{Het main-window van het programma \tool{design\_flow}}
  \label{main-window}
\end{figure}

Het main-window bestaat uit de volgende delen:
\begin{itemize}
\item Aan de bovenzijde de info-balk wordt aangegeven welke database
      met \tool{design\_flow} geopend is.
\item Hieronder bevindt zich de menubalk met een aantal menu's voor het lezen van
     \smc{vhdl} files, enige hulpfuncties en voor het instellen van de letter-grootte van het
      programma \tool{design\_flow}.
\item Aan de linkerzijde bevinden zich de commando-knoppen, waarmee de
      verschillende commando's voor het maken van de diverse beschrijvingen
      kunnen worden uitgevoerd.
\item Aan de onderzijde bevindt zich een window, waarin het programma
      mededelingen kan plaatsen, bijvoorbeeld over commando's die worden of zijn
      uitgevoerd.
\item In het grote midden-gedeelte van figuur \ref{main-window} wordt op een grafische wijze weergegeven
      welke packages, entities, architectures en configurations zich in de '\smc{vhdl} lib'
      (de database van de simulator) bevinden.
      Door de muis op een afgebeeld item te houden en de linker-muisknop in te drukken
      verschijnt een menu, waarmee operaties op het betreffende item
      kunnen worden uitgevoerd. Deze menu's worden nader besproken in de
      hoofdstukken over de package-, entity-, architecture- en configuration-menu's.
\end{itemize}

\section{De menubalk}
In de menubalk bevinden zich een drietal menu's, te weten:
\begin{itemize}
\item {\bf Het File-menu}: In dit menu
bevinden zich commando's voor het lezen en importeren en verwij\-deren van files.
Tevens de commando's voor het aanmaken van een nieuw entity en package.
\item {\bf Het Utilities-menu}: In dit menu bevinden zich hulp commando's.
Voor het herstellen van de huidige directory, het kopi"eren van files en het printen.
\item {\bf Het Fonts-menu}: Met dit keuze-menu kan de letter-grootte worden gewijzigd.
\end{itemize}
De drie menu's worden in de nu volgende paragrafen nader besproken.

\subsection{Het File-menu}
In dit menu bevinden zich de volgende sub-menu's:
\begin{itemize}
\item {\bf Read vhdl\_file}: Wanneer dit commando wordt gekozen verschijnt
                       er een lijst met alle \smc{vhdl}-files, die in de \smc{vhdl}-directory
                       aanwezig zijn. Wanneer een file wordt gekozen
                       verschijnt er een edit-window, waarin de file is
                       geplaatst en waarin deze file kan worden veranderd.
                       Voor een verdere beschrijving van het \smc{vhdl} edit window,
                       zie de sectie ''het \smc{vhdl} edit window''.
\item {\bf Import vhdl\_file}: Wanneer dit commando wordt gekozen verschijnt
               er een browser-window waarin een file kan worden gekozen.
               Deze file wordt gekopieerd naar de \smc{vhdl}-directory.
\item {\bf Import vhdl\_dir}: Wanneer dit commando wordt gekozen verschijnt
               er een browser-window waarin een directory kan worden gekozen.
               Alle \smc{vhdl}-files die in die directory staan
               worden gekopieerd naar de \smc{vhdl}-directory.
\item {\bf New entity}: Via dit commando kan een nieuwe entity worden
                         gegenereerd.
               Bij keuze van dit item verschijnt het window van
               figuur \ref{newcell-window}.
               \begin{figure}[htb]
               \centerline{\callpsfig{newcell_wdw.ps}{width=5cm}}
               \caption{New entity-window van het programma \tool{design\_flow}}
               \label{newcell-window}
               \end{figure}
               De naam van de entity moet nu worden ingevuld en de
               namen van de terminals ({\it ports}).
               Deze kunnen in het tekst-window worden getypt.
               Bij iedere terminal kan voorts
	       een {\it mode} ({\bf in}, {\bf out} of {\bf inout})
	       worden opgegeven en een aantal bits ({\it width}).
	       De default mode is {\bf in} of de mode van de vorige port.
	       De default width is altijd 1 bit.
	       Indien een andere width nodig is, kan default mode worden
	       aangegeven met een min-teken.
               De port gegevens moeten worden gescheiden door een \button{Tab}
	       en iedere regel moet worden afgesloten met \button{Enter}.
               De terminals worden default van het type 'std\_logic' gemaakt.
               Wordt na het invoeren van de hierboven genoemde gegevens
               op \button{OK} geklikt, dan wordt een edit-window gestart, waarin
               de gegenereerde code wordt getoond. Eventueel kan de code
               hierin nog worden aangepast.
               De file-naam die aan de beschrijving wordt gegeven
               is gelijk aan de naam van de entity met de extensie .vhd.
               Wordt op \button{Cancel} geklikt, dan verdwijnt het New\_entity-window
               en gebeurt er verder niets.
\item {\bf New package}: Door dit item te kiezen wordt er een basis-structuur voor
                   een \smc{vhdl}-package gegenereerd.
                  Bij keuze van dit item verschijnt het window van
                  figuur \ref{newpack-window}.
                  \begin{figure}[htb]
                  \centerline{\callpsfig{newpack_wdw.ps}{width=6.5cm}}
                  \caption{New package-window van het programma \tool{design\_flow}}
                  \label{newpack-window}
                  \end{figure}
                  \\
                  De naam van het package moet nu worden ingevuld.
                  Wordt na het invoeren van de naam
                  op \button{OK} geklikt,  dan wordt een edit-window gestart, waarin
               de gegenereerde code wordt getoond. Eventueel kan de code
               hierin nog worden aangepast.
                 De file-naam die aan de beschrijving wordt gegeven
                 is gelijk aan de naam van het package met achtervoegsel \_pkg
                 en extensie .vhd.
                 Wordt op \button{Cancel} geklikt, dan verdwijnt het New\_package-window
                 en gebeurt er verder niets.
\item {\bf Remove vhdl\_file}: Wanneer dit commando wordt gekozen verschijnt
                       er een lijst met alle \smc{vhdl}-files, die in de \smc{vhdl}-directory
                       aanwezig zijn. Wordt een van de getoonde files
                       hieruit gekozen, dan wordt deze file uit de \smc{vhdl}-directory
                       verwijderd.
\end{itemize}

\subsection{Het Utilities-menu}
In dit menu bevinden zich de volgende sub-menu's:
\begin{itemize}
\item {\bf Restore\_Cwd}: Door dit item te kiezen, wordt de directory waarin wordt
                    gewerkt weer terug gezet naar de directory waarin het
                    programma \tool{design\_flow} is gestart. Tijdens het werken met
                    het genoemde programma wordt soms van directory gewisseld.
                    In sommige gevallen, bijvoorbeeld wanneer fouten
                    optreden tijdens het uitvoeren van sub-commando's,
                    kan het voorkomen, dat de directory niet correct wordt
                    teruggezet. In dat geval biedt dit commando uitkomst.
\item {\bf Fix\_sealib}: Wanneer er foutmeldingen zijn over de library sealib of seadif, dan kan
                    met de commando er een reset worden gedaan waardoor het probleem
                    waarschijnlijk verholpen wordt.
\item {\bf Copy\_cell}: Via dit commando kunnen de gegevens die via \tool{design\_flow} in
                  een andere directory zijn gemaakt naar de huidige directory
                  worden gekopieerd.
                  Wanneer dit commando wordt gekozen verschijnt het window van
                  figuur \ref{copy-window}.
                  \begin{figure}[htb]
                  \centerline{\callpsfig{copy_wdw.ps}{width=8cm}}
                  \caption{Het Copy\_cell-window van het programma \tool{design\_flow}}
                  \label{copy-window}
                  \end{figure}
                  \begin{figure}[htb]
                  \centerline{\callpsfig{copy_wdw1.ps}{width=8cm}}
                  \caption{Ge"expandeerd Copy\_cell-window van het programma \tool{design\_flow}}
                  \label{copy-window1}
                  \end{figure}
                  Kies in dit menu de directory van waaruit gekopieerd moet
                  worden via de \button{Browse} button of door de directory-naam in de
                  bovenste entry te typen.\\
                  Door hierna op de \button{OK} button te klikken expandeert het window
                  tot het window van figuur \ref{copy-window1}.
                  Hierin kan een keuze worden gemaakt uit de 'working directory'
                  waaruit moet worden gekopieerd en de entity hieruit.\\
                  Bij een andere keuze zullen de betreffende files om te
                  kopi"eren automatisch worden aangepast.
                  De keuze of de file al dan niet moet worden gekopieerd kan
                  worden gemaakt door op de naam te klikken.\\
                  De default waarden die zijn ingesteld voldoen in de meeste
                  gevallen.\\
                  Ook kan een keuze worden gemaakt van welke \smc{nelsis} database
                  gekopieerd moet worden en welk deel hiervan: circuit, layout
                  of beide.\\
                  Door na de instellingen gemaakt te hebben op \button{Copy} te
                  klikken worden de aangegeven files en delen van de \smc{nelsis}
                  database gekopieerd. \smc{vhdl}-files worden tevens gecompileerd.\\
                  In het onderste gedeelte van het Copy\_cell-window verschijnen de
                  eventuele foutmel\-ding\-en en meldingen over wat het
                  programma gekopieerd heeft.
\item {\bf Print\_Layout}: Via dit commando kan een PostScript-file worden gemaakt van de
                     layout van een cell.
                    Wanneer dit commando wordt gekozen verschijnt het window van
                    figuur \ref{pr-layout-window}.
                    \begin{figure}[htb]
                    \centerline{\callpsfig{pr_layout_wdw.ps}{width=6cm}}
                    \caption{Print\_Layout-window van het programma \tool{design\_flow}}
                    \label{pr-layout-window}
                    \end{figure}
                    Door in dit window op \button{cell} te klikken verschijnt een lijst
                    van layout-cellen waarvan de PostScript-file kan worden gemaakt.
                    Door op een naam te klikken wordt de betreffende cell gekozen.
		    Voorts kan nog worden gekozen voor {\it landscape} of {\it portrait} orientatie.\\
                    Door op \button{MakeEps} te klikken wordt de PostScript-file gemaakt. Wordt
                    op \button{Cancel} ge\-klikt, dan verdwijnt het Print\_layout-window
                    zonder dat er een file wordt gemaakt.
                    De naam waaronder de PostScript-file wordt weggeschreven is de
                    naam van de cell met de extensie .eps.
\end{itemize}

\subsection{Het Fonts-menu}
Wanneer dit menu wordt gekozen verschijnen er een viertal letter-groottes, namelijk
{\it Small}, {\it Normal}, {\it Large} en {\it XLarge} waaruit een keuze gemaakt kan worden
door op het betreffende item te klikken.\\
De letter-grootte in alle windows van het programma \tool{design\_flow} zal dan
aan de gekozen grootte worden aangepast.

\section{De commando-knoppen}
Aan de linkerzijde van het \tool{design\_flow} window bevinden zich de commando-knoppen,
waarmee de verschillende commando's voor het maken van de diverse beschrijvingen kunnen worden uitgevoerd.
Het betreft de volgende knoppen:
\begin{itemize}
\item {\bf SchematicEntry}: Voor het starten van de schematic-editor.
\item {\bf Compile\_new}: Voor het compileren van nieuwe \smc{vhdl}-files.
\item {\bf Recompile\_lib}: Voor her-compilatie van oude onderdelen.
\item {\bf Add\_bondbar}: Voor het toevoegen van een bondbar.
\item {\bf Add\_bondleer}: Voor het toevoegen van een bondleer.
\item {\bf Make\_cmd\_file}: Voor het maken van een .cmd-file.
\item {\bf Make\_ref\_file}: Voor het maken van een .ref-file.
\item {\bf Simulate\_dac}: Voor het simuleren van een \smc{dac}.
\item {\bf Compare}: Voor het vergelijken van simulatie resultaten.
\item {\bf Quit}: Voor het afsluiten van het programma \tool{design\_flow}.
\end{itemize}
De commando-knoppen worden in de nu volgende paragrafen nader besproken.

\subsection{SchematicEntry}
Door het aanklikken van dit commando wordt de schematic-editor \tool{schentry} gestart,
waarmee op grafische wijze \smc{vhdl} structuur-beschrijvingen van een
circuit kunnen worden gemaakt.
Zie voor een beschrijving de handleiding van \tool{schentry}.

\subsection{Compile\_new}
Door het aanklikken van dit commando worden alle files uit de \smc{vhdl}-directory
die nog niet gecompileerd zijn, gecompileerd.

\subsection{Recompile\_lib}
Door het aanklikken van dit commando worden alle onderdelen van de
'working-library' die niet meer up-to-date zijn opnieuw gecompileerd.

\subsection{Add\_bondbar}
Een bondbar is een omgeving waarbinnen de schakeling van de hoofdopdracht
moet worden geplaatst. Zie ook de betreffende bibliotheek cell.
Met dit commando wordt een circuit aangemaakt bestaande uit een bondbar
en de schakeling die binnen de bondbar cell moet worden geplaatst.
Er zijn 32 aansluitpinnen waarmee de schakeling kan worden verbonden.

Wanneer dit commando wordt gekozen verschijnt het window van
figuur \ref{bondbar-window}.
\begin{figure}[htb]
\centerline{\callpsfig{bondbar_wdw.ps}{width=8cm}}
\caption{Het Add\_bondbar-window van het programma \tool{design\_flow}}
\label{bondbar-window}
\end{figure}
Hierin dient uit de lijst van circuits het circuit dat in de bondbar moet
worden geplaatst worden gekozen, door hierop te klikken.
In de midden van het window staan de terminals van de bondbar schakeling.
Nadat een circuit is gekozen verschijnen aan de rechterzijde van het
Add\_bondbar-window de terminals van het gekozen circuit, met de aanduiding
of het een {\bf ingang}(I), een {\bf uitgang}(O) of de terminal van mode {\bf inout}(B) is.
De verbindingen tussen de schakelingen worden
                   zodanig gemaakt dat de terminals in dezelfde rij worden
                   verbonden. De aangegeven verbindingen kunnen op de volgende
                   manier worden gewijzigd:
                   \begin{itemize}
                   \item Kies de terminal uit de linker-kolom waarvan de
                         aansluiting moet worden gemaakt of gewij\-zigd, door
                         hierop te klikken. Deze terminal wordt dan als
                         geselecteerd aangegeven.
                   \item Klik vervolgens op de terminal uit de rechter-kolom
                         waarop de geselecteerde terminal uit de linker-kolom
                         moet worden aangesloten. Deze terminal zal dan
                         geplaatst worden achter de ge\-se\-lecteerde terminal
                         in de linker-kolom. De terminal die oorspronkelijk
                         achter de geselecteerde terminal stond wordt verplaatst
                         naar de plaats waar de gekozen terminal uit de
                         rechter-kolom oorspronkelijk stond.
                   \item Uit de linker-kolom wordt vervolgens de volgende
                         terminal geselecteerd, om te worden aange\-sloten en
                         waarvoor dan weer uit de rechter-kolom een
                         terminal kan worden geselecteerd enz.
                   \end{itemize}
De naam voor het totale circuit wordt default gelijk gemaakt aan de naam van
het oorspronke\-lij\-ke circuit met hieraan '\_bb' toegevoegd.
Deze naam kan eventueel worden gewijzigd.
Door het klikken op \button{OK} wordt de schakeling met de bondbar aan het circuit
gedeelte van de \smc{nelsis} database toegevoegd.\\
Tevens wordt er een .buf-file gemaakt onder de naam
{\it circuit}.buf, waarbij {\it circuit} staat voor de circuit-naam,
van het gemaakte circuit.
De .buf-file bevat de informatie hoe de schakeling is aangesloten op de
bondbar cell en het soort terminal (input, output of direct).
Wordt op \button{Cancel} geklikt, dan verdwijnt het
window zonder dat er verder iets gebeurt.

\subsection{Add\_bondleer}
Dit commando is gelijksoortig aan het Add\_bondbar commando.
In plaats van een bondbar cel wordt hierbij echter een bondleer cel gebruikt,
die veel kleiner is en slechts 6 vrije aansluitpinnen heeft.
Voor de naam van het nieuwe circuit wordt de oorspronke\-lij\-ke circuit naam
gebruikt met daaran '\_bl' toegevoegd.
Voor het gebruik van het Add\_bondleer-window, zie het voorafgaande commando.

\subsection{Make\_cmd\_file}
Met dit commando kan een .cmd-file worden gegenereerd voor een sls-simulatie.
Voordat dit commando wordt gebruikt moet eerst via de \smc{vhdl}-simulator
een .lst-file zijn gemaakt van de simulatie waarvoor een .cmd-file moet
worden gegenereerd.
Dit kan op de volgende wijze:
\begin{itemize}
\item Kies nadat de simulatie is gedaan het commando {\bf Make\_list\_file} uit het
      {\bf Tools}-menu van het wave\_window. Er verschijnt nu een window waaruit
      een entity(architecture) is te kiezen.
\item Kies een entity(architecture) en klik \button{OK}. Er verschijnt nu een window
      waarin men de naam van de te schrijven lst-file kan opgeven.
\item Geef een naam en klik \button{Save}. De list-file zal nu worden gemaakt.
\end{itemize}

Wanneer het commando Make\_cmd\_file is gekozen verschijnt het window van
figuur \ref{mk-cmd-window}.

\begin{figure}[htb]
\centerline{\callpsfig{mkcmd_wdw.ps}{width=10cm}}
\caption{Het Make\_cmd\_file-window van het programma \tool{design\_flow}}
\label{mk-cmd-window}
\end{figure}

In dit window zijn onder het {\bf File}-menu de volgende commando's geimplementeerd:
\begin{itemize}
\item {\bf Read}: Met dit commando wordt een reeds bestaande .cmd-file in het
	tekst-window zichtbaar gemaakt.
\item {\bf Generate from}: Met dit commando kan de .lst-file worden gelezen, waaruit een
            .cmd-file moet worden gemaakt. De hieruit gegenereerde code
            zal zichtbaar worden gemaakt in het tekst gedeelte van het
            window.
\item {\bf UpdateForBondbar}: Met dit commando wordt m.b.v. de .buf-file de .cmd-file
             aangepast voor gebruik bij de in een bondbar
             geplaatste schakeling.
\item {\bf Write}: Met dit commando wordt de getoonde code weggeschreven naar een
             file met dezelfde naam als de .lst-file, echter nu met de
             extensie .cmd. Dus is de .lst-file {\it gerrit.lst}, dan wordt
             de gegenereerde .cmd-file {\it gerrit.cmd}.
\item {\bf Quit}: Door de keuze van dit commando wordt het Make\_cmd\_file window
              weer verwij\-derd.
\end{itemize}

\subsection{Make\_ref\_file}
Met dit commando kan een .ref-file worden gegenereerd waarmee de resultaten
van een sls-simulatie kunnen worden vergeleken.
Voordat dit commando wordt gebruikt moet eerst via de simulator een .lst-file
zijn gemaakt van de simulatie waarvoor een .ref-file moet worden gegenereerd.
Zie voor het gebruik van dit commando de beschrijving bij het commando {\bf Make\_cmd\_file}.

\subsection{Simulate\_dac}
\label{dac-section}
Met dit commando kan een spice-simulatie worden uitgevoerd op de layout van
een ontworpen digitaal-analoog convertor.\\
Wanneer dit commando wordt gekozen verschijnt het window van figuur \ref{dac-window}.

\begin{figure}[htb]
\centerline{\callpsfig{dac_wdw.ps}{width=12cm}}
\caption{Het Simulate\_dac-window van het programma \tool{design\_flow}}
\label{dac-window}
\end{figure}

De volgende instellingen moeten worden gemaakt:
\begin{itemize}
\item De naam van de \smc{dac} die moet worden gesimuleerd door een keuze te maken
     uit de lijst met cellen die verschijnt wanneer op de knop na cells wordt
     geklikt.
\item De waarde van de weerstand die moet worden aangesloten op de ingang van
      de \smc{dac}; deze moet als een integer worden opgegeven na value bij
      input-resistor.
\item De naam van de ingang van de \smc{dac} waaraan de weerstand moet worden
      aangesloten via de lijst terminals, die verschijnt na het klikken op de
      knop na 'to' op de input-resistor balk.
\item De verbinding van de ingangsweerstand naar vss of vdd, welke keuze
      gemaakt kan worden door het klikken op de knop na 'from' op de
      input-resistor balk.
\item De waarde van de weerstand die moet worden aangesloten op de uitgang van
      de \smc{dac}; deze moet als een integer worden opgegeven na value bij
      output-resistor.
\item De naam van de uitgang van de \smc{dac} waaraan de weerstand moet worden
      aangesloten via de lijst terminals, die verschijnt na het klikken op de
      knop na 'to' op de output-resistor balk.
\item De verbinding van de uitgangsweerstand naar vss of vdd, welke keuze
      gemaakt kan worden door het klikken op de knop na 'from' op de
      output-resistor balk.
\end{itemize}
Aan de linkerzijde van het window kunnen nog de volgende opties worden ingesteld:
\begin{itemize}
\item {\bf extraction}: Hiermee kan worden gekozen of voor de simulatie het circuit
      eerst opnieuw uit de layout moet worden ge"extraheerd ({\bf yes}, {\bf no} of {\bf auto}).
      In dit laatste geval wordt alleen een extractie uitgevoerd, wanneer geen
      up-to-date ge"extraheerd circuit meer beschikbaaar is.
\item {\bf capacitors}: Hiermee kan worden gekozen of de op de chip aanwezige
      capaciteiten ook moeten worden ge"extraheerd ({\bf yes}, {\bf no}). De simulatie wordt
      door de extractie van de capaciteiten wat nauwkeuriger ten koste
      van een langere simulatietijd.
\end{itemize}
Wanneer bovenstaande instellingen zijn uitgevoerd kan en spice-simulatie van het geheel
worden gedaan door op de button \button{Doit} te klikken.
Er wordt dan een command file gemaakt, die eventueel nog gewijzigd kan worden,
en de spice-simulatie wordt m.b.v. deze file uitgevoerd.
De resultaten van de simulatie worden in het tekst gedeelte van het
Simulatie\_dac window in tabelvorm getoond.\\
Door het klikken op \button{ShowResult} wordt een grafische voorstelling getoond van
de \smc{dac}-signalen.\\
Het Simulatie\_dac window verdwijnt weer door op \button{Cancel} te klikken.

\subsection{Compare}
Met dit commando kan een vergelijking worden uitgevoerd tussen een .ref-file en
een simulatie-resultaat van een sls-simulatie, een .res-file.\\
Wanneer het dit commando wordt gekozen verschijnt het window van
figuur \ref{compare-window}.

\begin{figure}[htb]
\centerline{\callpsfig{compare_wdw.ps}{width=11cm}}
\caption{Het Compare-window van het programma \tool{design\_flow}}
\label{compare-window}
\end{figure}

Onder het {\bf File}-menu van dit window zijn de volgende commando's beschikbaar:
\begin{itemize}
\item {\bf Read\_ref}: Met dit commando wordt een .ref-file gelezen. Deze file wordt
                zichtbaar gemaakt in het tekst-gedeelte van het window
                in de vorm van een kolom aan de lin\-kerzijde van dit window.
                Tevens worden default waarden voor de start-time, time-step en
                stop-time ingesteld, bepaald via de ref-file.
                Tevens worden de terminal-namen getoond in het
		terminal-gedeelte van het window.
\item {\bf Read\_res}: Met dit commando wordt een .res-file gelezen. Deze file
                wordt getoond als een kolom rechts van de kolom van de
                .ref-file.
\item {\bf Quit}: Met dit commando wordt het Compare-window weer verwijderd.
\end{itemize}
Onder het {\bf Commands}-menu van het Compare-window zijn de volgende commando's
beschikbaar:
\begin{itemize}
\item {\bf Compare}: Met dit commando wordt een vergelijking gedaan tussen de waarden
               van de .res-file en de .ref-file. De tijdstippen van vergelijking
               worden bepaald uit de opgegeven waarden van de start-tijd,
               tijdstap en stop-tijd.\\
               Op ieder tijdstip waarop een vergelijking wordt uitgevoerd
               wordt een nieuwe regel aan de tekst in het tekst-gedeelte
               toegevoegd met de waarden van de referentie en het resultaat
               en {\bf OK} wanneer de waarden gelijk zijn, en {\bf error} wanneer
               deze waarden verschillen. Tevens wordt op de onderste
               tekst-window aangegeven hoeveel vergelijkingen zijn uitgevoerd
               en het aantal fouten wat hierbij is gevonden.
               Door in het rechter tekst-gedeelte van het Compare-window op
               een regel met {\bf OK} of {\bf error} te klikken worden de ref en
               res waarden van deze regel gekopieerd naar het bovenste
               tekst-gedeelte van het Compare-window. Door nu in het
               terms-gedeelte de naam van een terminal aan te klikken
               worden de waarden van deze terminal en de ref- en res-waarden
               in het bovenste tekst-gedeelte van het Compare-window
               geaccentueerd.
\item {\bf First\_eror}: Door dit commando te kiezen wordt de eerste error-line
                  in het bovenste tekst-gedeelte getoond.
\item {\bf Next\_error}: Dit commando toont de volgende error-line
                  in het tekst-gedeelte.
\item {\bf Prev\_error}: Dit commando toont de voorgaande error-line
                  in het tekst-gedeelte.
\item {\bf Last\_error}: Dit commando toont de laatste error-line
                  in het tekst-gedeelte.
\end{itemize}
Voorts zijn in het Compare-window nog entries gemaakt om te kunnen
opgeven wat het eerste tijdstip is waarop een vergelijking moet worden
gemaakt (start-time), het tijdsverschil tussen opeenvolgende stappen (time-step)
en het laatste tijdstip waarop nog een vergelijking wordt gedaan.\\
Default wordt als time-step de periode van de klok van de schakeling
genomen en als start-tijd een zodanige waarde dat de vergelijkingen steeds
worden gemaakt tegen het einde van een klok-periode (default 80 procent).

\section{Het Package-menu}
Dit menu wordt getoond wanneer in het midden-gedeelte van \tool{design\_flow}
op de naam van een package wordt geklikt.
In dit menu bevinden zich de volgende sub-menu's:
\begin{itemize}
\item {\bf Edit}: Door dit item te kiezen wordt een edit-window gestart, waarin de
             \smc{vhdl}-beschrijving van het package kan worden gewijzigd.
Zie ook hoofdstuk \ref{edit-section}.
\item {\bf Recompile}: Door dit item te kiezen, wordt de \smc{vhdl}-beschrijving van
                  het package opnieuw gecompileerd.
\item {\bf Delete}: Door dit item te kiezen wordt een window getoond, waarmee het
             package kan worden verwijderd. De (fout)meldingen hierbij zullen
             in dit window worden getoond.
\end{itemize}

\section{Het Entity-menu}
Dit menu wordt getoond wanneer in het midden-gedeelte van \tool{design\_flow}
op de naam van een entity wordt geklikt.
In dit menu bevinden zich de volgende sub-menu's:
\begin{itemize}
\item {\bf Edit}: Door dit item te kiezen wordt een edit-window gestart, waarin de
             \smc{vhdl}-beschrijving van de entity kan worden gewijzigd.
Zie ook hoofdstuk \ref{edit-section}.
\item {\bf Recompile}: Door dit item te kiezen, wordt de \smc{vhdl}-beschrijving van
                  de  entity opnieuw gecompileerd.
\item {\bf Make Component}: Hiermee wordt vanuit de \smc{vhdl}-beschrijving
                     een component-beschrijving gegenereerd en geplaatst in de
                     components-directory voor later gebruik als component
                     in een architecture beschrijving of als cell in de
                     schematic-editor.
\item {\bf Add Architecture}: Door dit item te kiezen, wordt een edit-window
                         gestart met daarin een frame-work voor een
                         architecture bij de betreffende entity. Deze
                         architecture kan dan worden geedit en opgeslagen.
\item {\bf Delete}: Door de keuze van dit commando verschijnt er een window
                waaruit kan worden gekozen welke delen van de betreffende
                cell uit de diverse databases moeten worden verwij\-derd.
                De (fout)meldingen hierbij zullen ook in dit window worden
                getoond.
\end{itemize}

\section{Het Architecture-menu}
Dit menu wordt getoond wanneer in het midden-gedeelte van \tool{design\_flow}
op de naam van een architecture wordt geklikt.
In dit menu bevinden zich de volgende sub-menu's:
\begin{itemize}
\item {\bf Edit}: Door dit item te kiezen wordt een edit-window gestart, waarin de
             \smc{vhdl}-beschrijving van de architecture kan worden gewijzigd.
Zie ook hoofdstuk \ref{edit-section}.
\item {\bf Recompile}: Door dit item te kiezen, wordt de \smc{vhdl}-beschrijving van
                  de  architecture opnieuw gecompileerd.
\item {\bf Parse SLS}: (Alleen bij architectures van het type 'structural'.)
                  Genereer de circuit beschrijving.
\item {\bf Add Configuration}: Door dit item te kiezen wordt een nieuwe
                         configuratie gemaakt voor het betreffende
                         entity-architecture paar. Er verschijnt een window
                         waarin de naam van de nieuwe configuration moet
                         worden gegeven. Daarna wordt de configuration
                         gegenereerd. Voor de eventuele componenten in
                         de architectuur-beschrijving wordt gezocht naar
                         aanwezige configuraties hiervan.
                         Is er geen configuratie aanwezig, dan wordt de
                         generatie onderbroken, en moet deze eerst worden
                         gemaakt.  Is er slechts
                         \'e\'en configuratie, dan wordt deze gekozen.\\
                         Zijn er meerder configuraties aanwezig,
                         dan verschijnt er een window, waaruit men een keuze
                         kan maken voor \'e\'en van de configuraties.\\
                         Er verschijnt een edit-window, waarin de
                         gemaakte configuratie wordt getoond, en waarin deze
                         eventueel nog kan worden gewijzigd.
\item {\bf Delete}: Door dit item te kiezen wordt een window getoond, waarmee de
             architecture kan worden verwijderd. De (fout)meldingen hierbij
             zullen ook in dit window worden getoond.
\end{itemize}

\section{Het Configuration-menu}
Dit menu wordt getoond wanneer in het midden-gedeelte van \tool{design\_flow}
op de naam van een configuration wordt geklikt.
In dit menu bevinden zich de volgende sub-menu's:
\begin{itemize}
\item {\bf Edit}: Door dit item te kiezen wordt een edit-window gestart, waarin de
             \smc{vhdl}-beschrijving van de configuration kan worden gewijzigd.
Zie ook hoofdstuk \ref{edit-section}.
\item {\bf Recompile}: Door dit item te kiezen, wordt de \smc{vhdl}-beschrijving van
                  de  configuration opnieuw gecompileerd.
\item {\bf Simulate}: Door het aanklikken van dit commando wordt de ModelSim simulator
               gestart waarbij de betreffende configuration in de simulator
               wordt geladen.
\item {\bf Synthesize}: Door het aanklikken van dit commando wordt het window van
	figuur \ref{synth-window} gestart, van waaruit de synthese
	software van Synopsys kan worden gestart.
	Voor een uitleg, zie de paragraaf hieronder.
\item {\bf Delete}: Door dit item te kiezen wordt een window getoond, waarmee de
             configuration kan worden verwijderd. De (fout)meldingen hierbij
             zullen ook in dit window worden getoond.
\end{itemize}

\subsection{Het Synthesize commando}
Wanneer het Synthesize commando uit het Configuration-menu
wordt gekozen, wordt het window van figuur \ref{synth-window} gestart.
\begin{figure}[htb]
\centerline{\callpsfig{synth_wdw.ps}{width=12cm}}
\caption{Het Synthese-window van het programma \tool{design\_flow}}
\label{synth-window}
\end{figure}
De \smc{vhdl}-code van het bijbehorende entity-architecture paar zal daarbij worden
gesynthetiseerd, waarbij de gesynthetiseerde subcircuits zullen worden
gekozen aan de hand van de configuration.
Voordat de synthese plaats vindt moeten alle subcircuits zijn gesynthetiseerd.\\
In het bovenste gedeelte van het Synthese-window staat de bibliotheek met basis-cellen die worden
gebruikt om de schakeling samen te stellen.
Voor deze bibliotheek is een default-waarde gegeven, waarvan alleen in
bijzondere omstandigheden mag worden afgeweken.\\
Aan de linkerzijde van het window kan worden opgegeven in welk formaat het
resultaat moet worden getoond:
\begin{itemize}
\item {\bf infinite}: waarbij het totale ontwerp op \'e\'en A4-tje wordt afgebeeld of
\item {\bf A4}: waarbij het totale ontwerp over meerdere A4-tjes zal worden verdeeld.
\end{itemize}
Aan de linkerzijde van het Synthese-window bevinden zich de volgende
commando-knoppen:
  \begin{itemize}
  \item {\bf ReadSynthScript}: De opties voor de synthesizer kunnen in de vorm van een
                       tcl-script aan de synthese-software worden
                       opgegeven. Dit synthese-script kan automatisch
                       worden gegenereerd. (zie het volgende commando).
                       Het laatst-gemaakte script kan met het commando
                       'ReadSynthScript' in het tekst-window van het
                       synthesizer options window zichtbaar worden gemaakt.
  \item {\bf MakeSynthScript}: Via dit commando wordt een tcl-script gegenereerd
                       met aanwijzingen voor de synthese software hoe de
                       betreffende schakeling te synthetiseren. In het algemeen
                       zal dit automatisch gegenereerde script goed zijn voor
                       de synthese. Het wordt getoond in het tekst-window van
                       het synthese options window. Eventueel kan het script
                       hier nog worden aangepast.
  \item {\bf Synthesize}: Door het klikken op deze knop zal de synthese worden
      gestart. Allereerst wordt de te synthetiseren code nog door
      de Synopsys software gecompileerd en daarna wordt de synthese
      uitgevoerd.\\
      Eventuele foutmeldingen verschijnen in het tekst gedeelte van het
      synthese-window.
      Ook worden hierin wanneer de synthese succesvol is verlopen
      gegevens over het aantal gebruikte cellen weergegeven.
      Het is goed om te kijken, of vooral het aantal flipfloppen wat wordt
      gebruikt overeenkomt met het te verwachten aantal.\\
      Wanneer de synthese is gelukt worden gesynthetiseerde \smc{vhdl}-files
      gemaakt en in de \smc{vhdl}-directory geplaatst.
      Tevens wordt er van het gesynthetiseerde circuit een \smc{sls}-file gemaakt.
  \item {\bf Show circuit}: Via dit commando wordt een schema van het
                      gesynthetiseerde circuit getoond.
  \item {\bf Compile}: Via dit commando worden alle gemaakte \smc{vhdl}-files
	in de 'working-library' van de simulator gecompileerd voor verdere simulaties.
  \item {\bf Parse\_sls}: Via dit commando wordt de gemaakte \smc{sls}-file ingevoerd
                     in het circuit gedeelte van de \smc{nelsis} database die
                     is gekozen via het Database menu.
  \item {\bf Write logfile}: Door het klikken op deze knop zal de tekst van het
      tekst-gedeelte van het synthese-window naar een file worden weggeschreven.
      De naam van de file is dezelfde als die van de originele \smc{vhdl}-file,
      maar dan met de extensie synlog.
      Dus is de oorspronkelij\-ke file bijvoorbeeld {\it piet.vhd}, dan wordt de
      log-file {\it piet.synlog}.
  \item {\bf Cancel}: Door het klikken op deze knop zal het synthese-window weer
      verdwijnen.
  \end{itemize}

\section{Het circuit-menu}
Dit menu wordt getoond wanneer in het midden-gedeelte van \tool{design\_flow}
op de naam van een circuit wordt geklikt.
In dit menu bevinden zich de volgende sub-menu's:
\begin{itemize}
\item {\bf Place \& route}:
Via dit commando wordt de layout-editor \tool{seadali} gestart, waarmee de layout
van een circuit kan worden gemaakt d.m.v. placement (\tool{madonna}) en routing (\tool{trout}).
\item {\bf Run row placer}:
Via dit commando kan een alternatief plaatsingsprogramma worden opgestart.
Deze plaatser is vooral geschikt voor het op efficiente wijze plaatsen van 
(grote aantalen) sub-cellen uit de cel-blbliotheek.
Deze plaatser kan niet worden gebruikt voor het plaatsen van sub-cellen
die zelf ontworpen zijn.
Het bedraden moet daarna nog worden gedaan m.b.v. het programma \tool{trout}
via het commando Place \& route.
\item {\bf Simulate}:
Met dit commando kan een \tool{sls} of \tool{spice} simulatie worden uitgevoerd.
Dit is een simu\-latie op transistor-niveau van het circuit.
Wanneer het dit commando wordt gekozen verschijnt het window van
figuur \ref{sim-sls-window}.
\begin{figure}[htb]
\centerline{\callpsfig{sim_sls_wdw.ps}{width=12cm}}
\caption{Het Simulate-window van het programma \tool{design\_flow}}
\label{sim-sls-window}
\end{figure}
In dit window kan de commando file worden gekozen door op het tekst-vak
na 'commands:' te klikken of de simulator door op het tekst-vak na 'simulator:'
te klikken.
Door op de \button{Doit} knop te klikken wordt de simulatie uitgevoerd.
De resultaten van de simulatie worden o.a. weggeschreven op een .res file,
die weer door het commando {\bf Compare} kan worden gebruikt.\\
Door de knop \button{ShowResult} te kiezen worden de resultaten van de
simulatie getoond. (negeer een eventuele foutmelding over het niet aanwezig zijn
van de .dmrc-file).\\
Door de \button{Cancel} knop te kiezen wordt het Simulatie-window weer verwijderd.
\item {\bf Delete}:
Met dit commando kan het circuit worden verwijderd.
\end{itemize}

\section{Het layout-menu}
Dit menu wordt getoond wanneer in het midden-gedeelte van \tool{design\_flow}
op de naam van een layout wordt geklikt.
In dit menu bevinden zich de volgende sub-menu's:
\begin{itemize}
\item {\bf Place \& route}: Via dit commando wordt de layout-editor \tool{seadali} gestart, 
waarmee de layout opnieuw kan worden geplaatst en bedraad of waarbij de layout
eventueel op een andere manier kan worden aangepast.
\item {\bf Run row placer}:
Zie hetzelfde commando in het circuit-menu.
\item {\bf Extract VHDL}: Via dit commando kan de \smc{vhdl}-code uit de layout worden ge"extraheerd.
  Wanneer dit commando wordt gekozen verschijnt het window van
  figuur \ref{extr-vhdl-window}.
  \begin{figure}[htb]
  \centerline{\callpsfig{extr_wdw.ps}{width=12cm}}
  \caption{Het Extract-\smc{vhdl}-window van het programma \tool{design\_flow}}
  \label{extr-vhdl-window}
  \end{figure}

  In het algemeen zal alleen van de hoogste cell in de hierarchie een
  extractie behoeven te worden gedaan.
  Alleen als na simulatie blijkt dat de resultaten
  hiervan niet correct zijn, kan om dit nader te onderzoeken
  ook van sub-cellen een extractie worden gedaan.\\
  Aan de linkerzijde van dit window bevinden zich de commando-knoppen, waarmee
  de volgende commando's zijn geimplementeerd:
  \begin{itemize}
  \item {\bf Get}: Door de keuze van dit commando wordt vanuit de layout van het
           gekozen design uit de gekozen \smc{nelsis} database een \smc{vhdl}-beschrijving
           gemaakt, die wordt getoond in het rechter gedeelte van het
           extract vhdl window.\\
           De extractie is zodanig, dat de hierarchie uit het ontwerp wordt
           gehaald, zodat alleen basis-cellen in de ge"extraheerde beschrijving
           aanwezig zijn.
  \item {\bf Write}: Door de keuze van dit commando wordt het resultaat van de
             extractie, zoals getoond in het window weggeschreven naar een
             file. De naam waaronder dit gebeurd is dezelfde als die van de
             cell, maar dan met de toevoeging -extracted. De extensie is .vhd.
             Dus is de naam van de cell {\it klaas}, dan wordt de naam van
             de file: {\it klaas-extracted.vhd}. De directory waarin de file wordt
             geschreven is \smc{vhdl}.
  \item {\bf Compile}: Hiermee wordt de gegeneerde \smc{vhdl} beschrijving compileerd.
  \item {\bf Cancel}: Door de keuze van dit commando wordt het Extract\_vhdl-window
                verwijderd.
  \end{itemize}
\item {\bf Simulate}:
Dit commando komt overeen met het simulate commando uit het circuit menu.
Omdat nu voor de simulatie het circuit eerst ge"extraheerd wordt uit de layout,
kunnen hier echter nog een aantal opties voor extractie worden ingesteld.
Zie ook de {\bf Simulate\_dac} paragraaf (\ref{dac-section}).
\item {\bf Make LDM}:
Met dit commando kan een layout-file worden gemaakt in het \smc{ldm}-formaat.
Wanneer dit commando wordt gekozen verschijnt het window van
figuur \ref{ldm-window}.

\begin{figure}[htb]
\centerline{\callpsfig{ldm_wdw.ps}{width=11cm}}
\caption{Het make\_ldm-window van het programma \tool{design\_flow}}
\label{ldm-window}
\end{figure}

De naam van de file die moet worden gemaakt moet worden opgegeven in
het venster achter outfile. De extensie .ldm zal automatisch worden
toegevoegd en behoeft dus niet te worden opgegeven.\\
Omdat onder de naam van de outfile een layout wordt aangemaakt,
moet deze naam anders zijn dan een bestaande layout cell.
Bijvoorbeeld {\it hotelout} voor de cell {\it hotel}.\\
De generatie van de \smc{ldm}-file zal worden uitgevoerd door het klikken
op de \button{Make\_ldm} knop.\\
Wordt op \button{Cancel} geklikt, dan verdwijnt het make\_ldm-window weer.
\item {\bf Delete}:
Met dit commando kan de layout worden verwijderd.
\end{itemize}

\section{Het \smc{vhdl} edit-window}
\label{edit-section}
Voor het genereren en veranderen van \smc{vhdl}-files wordt in \tool{design\_flow} een
edit-window gebruikt wat wordt opgeroepen op een van de hiervoor behandelde manieren.
Het ziet er uit als aangegeven in figuur \ref{edit-window}.
In de titelbalk van dit window staat vermeld welke file is geladen in de editor.

\begin{figure}[htb]
\centerline{\callpsfig{edit_wdw.ps}{width=12cm}}
\caption{Het edit-window van het programma \tool{design\_flow}}
\label{edit-window}
\end{figure}

De menubalk bevat een drietal menu's:
\begin{itemize}
\item {\bf File}: Hierin zijn de volgende commando's opgenomen:
            \begin{itemize}
            \item {\bf Read}: Voor het inlezen van een andere  \smc{vhdl}-file uit de
             \smc{vhdl}-directory
            \item {\bf Save}: Voor het Saven van de getoonde tekst in een file
              met de naam als getoond in de titelbalk.
            \item {\bf Quit}: Voor het afsluiten van de editor.
            \end{itemize}
\item {\bf Edit}: Hierin zijn de volgende commando's opgenomen:
            \begin{itemize}
            \item {\bf Search/Replace}: Voor het zoeken naar tekst en het
            vervangen hiervan door andere tekst in het tekst-window van de editor.
            \item {\bf Add component}: Voor het toevoegen van een component
            declaratie aan de architecture beschrijving van een circuit.
            Wanneer dit commando wordt gekozen verschijnt een lijst met
            bestaande componenten, waaruit er een gekozen kan worden.
            Deze wordt dan aan de architecture toegevoegd.
            \end{itemize}
\item {\bf Font}: Hiermee kan de grootte van de letters in het tekst-window worden
            veranderd.
\end{itemize}
Voorts bevat de menubalk nog de \button{compile} knop. Door hierop te klikken
wordt de getoonde file gecompileerd voor gebruik in de simulator.\\
De eventuele foutmeldingen die hierbij optreden worden in het onderste
gedeelte van het edit-window getoond.\\
Rechts in de menubalk wordt nog de positie van de cursor aangegeven wanneer
de linker muisknop wordt ingedrukt.\\
In het tekst-window wordt de gemaakte \smc{vhdl}-code getoond.\\
Hierin kan op de normale manier tekst worden ingevoerd.
Text kan worden geselecteerd door het indrukken van de linker-muisknop en met
ingedrukte knop over de tekst te bewegen. De tekst die is aangeduid wordt dan
geselecteerd wanneer de linker-muisknop weer wordt losgelaten.
Copy en paste operaties kunnen worden uitgevoerd via CTRL-C en CTRL-V, maar ook
door wanneer tekst is geselecteerd met de middelste-muisknop te klikken op de
plaats waar de geselecteerde tekst geplaatst moet worden.
\clearpage
