\input{man_init}
\setcounter{secnumdepth}{0}

\subsubsection{Railway traffic lights}
On a railway line a new station is built.
To guarantee a good throughput of express trains,
an extra track is made near the platforms.
This way the main track remains free for passing
express trains.

%figuur trein.eps
\begin{figure}[bth]
\centerline{\callpsfig{train.eps}{width=1\textwidth}}
\caption{Situation}
\end{figure}

To prevent accidents, a system with traffic lights is constructed.
There are four light ($S_{1a}, S_{1b}, S_{2a}, S_{2b}$)
that are controlled by two signals
($S_{1}\ en\ S_{2}$). 
When a signal is at a logical '1', the 
corresponding lights $S_{na}\ en\ S_{nb}$
are yellow and red respectively.
In the other case, they are both green.
To control the lights, there are three sensors
($d_{1},d_{2},d_{3}$).
They give a logical '1' when the train is in that segment.
It is your task to create a circuit that takes care that the lights are
controlled in such a way that, when the train drivers obey the signals,
there will not occur any accidents.
Further, the design should be such that when two trains are arriving
at the same time on the main track and side track, the train at the main 
track gets priority.
Further it holds that:
\begin{itemize}
\item
The distance to stop a train is less than the distance
between $S_{na}\ and\ S_{nb}$.
\item
It is not possible that a train activitates two sensors at the same time.
\item
The trains are only going from left to right.
\end{itemize}
\end{document}
