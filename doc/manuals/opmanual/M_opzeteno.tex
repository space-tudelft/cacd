\section{Opzet en Organisatie}
\subsection{Inleiding}
In dit hoofdstuk vindt u informatie over de infrastructuur, de begeleiding,
de beoordeling en een tijdschema (indeling van de dagdelen).
In het eerste kwartaal zijn 19.5 dagdelen geprogrammeerd
(3 dagdelen per week, te beginnen vanaf de eerste
week in het kwartaal).
Eventueel kan er nog 1.5 dagdeel worden gebruikt als uitloop.
Twee kwartalen later is verder nog een dagdeel gepland voor het 
testen van het ontwerp.

Het is overigens niet zo dat de groep zich
strikt moet houden aan het voorgestelde tijdschema. 
De eerste dagdelen liggen redelijk vast,
maar de overige dagdelen kunnen in overleg anders worden ingedeeld. 
Opmerking: hoewel in het volgende deel wordt gesproken over middagen
kan een dagdeel voor sommige groepen ook uit een ochtend bestaan.

\subsection{Indeling middagen}

\begin{enumerate}

\item
\smc{Kennismaking} 1 middag
\begin{description} 
\setlength{\itemsep}{0cm}
\item[Doel:]
Kennismaken van de groep en begeleiders. 
Eerste aanzet tot de ontwikkeling van een {\it wij-gevoel}. 
Verduidelijking van de doelstellingen van het practicum.\\
Kennismaking met de computer-omgeving. 
Introductie van \smc{linux} en de te
gebruiken \smc{pc}'s.
Eerste kennismaking met de ontwerpomgeving.
\item[Vorm:]
Eerste deel plenair, tweede deel in groepjes van twee.
\item[Methode:]
Kennismakingsspel?
Kennismaken met de computer.
\item[Begeleiding:]
Alle begeleiders (co\"ordinator, student-assistent en eventueel specialist) 
zijn de gehele middag aanwezig. 
De co\"ordinator is vooral actief in het eerste deel van de
middag, de student-assistent in het tweede deel.
\item[Voorbereiding:]
De studenten moeten hoofdstuk 1 t/m 7 
van de practium\-hand\-leiding gelezen hebben.
Zie verder ook appendix \ref{linux}.
Zo nodig voor zichzelf vragen formuleren t.a.v.\ onduidelijkheden.
\end{description} 
\item
\smc{Inwerken} 2 middagen
\begin{description} 
\setlength{\itemsep}{0cm}
\item[Doel:]
Verder vertrouwd raken met de ontwerphulpmiddelen.
De studenten leren het synthese-trajekt 
(functie$\rightarrow$schema$\rightarrow$layout) 
en de analyse-hulpmiddelen (extractie en simulatie)
te gebruiken. Bovendien leren zij omgaan met de 
ontwerpomgeving op \smc{pc}'s en het gebruik van \smc{vhdl}. \\
Kennismaking met de sea-of-gates chip en fishbone image.
\item[Vorm:]
In groepjes van 2 achter een \smc{pc}.
\item[Methode:]
Aan de hand van een eenvoudige voorbeeldschakeling komen alle 
relevante ont\-werp\-gereedschappen aan bod. Tevens wordt de
hi\"erarchische ontwerpmethode uitgelegd. Aan het begin van elke
middag worden zonodig klassikaal (aan de hele groep) nog enkele
aspekten toegelicht.
\item[Opdracht:]
In de inwerkfase krijgen de studenten een opdracht uit te voeren,
die bestaat uit het ontwerpen van een eenvoudige besturing.
Ze doen hiermee ervaring op met het maken van toestandsdiagrammen welk begrip
in het college digitale systemen is ge{\ii}ntroduceerd. Ze maken hier
kennis met de celbibliotheek en de mogelijkheid van logische synthese
m.b.v.\ een hardware beschrijvings-taal. 
Hi\"erarchie komt hier expliciet aan de orde.
\item[Begeleiding:]
De ontwerpstappen die moeten worden genomen staan in de handleiding,
waardoor de studenten veel zelfstandig kunnen
doen. 
Intensieve (specialistische) begeleiding is echter wel
noodzakelijk om de studenten niet te lang op een dood spoor
te laten zitten.\\
De student-assistent (en eventueel specialist) 
zijn gedurende deze middagen altijd aanwezig.
Van de co\"ordinator wordt verwacht dat hij het indelen in groepjes
co\"ordineert en dat hij tijdens deze periode met elk groepje tenminste
een keer contact heeft gehad.
\item[Voorbereiding:]
Voor een vlotte voortgang van deze 'stoomkursus' is het
noodzakelijk dat de studenten tevoren de relevante stof 
(hoofdstuk 6 en 7 van de practicumhandleiding) be\-stu\-deren. 
Daarnaast zullen ze de opdrachten thuis moeten uitwerken.
\item[Beoordeling:]
Voorbereiding zal beoordeeld worden middels gesprekjes aan het begin
van een middag. Verder zal er beoordeeld worden op het uiteindelijke
produkt (werkt het). 
Over de opdracht moet een kort verslag geschreven worden waarin
uitwerking, realisatie en simulatie beschreven staan. 
Richtlijn voor de omvang van de geschreven tekst is 5 A4's.
\end{description}
\item
\smc{Systeemspecificatie} 3 middagen.
\begin{description}
\setlength{\itemsep}{0cm}
\item[Doel:]
Leren ontwerpen in een groep. Opdelen van het systeem in 
een aantal deelsystemen met eenduidige specificaties.
Een aantal implementatie-vormen afwegen op grond van 
theoretische en praktische aspekten.
Deze drie middagen zijn erg belangrijk voor het verdere functioneren
van de groep. Gedurende deze middagen moet het {\it wij-gevoel} verder
vorm krijgen, ze moeten het gevoel krijgen dat ze gezamenlijk de klus
moeten en kunnen klaren. Aan het eind van deze middagen moet het voor iedereen
duidelijk zijn dat het zich niet houden aan de gemaakte afspraken
fatale consequenties kan hebben voor de schakelingen van de anderen.
\item[Vorm:]
In de groep van 10 personen, vergaderzaal met bord.
Het lijkt zinvol om deze bij\-een\-komsten deels als een echte vergadering
te structureren. 
Verder zal de groep regelmatig in deelgroepen idee\"en uitwerken welke
weer plenair besproken worden.
\item[Methode:]
Aan het begin van de kursus is de opdracht al uitgereikt, zodat
de studenten er tijdens het inwerken al over na kunnen denken.
De groep moet tot een organieke opde\-ling van het systeem komen
(d.w.z.\ er moet een blokschema van de tophi\"erarchie gemaakt worden).
De practicumhandleiding bevat in principe de benodigde informatie over 
de mo\-gelij\-ke bouwstenen van de implementatie. Ontbrekende informatie
over de haalbaarheid van een optie moet door de begeleider verschaft
worden. Hierna worden voor elk deelsysteem nauwkeurige schriftelijke 
specificaties geformuleerd
(d.w.z.\ aantal pinnen, de ti\-ming, signaalformaat etc.). Deze moeten ook
criteria bevatten op grond waarvan be\-oordeeld kan worden wanneer
de schakeling correct werkt.
Bovendien bevat deze beschrij\-ving een schatting van de grootte van 
elk onderdeel. Hiermee kan dan al een voorlopig 'floorplan'
gemaakt worden. 
Als laatste wordt een schatting van de ontwerp\-inspanning voor elk 
deelsysteem (tijdplanning) gemaakt. Afhankelijk daarvan wordt de
groep opgedeeld in een aantal 'ontwerpgroepjes' van 2-3 man. Elk
ontwerpgroepje zal \'e\'en deelschakeling ontwerpen.
\item[Begeleiding:]
Het in de juiste banen leiden van dit groepsproces is niet eenvoudig.
In principe moeten de studenten zelf alles doen en heeft de
begeleiding slechts een adviserende rol. In de praktijk zal het
waarschijnlijk wel nodig zijn dat de vergadering nadrukkelijk 
gestructureerd wordt (opdrachten voor de volgende middag, etc.).
De begeleiding moet hints geven als de groep het spoor kwijt is.
In het uiterste geval moet zelfs een groot deel van de uitwerking
door de begeleiding gegeven worden.
De co\"ordinator zal zich moeten zien als de directeur van een
ontwerpbureau. Hij zal er voor moeten zorgen dat de groep uiteindelijk
uit elkaar gaat met haalbare, goed gespecificeerde deelsystemen.\\
Beide begeleiders zijn gedurende elke middag aanwezig. Indien nodig
kan externe deskundigheid van een specialist ingeroepen worden.
\item[Voorbereiding:]
De studenten dienen vooraf hoofdstuk 8 van de practicumhandleiding te bestuderen.
Verder zullen ze thuis het systeem of deelsystemen moeten uitwerken.
\item[Beoordeling:]
Door de begeleiders (individueel). Hierbij spelen de assertiviteit en
creativiteit 
van de studenten een grote rol, maar ook het kunnen en willen
meedenken over andere dan de eigen oplossingen.
\end{description} 
\item
\smc{Ontwerp van de hoofdopdracht} 13 middagen.
\begin{description} 
\setlength{\itemsep}{0cm}
\item[Doel:]
Maken van het ontwerp onder gegeven randvoorwaarden. 
\item[Vorm:]
In ontwerpgroepjes van 2-3 man de nauwkeurig omschreven
deelschakelingen beschrij\-ven in \smc{vhdl} en testen op een correcte
werking.
Samenstellen van het geheel en testen voordat de layout
wordt gegenereerd.
Genereren van de layout van de totale schakeling.
\item[Methode:]
Van de deelschakelingen worden per deelschakeling de \smc{vhdl}-files
voor de schakeling gemaakt. Het zal in het algemeen nodig zijn om,
gezien de complexiteit ook de deelschakelingen weer onder te
verdelen in sub-schakelingen.
De beschrijvingen moeten worden getest en daarna moeten de beschrijvingen
worden samengesteld tot een geheel en moet dit ook weer worden getest.
Wanneer dit goed werkt kan de layout van het geheel worden
gemaakt.
Belangrijk is de opdracht zodanig te ontwerpen,
{\it dat de schakeling, nadat
hij is vervaardigd, en dus alleen aan de in- en uitgangen van de schakeling
kan worden gemeten, nog steeds is te testen}.
In de eerste middag wordt een 
taakverdeling en tijdsplanning gemaakt (welke ook door de begeleiders
op haalbaarheid getoetst moet worden).
\item[Begeleiding:]
Aangezien er nu hele wilde schakelingen gemaakt kunnen worden is een
goede technische begeleiding van groot belang.
Om vruchteloze arbeid zoveel mogelijk te voorkomen moet
een werkplan (met blokschema) door
de begeleider goedgekeurd worden voordat de deelschakeling
ge{\ii}mplementeerd wordt.
Voorts moet de begeleiding techni\-sche hulp bieden.
Verder zal de begeleiding moeten stimuleren dat er tussen de
ontwerpgroepjes kontakt blijft,
b.v.\ door geregeld met een afvaardiging bij elkaar te komen.\\
Zowel co\"ordinator als
student-assistent (eventueel specialist) 
zijn elke middag aanwezig.\\
De co\"ordinator zorgt ervoor dat er b.v.\ aan het begin van elke middag een
kort werkoverleg is tussen de ontwerpgroepjes zodat afstemming op
elkaars werk gegarandeerd blijft.\\
Daar waar nodig kunnen de ontwerpgroepjes, in overleg met de student-assistent,
externe deskundigheid inroepen.
Gedurende het ontwerp zullen de ontwerpgoepjes een presentatie houden
over de wijze waarop hun deelschakeling functioneert.
\item[Beoordeling:]
Begeleiders beoordelen wie wat gedaan heeft. 
\end{description}

\item
\smc{Nabespreking, evaluatie} 0.5 middag
\begin{description}
\setlength{\itemsep}{0cm}
\item[Doel:]
Evaluatie van het groepsproces, de samenwerking en het practicum. 
\item[Vorm:]
Groep van 10, vergaderzaaltje.
\item[Methode:]
Met ontwerpgroep evalueren: wat ging fout en wat ging goed?
Het groepsproces wordt besproken, ondervonden problemen en conflicten ge\"evalueerd
en in een groter kader geplaatst. Koppeling van de eigen ervaring naar ontwerpen
in groepen in praktijk-situaties.
Groepsbeoordeling?\\
\item[Begeleiding:]
De begeleiders brengen hun kennis en ervaring in en proberen de ervaringen
van de studenten te vertalen naar de praktijk.
\item[(Eind)Beoordeling:]
De ontwerpfase is nu afgesloten. 
Binnen 4 weken na afloop van het practicum dient het verslag
over de groepsopdracht te worden ingeleverd.
Eventueel, indien nodig, kunnen er
nog individuele gesprekken plaatsvinden tussen begeleiders
en studenten.
Pas 2 kwartalen later, na het afronden van het testen en meten, 
en het inleveren van het meetverslag,
zal de definitieve beoordeling voor het practicum worden vastgesteld.
\end{description}
\item
\smc{Testen en Meten hoofdontwerp} 1 middag (2 kwartalen later)
\begin{description}
\setlength{\itemsep}{0cm}
\item[Doel:]
leren meten, werking eigen chip vaststellen.
\item[Vorm:]
Per deelgroep van het ontwerp testen van (een deel van) de ontworpen chip
\item[Methode:]
Eerste deel van de middag: In groepjes (delen van) het ontwerp testen met een logic analyzer.\\
Tweede deel van de middag: Het met de gehele groep plaatsen van de schakeling in de
benodigde periferie en testen of de totale schakeling werkt volgens de
spacificaties.\\
Meetverslag maken.
\item[Begeleiding:]
Specialist en eventueel inzet van student-assistenten.
\item[Beoordeling:]
Op grond van het meetverslag.
\end{description}

\end{enumerate}

\subsection{Infrastructuur}

Het ontwerppracticum wordt gehouden in het 
TU gebouw 35 aan de Cornelis Drebbelweg 5, in de zalen 011 en 010.
Per groep is een ruimte beschikbaar met een vergadertafel en een
zestal \smc{pc}'s die werken onder \smc{linux}.
Verder kan gebruik worden gemaakt van een laserprinter.
Om te werken op de \smc{pc}'s dient men gebruik te maken
van zijn/haar via netID verkregen gebruikersnaam en password.
Zie ook appendix \ref{linux}.

\subsection{Begeleiding}

Elke groep van 10 studenten krijgt twee vaste begeleiders; een
co\"ordinator en een student-assistent. 
Verder is de meeste middagen ook een specialist aanwezig
voor aanvullende begeleiding van de groepjes die op dat moment
bezig zijn.
De begeleidings-taken zijn globaal als 
volgt verdeeld:

Taken {\it groepsco\"ordinator} :
\begin{itemize}
\item
Structuur aanbrengen in de vergaderingen.
\item
Groepsproces in de gaten houden. Komt iedereen aan bod, heeft
iedereen een inbreng, loopt iemand de kantjes eraf, etc.
\item
Bewaken van de voorgang van het ontwerpproces. 
Worden er afspraken gemaakt. 
Houdt iedereen zich aan de afspraken. 
Is er een tijdschema. 
Zijn er deadlines gesteld, etc.
\item
Opdelen van de groep in subgroepen. 
Zorgen voor wisselende samenstelling van subgroepen.
\item
Het in de gaten houden van de communicatie tussen de groepen.
\item
Bemiddelaar/scheidsrechter. 
\item
Controle op juiste afspraken t.a.v. communicatie.
\item
Controleren van voortgang ontwerpproces.
\item
Beoordelen.
\end{itemize}

\vglue 20pt
Taken {\it student-assistent} :
\begin{itemize}
\item
Adviseren op inhoudelijk gebied.
\item
Andere mogelijkheden onder de aandacht brengen.
\item
Beantwoorden van technische vragen.
\item
Assistentie bij gebruik \smc{pc}'s en software-tools.
\item
Trouble shooting.
\item
Verzamelen van bugs in software en opdracht.
\item
Doorverwijzing naar specialist.
\item
Beoordelen.
\end{itemize}
%%
%%

\vglue 20pt
Taken {\it specialist} :
\begin{itemize}
\item
Aanvulling van de taken van de student-assistent.
\end{itemize}

\subsection{Beoordeling}

\begin{itemize}
\item
De beoordeling resulteert niet in een cijfer maar in een voldoende of
niet voldoende beoor\-de\-ling. Studenten die een niet voldoende
beoordeling krijgen moeten een extra opdracht uitvoeren waarvan de
omvang zal afhangen van de mate van niet voldoende. Dit ter
beoordeling van de begeleiders.
\item
Een belangrijk criterium is het actief meedoen met de groep. Indien men
zich niet voldoende inzet kan dit uitsluiting van het ontwerppracticum tot
gevolg hebben.
\item
Mogelijke beoordelingscriteria (niet in volgorde van belangrijkheid):
\begin{enumerate}
\item[-]
De voorbereiding.
\item[-]
Gedrag in de groep (meedenkend, koppig, inbreng
in discussies).
\item[-]
Zelfwerkzaamheid. Roept de student te vroeg c.q.\ te laat de hulp van
anderen in.
\item[-]
Is de student in staat eigen oplossingen te genereren.
\item[-]
Blijft de student te lang stilstaan bij details. Is de student in
staat beslissingen te nemen, de knoop door te hakken.
\item[-] Houdt de student zich aan de gemaakte afspraken.
\item[-] Is de student in staat om uitleg te geven over zijn of haar
eigen resultaten.
\item[-] Kwaliteit van de mondelinge voordracht.
\item[-] Kwaliteit van het schriftelijk verslag.
\item[-]
Werkt de gemaakte (deel)schakeling.
\item[-] Aandeel in het groepswerk.
\item[-] Is de student in staat zijn probleem te specificeren.
\end{enumerate}
\end{itemize}

\cleardoublepage

