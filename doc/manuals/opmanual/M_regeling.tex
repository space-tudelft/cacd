\section{Regelingen}
\subsection{Voorkennis en toelatingseisen}
Om aan het practicum te mogen deelnemen moet men 
een voldoende hebben gehaald voor het vak
Digitale Systemen en het Propedeutisch Laboratorium.
Verder wordt er gebruik gemaakt van kennis opgedaan bij de vakken
Elektroni\-sche Circuits en Lineaire Elektrische Circuits.
\subsection{Inschrijving}
Studenten die aan de toelatingseisen voldoen kunnen
zich aanmelden via TAS (http://www.tas.tudelft.nl)
voor deelname aan het practicum.
\subsection{Groepsbijeenkomsten}
In het practicum wordt gewerkt in groepen van 10 studenten. Elke groep
krijgt \'e\'en vaste begeleider van de staf 
en \'e\'en student-assistent toegewezen. 
Per week worden drie
practicummiddagen of och\-tenden gepland. De aanwezigheid op deze
ochtenden en middagen is verplicht. \\
{\bf Indien men door ziekte of bijzondere
omstandigheid een middag of ochtend moet verzuimen, dan dient men dit
van te voren door te geven aan de begeleiders. Bij meer dan 2 middagen
afwezigheid zal er, in overleg met de begeleiders, een inhaalprogramma
worden vastgesteld, bijvoorbeeld in de vorm van een extra opdracht.}
\subsection{Voorbereiding}
Naast de geroosterde middagen en ochtenden dient men ook tijd uit
te trekken voor de voorbereiding, gemiddeld circa 2 uur per middag of
ochtend.\\
Voor het slagen van het practicum is het belangrijk dat men elke middag
en ochtend goed voorbereid aan de start verschijnt. Deze voorbereiding
kan bestaan uit het lezen van de practicumhandleiding, het bestuderen
van manuals of delen van collegedictaten,
het uitwerken c.q.\ voorbereiden van een opdracht,
het voorbereiden van een voordracht, etc.
De voorbereiding zal door de begeleiders gecontroleerd worden en
meegenomen in de eindbeoordeling.
\subsection{Verslagen}
Over de inwerkopdrachten, de groepsopdracht en later het testen van
de schakeling dient een verslag te worden geschreven.  \\
Het verslag over de inwerkopdracht dient kort te zijn en moet een
beschrijving geven van de uitwerking, realisatie en simulatie van de
ontworpen schakeling. Het verslag dient {\bf twee weken} na be\"eindiging van de
inwerkopdrachten ingeleverd te zijn.
De inwerkopdracht wordt in groepjes van 2 gedaan en men mag 
met z'n twee\"en een verslag schrijven. \\
Het verslag over de groepsopdracht zal veel omvangrijker zijn.
Naast een inleiding en een beschrijving van de schakeling als geheel, 
dient er een uitgebreide beschrijving van elke ontworpen deelschakeling 
te worden opgenomen. 
Ieder groepslid dient een herkenbare bijdrage te leveren
Dit verslag zal tevens dienen als documentatie voor de te
volgen procedures en metingen in de testfase. \\
Het verdient aanbeveling om al tijdens het practicum een begin te
maken met het samenstellen van het verslag over de groepsopdracht. 
Het verslag moet uiterlijk {\bf vier weken} na be\"eindiging van de ontwerpfase van het
practicum ingeleverd zijn.\\
Het meetrapport n.a.v. het testen van de schakeling moet een korte samenvatting 
zijn van de testresultaten
van de ontworpen schakeling en een daaraan verbonden conclusie
t.a.v.\ het ontwerp.
Dit meetrapport dient {\bf een week} na het testen en meten
ingeleverd te zijn.
\subsection{Nabespreking}
Aan het eind van de ontwerpfase van het practicum is er een
groepsbijeenkomst gepland waarin de ervaringen van het werken in de
groep besproken zal worden. 

\cleardoublepage
