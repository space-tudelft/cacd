\section{Richtlijnen voor het Schrijven van een Verslag}

\subsection{Inleiding}

In het eerste jaar heeft u bij het practicum natuurkunde al kennis gemaakt met 
het schrijven van een (wetenschappelijk) verslag.
De daar gehanteerde richtlijnen,
zoals vermeld in de studiehandleiding "Richtlijnen voor het samenstellen
van een rapport" van de faculteit der Technische Natuurkunde, zullen wij 
gebruiken als uitgangspunt voor het schrijven van het verslag. In 
het onderstaande zetten we de belangrijkste punten uit deze handleiding op een 
rijtje.

Als vervolg op het leren samenstellen van een verslag wordt in het
derde/vierde studiejaar de cursus "schriftelijk rapporteren" gegeven.
Deze cursus bouwt 
voort op de in de practica aangeleerde basisvaardigheden voor het schrijven 
van een verslag en stelt ook onderwerpen als doelgericht schrijven in de 
beroepspraktijk aan de orde. 

\subsection{Nut van het rapporteren }

In iedere funktie die u na uw studie gaat uitoefenen, is contact met anderen 
noodzakelijk: in gesprekken, maar vaak ook in schriftelijke vorm. De manier 
waarop u uw standpunt presenteert bepaalt in sterke mate de waardering voor 
uw persoon en uw werk. Of het nu gaat om onderzoeksresultaten of om 
besluitvorming rond een nieuw te openen bedrijf, voor een goed vervolg is nauwgezet 
vastleggen van de resultaten met de juiste argumentatie van groot belang. 

\subsection{Doel van het schrijven van het practicumverslag }

Het doel van het schrijven van het practicumverslag is drieledig: 
\begin{itemize}
\item
Oefening in het schrijven van een goed verslag. Terugkoppeling vindt plaats in 
de vorm van een verslagbespreking.
\item
Beoordeling van uw practicumwerk.
De inhoudelijke kant van het verslag zal mede een rol spelen bij de beoordeling.
Aan de hand van het verslag zal er een gesprek 
plaatsvinden tussen u en uw begeleider. Tijdens dit gesprek zult u uw 
verslag, waar nodig, moeten toelichten en kan uw begeleider bijsturen. Indien 
blijkt dat u het onderwerp onvoldoende beheerst kan er een aanvulling van 
het verslag worden verlangd of zult u een gedeelte van het practicum moeten 
overdoen. 
\item
Documentatie bij het testen en meten.\\
In het verslag moet staan wat u gemaakt heeft,
hoe het aangesloten moet worden,
hoe er getest moet worden, etc.
Dit moet niet alleen bruikbaar zijn voor de
schrijver maar ook voor derden.
\end{itemize}

\subsection{Procedure bij het samenstellen van het verslag }

Begin zo snel mogelijk nadat u uw opdracht, of een belangrijk deel daarvan, 
hebt voltooid met het schrijven van het verslag. Alles ligt dan nog vers in 
het geheugen. \\
Belangrijk is dat u de gevolgde ontwerpprocedures en de resultaten 
goed hebt vastgelegd. Regelmatig maken van aantekeningen is hierbij 
noodzaak !! \\
Begin met het maken van een schema van de structuur van de hoofdtekst. Voor de 
hoofdindeling voldoet meestal de volgende indeling :  \\
probleemstelling - probleemanalyse - ontwerp  
 - resultaten - conclusies.\\ 
Naast de hoofdtekst zal het verslag ook een openingsgedeelte bevatten 
(titelpagina, inhoudsopgave, inleiding) en eventueel een slotgedeelte. \\
Beslis welke delen van het verslag u zult onderbrengen in bijlagen
(b.v.\ uitgebreide afleidingen, computeruitvoer, etc.). \\
Maak vervolgens een concept van de hoofdtekst zonder nog veel aandacht te 
besteden aan de vormgeving. Bedenk dat het verslag geen chronologische weergave 
is van wat u hebt gedaan. Wel beschrijft het in logische volgorde dat wat gedaan is 
om van probleemstelling tot oplossing te komen. \\
Houd het verslag zo beknopt mogelijk. \\
Nadat het concept is vastgelegd stelt u de definitieve versie samen (eventueel 
na overleg met uw begeleider), daarna het openingsgedeelte en slotgedeelte. \\
Corrigeer het verslag voordat het getypt of in het net geschreven wordt. Let 
hierbij op de inhoud (niets vergeten, staat er wat er bedoeld wordt), de taal 
(lopen de zinnen goed, spelling, leestekens) en vormgeving (juiste nummering 
van pagina's, figuren en tabellen, leesbaarheid van grafieken).

\subsection{Onderdelen van het verslag }

Elk verslag bevat in ieder geval de volgende onderdelen: 
\begin{description}
\item[-] omslag 
\item[-] titelpagina 
\item[-] samenvatting 
\item[-] inhoudsopgave
\item[-] inleiding 
\item[-] genummerde hoofdstukken 
\item[-] conclusies
\item[-] literatuurlijst
\end{description}
Daarnaast kunnen de volgende onderdelen voorkomen: \\
\\
\begin{tabular}{lll}

- voorwoord              &- verklarende woordenlijst    &- bijlagen  \\
- lijst van symbolen     &- aanbevelingen               &- register  \\
\end{tabular}

Voor de belangrijkste onderdelen volgen hieronder enkele aanwijzingen:
\begin{itemize}
\item
Omslag en Titelpagina\\ 
De omslag heeft een beschermende funktie. \\
De titelpagina moet de volgende informatie bevatten: titel, naam en 
voorletters van de auteur, plaats, jaar, instelling. Voor het practicum is van 
belang dat u ook uw begeleiders en practicumnummer vermeldt.
\item
Samenvatting\\ 
De samenvatting bevat wat onderzocht is, hoe het onderzocht is en wat de 
conclusie is. Het moet begrijpelijk, kort en informatief zijn.  
\item
Inhoudsopgave \\
De inhoudsopgave bevat alle titels en paginaverwijzingen van de genummerde 
onderdelen van de hoofdtekst, alsmede de (ongenummerde) onderdelen van het 
openingsgedeelte en slotgedeelte. 
\item
Inleiding \\
De inleiding vormt de sleutel tot het verslag. Het geeft nadere informatie 
over het onderwerp, het beschrijft het doel van het verslag en het geeft een 
beschrijving van de opbouw van de tekst. 
\item
Hoofdtekst\\ 
De indeling zal een variant zijn op de reeks:\\
 probleemstelling - methode - resultaten - discussie.\\
Een indeling van het verslag zou b.v.\ kunnen zijn :
\begin{description}
\item[-] Inleiding (doel van de opdracht, waarom van de opdracht) 
\item[-] Probleemstelling (omschrijving, specificaties, randvoorwaarden) 
\item[-] Globale oplossing (beschrijving deelsystemen, taak deelsystemen) 
\item[-] Ontwerpprocedure en ontwerp deelsysteem (specificatie deelsysteem, onderverdeling)  
\item[-] Resultaten (simulaties, vergelijking specificaties)
\item[-] Conclusies
\end{description}
\item
Literatuurlijst \\
Een literatuurlijst bevat complete titelbeschrijvingen van gebruikte werken:
\begin{description}
\item[Boek] Naam en voorletters auteur (geen titels), volledige titel. 
Druk. Plaats, jaar. 
\item[Artikel] Naam en voorletters auteur, volledige titel.
Naam tijdschrift, jaargang, jaartal, bladzijden. 
\item[Rapport/dictaat] Naam en voorletters auteur, volledige titel. 
Plaats, jaar. Instelling of bedrijf. 
\end{description}
\end{itemize}
\subsection{Vormgeving van het verslag }

Het verslag moet worden geschreven op A4-formaat (210 x 297 mm$^2$) en goed 
leesbaar zijn. Getypt werk is vrijwel altijd goed leesbaar, waarbij formules  
eventueel met de hand kunnen worden bijgeschreven. De tekst wordt slechts op een
zijde van het papier geschreven of getypt. Bij wijze van uitzondering kunnen 
grote figuren op de andere (linker) zijde staan. \\
De bladzijden, tabellen, figuren en formules moeten worden genummerd. Verklaar 
alle in de tekst en formules voorkomende symbolen waar deze voor het eerst
gebruikt worden (verwijs zonodig naar een figuur). \\
Voorzie tabellen van een bovenschrift en figuren (grafieken, schema's) van een 
onderschrift. Voorzie ook aanhangsels (b.v.\ SLS of SPICE gegevens) van commentaar. 

\cleardoublepage
