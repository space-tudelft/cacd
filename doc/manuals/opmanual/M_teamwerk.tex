\section{Werken in Teamverband}

\subsection{Inleiding}

Een van de belangrijkste kenmerken van het ontwerppracticum
is, dat er wordt gewerkt in groepen, d.w.z.\ dat de leden van
de groep samen een bepaalde,
gemeenschappelijke einddoelstelling moeten realiseren.

Werken in groepen betekent, dat de groepsleden niet alleen
vak-inhoudelijk bezig zijn met het projekt, maar ook dat zij
\begin{itemize}
\item
samen moeten kunnen werken (werken met anderen, teamwork) en
\item
samen moeten kunnen communiceren (o.a.\ discussi\"eren en vergaderen).
\end{itemize}
Beide zaken zijn aanmerkelijk minder gemakkelijk dan men op
het eerste gezicht zou veronderstellen.

In een team kunnen werken en kunnen communiceren zijn vormen
van sociale vaardigheid. Het zijn extra dimensies van 
projektonderwijs zoals het ontwerppracticum. 
Bij individuele studie is iedere student immers
afzonderlijk bezig met de leerstof. 

Een van de voordelen van projektonderwijs is, dat de student
gedwongen wordt, naast het opdoen van zuiver technische kennis
(het leren van het ''vak''), zich ook bezig te houden met sociale
 vaardigheden en dat hij er nu al, tijdens zijn studie,
ervaring  in kan opdoen.

Ook in zijn latere beroepspraktijk zal de ingenieur immers
bijna nooit alleen (als enkeling) technisch handelen; ook dan
zal hij met anderen moeten samenwerken en dus ook met anderen
moeten communiceren.

Helaas ontbreekt in het OP de tijd voor speciale instructie en
training in deze sociale vaardigheden. Maar om te voorkomen
dat de groepsleden geheel onvoorbereid aan hun taak
zouden moeten beginnen, worden in dit hoofdstuk enkele facetten van
de genoemde vaardigheden behandeld. Volledigheid is onmogelijk. Maar bestudering van het hier gegeven minimum aan theoretische achtergrond en toepassing van de daaruit af te leiden
praktijk-adviezen is een zeer zinnige zaak.

\subsection{De drie dimensies van het werken in een groep}

In de inleiding werden ''samen kunnen werken'' en ''samen kunnen praten''
vormen van sociale vaardigheid genoemd.
Wat is sociale vaardigheid? Dat is: bekwaamheid in sociaal
handelen, anders gezegd: bekwaamheid in het omgaan met mensen
in een groep ten behoeve van een met en door de leden van die
groep te verrichten taak. Bij het OP is deze taak de groepsopdracht.
Als een groep op weg is naar een doel, dan kunnen aan dit ''op weg zijn''
drie aspekten worden onderscheiden:
\begin{enumerate}
\item
     Tijdens de groepsbijeenkomst wisselen de groepsleden, om
     tot hun gemeenschappelijk doel te komen, verzamelde kennis 
en informatie uit; ze selekteren de informatie en
     brengen er een logische samenhang in aan; ze bespreken
     vaktechnische zaken en resultaten van literatuurstudie en
     onderzoek; ze rekenen en konstrueren, etc.
     Kortom: de groepsleden zijn inhoudelijk bezig met de
     uitvoering van een taak, met de verwezenlijking van de
     doelstelling. Men noemt dit de inhoudsdimensie van het
     groepsgedrag.
\item
     Aan het bespreken van en de discussie over de inhoudelijke elementen dient - terwille van de begrijpelijkheid,
     overzichtelijkheid en doelmatigheid - duidelijk vorm te worden gegeven.
     Er moet volgens bepaalde ''spelregels'' worden gepraat.
     Voor de discussie over een probleem moet
     tijdens de bijeenkomst een verantwoorde werkwijze worden
     uitgestippeld; de bespre\-king\-en dienen gestructureerd te
     verlopen; men moet methodisch naar oplossingen zoeken,
     zich beradend op criteria en randvoorwaarden waaraan de
     aangedragen oplossingen moeten voldoen; besluiten moeten
     weloverwogen worden genomen.
     Kortom: de groepsleden moeten tijdens hun besprekingen
     van de inhoudelijke elementen van het projekt een bepaalde aanpak volgen: ze moeten een procedurele vorm geven
     aan de inhoud. Men noemt dit de procedure-dimensie van het
     groepsgedrag.
\item
     Tijdens de groepsbijeenkomst ontstaat er een bepaalde
     sfeer: individuen blijven in een groep weliswaar individuen, maar gaan tegelijk ook gedeeltelijk op in het
     groepsgebeuren. De groeps\-leden reageren op elkaar, waardoor zij een ander gedragspatroon gaan vertonen dan het
     louter individuele. Groepsleden be{\ii}nvloeden elkaar bewust
     en onbewust; ze be{\ii}nvloeden elkaar positief en negatief,
     waardoor een klimaat van enthousiasme en co\"operativiteit,
     maar anderzijds ook van vrijblijvendheid of soms zelfs
     van verveling, irritatie of gedeprimeerdheid kan ontstaan: er ontstaan
     waarde-oordelen over de personen in de groep en hun bijdragen; er groeit gemotiveerdheid, een wil tot samenwerking en inschikkelijkheid, of anderzijds een neiging tot
     wedijver en konflikt; er kan sprake zijn van subgroepvorming, met goede of slechte gevolgen voor het
     groepsgeheel.
     Kortom: tussen groepsleden groeien onderlinge relaties en
     gevoelens, vormen van interactie en groepsdynamiek.
     Men noemt dit de procesdimensie van het groepsgedrag.
\end{enumerate}
%Visueel samengevat:
%              ________ inhoud    ________
%             |                           |
%groep -------|-------- procedure --------|------- doel
%             |                           |
%              -------- proces    --------


Inhoud, procedure en proces zijn dus drie aspekten van het
werken in groepen. Deze drie dimensies be{\ii}nvloeden elkaar
voortdurend. Ze zijn wel te onderscheiden maar niet te scheiden.

     Als over inhoudelijke elementen alleen maar chaotisch
     (ongestructureerd) of in een verziekt klimaat (onwerkbare
     sfeer) gediscussieerd wordt, komt men niet ver; althans
     heel wat minder ver dan men zou kunnen komen, en ook met
     heel wat minder plezier in het werk.
     Inzicht in en beheersen van de genoemde drie dimensies
     zijn onontbeerlijk voor het bereiken van een goed eindresultaat in een goede werksfeer.

\subsection{De vergadering}

De projektvergadering is het moment waarop de groep daadwerkelijk samen aan het werk is. Hier laten mensen zien wat hun
vorderingen zijn en hier wordt besloten hoe er verder wordt
gegaan.
Voor het begin van de vergadering is er een agenda opgesteld \index{agenda}
die door de voorzitter aan de groep bekend wordt gemaakt
(schrijf deze agenda op, of lees de agenda door). Nu weet
ieder wat hem/haar te wachten staat gedurende de vergadering.
De voorzitter (die de agenda van tevoren opstelt) moet er voor
zorgen dat deze een goede opbouw heeft:
\begin{itemize}
\item
     Belangrijke punten eerst: deze mogen nooit door tijdgebrek niet goed aan de orde komen.
\item
     Zorg dat er samenhang is tussen de te behandelen punten
     (logische opbouw: Als je het een weet, kun je verder met
     het volgende).
\item
     De agendapunten moeten zo worden geformuleerd dat deelnemers weten waar het precies over gaat.
\end{itemize}

Een agenda heeft gedeeltelijk een vaste opbouw:
\begin{description}
\item[-]
vaste punten
\begin{itemize}
               \item    opening en mededelingen
               \item    wijzigingen in de agenda (uitsluitend als
                    er zeer dringende redenen daarvoor zijn)
               \item    notulen/verslag + besluitenlijst van de
                    vorige vergadering
               \item    binnengekomen en verzonden stukken
\end{itemize}
\item[-]
variabele punten
\begin{itemize}
              \item    de eigenlijke agendapunten (in weloverwogen
                    volgorde en goed geformuleerd, zie
                    boven)
\end{itemize}
\item[-]
vaste punten
\begin{itemize}
                \item    vaststelling datum, plaats, tijd en 
                    concept-agenda van de volgende vergadering
               \item    rondvraag
               \item    sluiting
\end{itemize}
\end{description}


\subsection{Rollen en funkties}

\paragraph{Rol van de groepsleden}
In een vergadering vervult ieder lid van de groep een bepaalde
funktie. Wanneer je dus bijvoorbeeld geen voorzitter of notulist bent, ben je toch mede verantwoordelijk voor het goede
verloop van de groepsbijeenkomst. Dat betekent in ieder geval
een goede voorbereiding van wat er gaat gebeuren (presentatie
van het werk dat je gedaan hebt of je gedachten bepalen over
een bepaald onderwerp).
De volgende personen hebben een bijzondere funktie in de
vergadering:

\paragraph{De voorzitter} \index{voorzitter}
De voorzitter is verantwoordelijk voor een effici\"ent verloop
van de vergadering. Dat betekent dat er voor hem/haar relatief
veel tijd in de voorbereiding gestoken moet worden. Aan de
hand van de notulen van de vorige vergadering en m.b.v.\ het
tijd-werkschema kan een agenda opgesteld worden. Van tevoren
moet de voorzitter in gedachten bepalen wat het uiteindelijke
resultaat moet zijn van ieder te behandelen punt: Een besluit,
op de hoogte brengen van andere groepsleden of bijvoorbeeld
met elkaar van gedachten wisselen om een mening te vormen over
een bepaald onderwerp. Wanneer je je dat niet van tevoren
realiseert weet gedurende de vergadering misschien helemaal
niemand meer waar de groep nu precies mee bezig is.

Het is zinvol om van tevoren te schatten hoeveel tijd een
bepaald onderwerp in beslag gaat nemen. Hierdoor voorkom je
dat je een te lange agenda maakt voor de beschikbare tijd
waardoor er mis\-schien bepaalde beslissingen te overhaast
worden genomen. Belangrijke beslispunten dienen vooraf op
schrift geformuleerd te worden. 
Tijdens de vergadering is de voorzitter formeel de leider. Dat
betekent dat hij/zij bepaalt wie er aan het woord is en in mag
grijpen wanneer er zaken ter sprake komen die volgens hem/haar
niets zinnigs bijdragen aan het doel van het agendapunt. De
hele groep moet echter meewerken en stil zijn wanneer de
voorzitter aan iemand anders het woord heeft gegeven. Dat
lijkt kinderachtig maar komt het ordelijk (snel en duidelijk)
vergaderen ten goede.
Het ligt voor de hand dat er een goede samenwerking moet zijn
tussen de groep en de voorzitter. Wanneer de voorzitter zich
te autoritair opstelt zullen de mensen zich of gepasseerd
voelen of onge{\ii}nteresseerd raken door de vervelende sfeer die
kan ontstaan.
Wanneer de voorzitter echter ''te vriendelijk'' is, zal de vaart
snel uit de vergadering zijn. Er zit geen structuur meer in,
omdat niemand die er bewust inbrengt. De voorzitter moet dus
wel duidelijk aanwezig zijn.
Het ideaal lijkt ook hier weer in het midden te liggen. De
zgn.\ democratische stijl van leidinggeven zorgt ervoor dat de
voorzitter voortdurend oplet wat er in de groep leeft en van
daaruit ook handelt. De democratisch leider ziet de groep als
iets waar voorzichtig mee moet worden omgesprongen maar waar
wel duidelijk leiding aan moet worden gegeven. Respekt voor
elkaar (goed luisteren!) is een van de belangrijkste aspekten
voor het slagen van de groepssamenwerking.
De voorzitter moet de aandacht van de groepsleden vragen bij
het te behandelen onderwerp. Dat kan hij/zij bijvoorbeeld doen
door regelmatig samen te vatten wat er besproken is. Het 
maken van aantekeningen kan daarbij een grote hulp zijn. Soms
is het van belang om zaken op het bord of op papier te zetten.
Hierdoor wordt de aandacht van de groep ''centraal'' gehouden.
In feite verloopt het oplossen van een probleem, met bijbehorende besluitvorming, in een vergadering net als in het hele
projekt (zie paragraaf \ref{besluitvorming}):
\begin{itemize}
\item[Fase 1]  Beeldvorming:
          Doelstellingen, randvoorwaarden, uitgangspunten, probleemstellingen, etc.\\
	  Hierin bepaal je: ''Waar hebben we het precies over en wat moet ons resultaat zijn''.
\item[Fase 2]  Het genereren van oplossingen:
          Dit is een wat brainstormachtig gebeuren waarbij
          alle (voor de groep) denkbare oplossingen op tafel
          moeten komen. Schrijf alles op!
\item[Fase 3]  Concentreer fase:
          Onder leiding van de voorzitter worden de criteria
          geformuleerd waarmee de oplossingen beoordeeld gaan
          worden. In grote lijnen vallen er nu al de nodige
          oplossingen af. Al pratend over de oplossingen zullen er meer criteria komen en tenslotte komt men bij de besluitvorming.
\item[Fase 4]  Besluitvorming:
          De voorzitter moet hier zorgvuldig te werk gaan.
          Spreek duidelijk af waarover precies besloten gaat
          worden en zorg dat niemand ''vergeten'' wordt.
          Iedereen moet de kans hebben zijn/haar voorstel te
          verdedigen om te voorkomen dat men er later op terug
          wil komen.
\item[Fase 5]  Besluit-uitvoering:
          Hierin moet afgesproken worden wie wat gaat doen.
          Probeer het werk eerlijk te verdelen. Vriendjes gaan
          graag samen in een subgroep waardoor er wel eens
          mensen uit de boot vallen en met een minder boeiend
          onderwerp worden opgescheept.
          De voorzitter moet in de volgende vergadering de
          uitvoering van het besluit kontroleren.
\end{itemize}

\paragraph{De notulist} \index{notulist}
Van iedere vergadering moet een verslag gemaakt worden. Dat is
van belang om onduidelijkheid over de genomen besluiten te
voorkomen en kontrole uit te kunnen oefenen. Zorg er dus ook
voor dat de notulen bewaard worden en voor iedereen te vinden
zijn (b.v.\ ruim voor de volgende vergadering in het postvak).
De volgende zaken moeten in de notulen vermeld staan:
Of men kiest voor uitgebreide notulen of voor een beknopt
verslag, altijd zullen de volgende zaken vermeld moeten worden:
\begin{itemize}

\item    datum en tijdsduur van de vergadering
\item    aanwezig/afwezig (namen van de aanwezigen opsommen -
     funktie van voorzitter en verslaglegger apart bij de
     desbetreffende namen vermelden -, bij de afwezigen vermelden of niet met of zonder bericht van verhindering
     is).
\item    weergave van de behandeling van alle agendapunten (volgorde van de agenda aanhouden).
     altijd vermelden:
\begin{description} 
  \item[-] wat is er besloten?
                         \item[-] werkafspraken (wie doet wat voor     
                          wanneer?)
\end{description}
\item    besluitenlijst (deze is tweeledig):
\begin{description}
               \item[-]    opsomming van alle genomen besluiten en
                    gemaakte afspraken.
               \item[-]    opsomming van besluiten en werkafspraken
                    uit vroegere vergaderingen,\\
		    die om welke reden dan ook nog niet zijn afgewerkt.
\end{description}
\end{itemize}
NB.: In de notulen/het verslag dient objektief te worden
vermeld, wat er gezegd/gebeurd is. Geen eigen meningen of
kommentaar van de verslaglegger. Ook de eigen deelneming aan
de discussie behoort door de verslaglegger objektief te worden
weergegeven.

Wanneer de groep gekozen heeft voor een roulerend voorzitterschap
(een andere voorzitter per vergadering),
kan het nuttig zijn om de notulist van periode ''n'' de
voorzitter voor periode ''n+1'' te laten zijn. Deze weet dan
precies wat er gaande is.

\paragraph{De taken van de deelnemers}

Het projektgroepswerk, met daarin als belangrijke komponenten
de discussie en vergadering, is een gezamenlijk proces van
alle groepsleden, van voorzitter en deelnemers.\\
Te vaak denken de deelnemers dat het wel en wee van een bijeenkomst uitsluitend in handen ligt van de voorzitter. Maar
zelfs de beste voorzitter moet falen als de deelnemers niet
meewerken.\\
De taak van de voorzitter is veelomvattend en niet gemakkelijk; van de deelnemers mag men vragen, dat zij de voorzitter
de uitvoering van zijn taak mogelijk maken. Dat is hun voornaamste taak; alle andere vloeien eigenlijk daaruit voort.
Indirekt zijn ze op de vorige bladzijden al ter sprake geweest. We zetten de voornaamste, voor zover zij de groepsvergadering betreffen, nu nog even op een rij.
\begin{itemize}
\item    Verwacht mag worden, dat alle deelnemers met funkties
     (verslaglegging. secretariaat, archi\-ve\-ring) loyaal met de
     voorzitter meewerken en in goed overleg steeds tijdig hun
     plichten vervullen.
\item    Alle deelnemers behoren zich voor te bereiden op de bijeenkomst.
         Dat begint al met ''kleinigheden'' als op tijd aanwezig zijn. In feite is het opzettelijk of nonchalant
          te laat komen een bele\-di\-ging jegens de andere deelnemers en de voorzitter, die wel de moeite hebben
          genomen om op tijd te komen; ''het vergeten'' van een
          vergadering getuigt evenmin van een juiste instelling, nog afgezien van het feit dat zo'n vergadering
          daardoor kan inboeten aan doelmatigheid. Er ontstaat
          vaak extra werk in een later stadium.
         De echte voorbereiding omvat natuurlijk ook het zich
          bezinnen op de agendapunten: het tevoren bestuderen
          van documentatie-materiaal; het prepareren van bijdragen, niet alleen qua inhoud, maar ook qua presentatie en formulering, het verzamelen van (eventueel
          visueel) bewijsmateriaal bij argumenten, etc.
\item    Alle deelnemers die in een vorige vergadering taken
          opgedragen gekregen hebben, dienen deze voor de
          volgende nauwgezet te volbrengen.
\item    Gedurende de vergadering zelf mag niet alleen van de
     voorzitter, maar ook van de deelnemers verwacht worden
     dat zij aantekeningen maken; daardoor kunnen veel overbodige herhalingen en onnodige afdwalingen voorkomen worden.

\item    Deelnemers moeten zich tijdens de vergadering onthouden
     van gedrag dat een bijeenkomst nodeloos lang doet duren,
     het goede klimaat verstoort en irritatie wekt bij de
     andere aanwezigen.
     Zij moeten daarom o.a.:
\begin{description}
     \item[-]    goed naar elkaar luisteren
     \item[-]    elkaar laten uitpraten en niet steeds in de rede
          vallen
     \item[-]    meewerken aan het binnen de vastgestelde tijdgrenzen
          blijven (niet het woord nemen als er niets is aan te
          vullen op wat anderen al gezegd hebben; mensen die
          omslachtig vorige sprekers herhalen en er vervolgens
          aan toevoegen dat ze het er helemaal mee eens zijn,
          zijn beruchte tijdvreters).
     \item[-]    bij onverhoopt uitlopen van de vergadering niet er
          vandoor gaan of met tekenen van ongeduld en bedekte
          verwensingen ''de tijd uitzitten'' (dit geldt ook voor
          de rondvraag).
     \item[-]    loyaal meewerken (bij democratisch genomen besluiten) aan de besluitvoering, ook al was men aanvankelijk tegen.
\end{description}
\item    Ieder groepslid heeft tot taak, gedurende de vergadering
     de voorzitter te steunen door het vervullen van informele
     leiders-taken, voorzover dat in zijn vermogen ligt en het
     gewenst is.
\end{itemize}

\subsection{Praktische tips voor de organisatie van de projektgroep}
\begin{itemize}

\item   Duidelijkheid is een eerste vereiste voor een goede 
     organisatie.
\item
  Met heldere afspraken kan een belangrijk deel van de
     duidelijkheid bereikt worden.
\item
   Wanneer dan iedereen zich aan deze afspraken houdt, is
     reeds veel gewonnen.
\item
   Voortgangskontrole kan dit bevorderen.
\item
   In het Projektonderwijs hangt veel af van de vergaderingen van de projektgroep.
\item
   Een goede voorbereiding op deze vergaderingen is van veel
     belang voor het slagen ervan.
\item
   De voorzitter moet deze vergadering primair voorbereiden:
\begin{description}
     \item[-]    agenda samenstellen
     \item[-]    problemen formuleren
     \item[-]    stukken tevoren rondzenden.
\end{description}
\item
   Er moeten harde afspraken gemaakt worden: wie doet wanneer wat.
\item
   In de notulen dienen deze afspraken te worden vastgelegd.
     Uitgebreide notulen zijn meestal niet nodig, zelfs ongewenst want dan worden ze niet meer gelezen. Leg geen hele
     discussies vast! Hooguit de konklusies.
\item
  Loop bij de start van de volgende vergadering deze afspraken na bij wijze van voortgangs\-kon\-tro\-le:
  heeft iedereen zijn taak gedaan? Kan iedereen voldoende vooruit?
     Zullen de gegevens op tijd aanwezig zijn?
\item
  Probeer taken zo vroeg mogelijk te verdelen zodat men
     zich daar vast op kan voorbereiden.
\item
  Hanteer, zowel bij het maken van afspraken als bij de
     voortgangskontrole, het tijdschema als uitgangspunt.
\end{itemize}

\subsection{De procedure voor probleemoplossing en besluitvorming}
\label{besluitvorming}
Voor een correcte probleemoplossing en besluitvorming geldt de
volgende procedure in vijf fasen:
\begin{enumerate}
\item    In de eerste fase, die van de beeldvorming of probleemstelling zorgt de voorzitter ervoor, dat allen een helder
     beeld krijgen van wat het probleem precies inhoudt. Dit
     is nl. lang niet altijd voor alle deelnemers onmiddellijk
     duidelijk.
     Hierbij kunnen de volgende vragen de revue passeren:
\begin{itemize}
     \item    hoe en in welke context is het probleem ontstaan?
          (oorzaken)
     \item    voor wie is het een probleem?
     \item    is het een eenmalig of structureel probleem?
     \item    is alle informatie beschikbaar? Welke gegevens ontbreken nog?
     \item    is het probleem in eerste instantie correct geformuleerd? Wat ontbreekt?
\end{itemize}     
     De beantwoording van deze vragen brengt vaak nieuwe facetten aan het licht of vereenvoudigt het aanvankelijk
     verwarde beeld.
     Aan het einde van de beeldvormingsfase blijkt vaak, als
     resultaat van de discussie, herformulering van het probleem noodzakelijk; in ieder geval moeten alle deelnemers
     nu een volledig en identiek beeld hebben van het probleem: onvolledige beeldvorming kan zich wreken tijdens
     de volgende fasen.
     Het kan nuttig zijn, het probleem in zijn definitieve
     formulering op het bord te schrijven.
\item    In de tweede fase (creatieve fase of brainstorming) leidt
     de voorzitter het zoeken van de groep naar mogelijke
     oplossingen.
     Hij benadrukt dat de leden zich niet hoeven te beperken
     tot pasklare oplossingen; iedere suggestie is welkom.
     Wellicht kunnen ze in een latere fase gekombineerd worden
     tot een werkelijke oplossing. Creatief en associatief
     denken is tijdens de brainstorming zeer welkom.
     De brainstorming kan frappante resultaten opleveren op
     voorwaarde dat de voorzitter niet toestaat, dat de deelnemers onmiddellijk kommentaar geven op voorstellen van
     andere deelnemers. Dit is noodzakelijk. Een brainstorming
     is gedoemd te mislukken als deelnemers zich geremd voelen
     in hun creatief denkproces door mogelijk schampere reacties. De kritische afweging komt pas in fase 3; de creatieve fase is bedoeld voor het vrij genereren van idee\"en,
     en het elkaar enthousiasmeren in een alles-is-mogelijk-
     niets-is-gek-klimaat.
     De voorzitter noteert voor allen duidelijk zichtbaar op
     het bord alle suggesties en idee\"en die geopperd worden.
     Als niemand meer iets aan de inventarisatie heeft toe te
     voegen, gaat hij over naar de derde fase.
\item    In de derde fase (concentratie fase) begint de voorzitter
     met het samen met de groep formuleren van de criteria
     waaraan oplossingen moeten voldoen.
     Dit zou ook aan het einde van de eerste fase kunnen gebeuren. De kans is dan echter groot dat de groepsleden
     door hun vroegtijdige kennis van deze criteria zich tijdens de creatieve fase geremd voelen en niet durven komen
     met ''gewaagde'' idee\"en.
     Vervolgens worden de suggesties aan de criteria getoetst
     en waar mogelijk deeloplossingen met elkaar gekombineerd.
     Eerst laat men nu die voorstellen vallen, die het minst
     aan de criteria voldoen. Al discussi\"erend, kombinerend en
     schrappend komt men tot een steeds strengere selektie,
     tot er waarschijnlijk enkele haalbare oplossingen overblijven.
     Is geen enkele oplossing bruikbaar, dan zal men de criteria ruimer moeten stellen en het laatste deel van de
     procedure moeten herhalen.
     De voorzitter ziet erop toe dat ook tijdens deze fase
     alle groeps\-leden in de gelegenheid zijn hun visie te
     geven of hun bijdrage te leveren. Dat betekent niet dat
     iedereen iets moet zeggen, maar wel dat iedereen iets
     moet kunnen zeggen.
     Ook ziet hij toe op logische redeneerwijze en argumentatie: voor- en tegenstanders van bepaalde oplossingen
     willen nog wel eens met louter emotionele middelen hun
     standpunt verdedigen of het gelijk aan hun zijde krijgen.
\item    In fase vier (besluitvorming) wordt de definitieve
     oplossing gekozen (het besluit genomen).
     Zonodig herhaalt de voorzitter de overgebleven oplossingen, schrijft ze op het bord met daarnaast de criteria en
     konsekwenties. Alle pro's en kontra's van de oplossingen
     worden tegen elkaar afgewogen (eventueel ook op het bord
     schrijven, ze zijn dan beter met elkaar te ver\-ge\-lij\-ken).
     Tenslotte valt het besluit.
     Het prettigst is het, als het besluit unaniem genomen
     wordt; dan is iedereen voor dezelfde oplossing en zal
     zich ook geheel voor de uitvoering van het besluit inzetten. Is unanimiteit niet haalbaar, dan is consensus het
     meest aangewezen principe. Dat betekent dat niemand zich
     tegen de keuze van een bepaalde oplossing verzet.
     Zijn er wel tegenstanders, dan wordt het besluit genomen
     bij meerderheid van stemmen.

\item    Is het besluit eenmaal genomen, dan dient de groep zich
     nog te bezinnen op de uitvoering ervan (vijfde fase:
     besluit-uitvoering). Er dienen afspraken gemaakt en regelingen getroffen te worden (wie doet wat voor wanneer?).
     De voorzitter ziet erop toe dat deze afspraken en regelingen in de notulen en het verslag worden opgenomen.
\end{enumerate}
     De loyaliteit eist dat groepsleden die aanvankelijk tegen
     het genomen besluit waren, zich achter het besluit stellen (mits het uiteraard op correcte wijze genomen is) en
     meewerken aan de besluit-uitvoering. Wellicht is het voor
     de voorzitter mogelijk, bij de taakverdeling rekening te
     houden met de aanvankelijke standpuntbepaling van die
     groepsleden.

\subsection{Evaluatie}

De evaluatie beoogt beschrijving en waardering van de wijze
waarop gewerkt wordt en, daaruit volgend, - voorzover nodig en
mogelijk - verbetering van de activiteiten en het gedrag.

Evalueren is eigenlijk een vorm van feedback geven.
We zouden kunnen zeggen dat bij de evaluatie feedback wordt
gegeven over alle aspekten van het groepswerk: alle activiteiten worden gewaardeerd,
en wel aan de hand van criteria (normen),
die impliciet of expliciet worden ontleend aan de doel\-stel\-ling\-en van de groep.

Voorwaarde voor een geslaagde evaluatie is, dat er een open
sfeer heerst, waarin ieder groepslid alles ter sprake moet
kunnen brengen wat hem op het hart ligt, zonder dat hem dat
wordt kwalijk genomen, dus zonder dat andere groepsleden zich
persoonlijk gegriefd voelen. Men moet zijn opmerkingen echter
wel kunnen beargumenteren, d.w.z.\ kunnen zeggen wat men heeft
waargenomen (de voor allen zichtbare, objektieve feiten) en
hoe men het waargenomene heeft ge{\ii}nterpreteerd.
Waarneming en interpretatie moeten dus steeds duidelijk onderscheiden worden. Tijdens de bespreking blijkt dan wel of de
interpretatie van de een overeenkomt met wat de ander bedoelde

\cleardoublepage
