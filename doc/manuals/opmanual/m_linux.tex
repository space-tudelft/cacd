\section{Enige aanwijzingen voor het gebruik van Linux}
\index{linux}
\label{linux}
\subsection{Eerste maal inloggen}
Voor het practicum moet worden ingelogd onder linux,
gebruikmakende van je NetID gebruikersnaam en password.
Voor problemen hiermee kun je terecht bij het EWI steunpunt
bij de portiersloge in het gebouw aan de Drebbelweg.

Na het inloggen moet een terminal window worden worden geopend
en moet daarin het volgende commando worden getypt:
\begin{verbatim}
   source /data/public/common/software/opprog/bin/op_init
\end{verbatim}
Dit zorgt ervoor, dat alle paden naar de te gebruiken
programmatuur worden gezet.
Alle programma's die tijdens het practicum gebruikt worden
zijn nu bereikbaar.
Deze procedure behoeft slechts eenmalig bij de eerste keer inloggen
te worden uitgevoerd.

\subsection{Enkele linux commando's}
Enkele veelgebruikte linux commando's die in een terminal window gegeven kunnen worden zijn:
\begin{description}
\setlength{\itemsep}{0cm}
\item[ls]    
Toon de inhoud van een de huidige directory.
De optie -l geeft details over elke file en subdirectory.
\item[mkdir]   
Maak een nieuwe directory. \\
mkdir $<$dir\_name$>$ \hspace*{0.5cm}maak een een directory $<$dir\_name$>$ onder de huidige directory.
\item[cd]
Verander van directory. \\
cd \hspace*{3.2cm}ga naar de home\_directory \\
cd .. \hspace*{2.8cm}         ga naar de parent\_directory \\
cd $<$dir\_name$>$ \hspace*{1.0cm} ga naar de opgegeven directory 
\item[cp]    
Copieer een file. \\
cp $<$file\_from$>$ $<$file\_to$>$ \hspace*{0.5cm}maak een copie van $<$file\_from$>$ onder de naam $<$file\_to$>$ \\
cp $<$file\_from$>$ $<$dir\_to$>$ \hspace*{0.58cm}maak een copie van $<$file\_from$>$ in directory $<$dir\_to$>$ \\
\hspace*{4.7cm}met de naam $<$file\_from$>$ 
\item[rm]   
Verwijder een file. \\
rm $<$file\_name$>$ \hspace*{1.2cm} verwijder de file $<$file\_name$>$ uit het file\_systeem. \\
rm -rf $<$dir\_name$>$ \hspace*{0.8cm} verwijder de directory $<$dir\_name$>$ en al zijn files en subdirectories \\
\hspace*{4cm}uit het file\_systeem
\item[man]    
Geef uitleg over een commando. \\
man rm \hspace*{2.6cm} geef uitleg over het commando rm
\end{description}
Voor file en directory namen kunnen pattern matching karakters zoals het
karakter * gebruikt worden.
Bijvoorbeeld, het volgende commando copieert alle files die eindigen op .vhd naar 
een directory proj/VHDL
\begin{verbatim}
   cp *.vhd proj/VHDL
\end{verbatim}
\cleardoublepage
