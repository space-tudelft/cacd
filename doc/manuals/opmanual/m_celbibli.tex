\section{De OP Cellenbibliotheek (oplib)}

\subsection{Inleiding}

In dit hoofdstuk zal een korte beschrijving gegeven worden van de bibliotheekcellen die beschikbaar zijn voor het ontwerppracticum (OP).
De beschrijving bestaat voor elke cel uit:
\begin{itemize}
\item
Functie
\item
Terminal aansluitingen
\item
IEC-symbool
\item
Waarheidstabel
\item
Timing parameters
\item
Equivalent chip oppervlak
\item
Fanout
\end{itemize}

Er zijn vier soorten timing parameters:
\begin{tabbing}
xxxxxxxxxxxxxxxxxx\=\kill
$T_{PLH}$ en $T_{PHL}$\> De vertragingstijd van de cel bij eenheidsbelasting (0.12 pF)\\
\\
${\Delta}T_{PLH}$ en ${\Delta}T_{PHL}$\> De vertragings-co\"effici\"ent bij capacitieve belasting\\
\\
$C_{in}$\> De ingangscapaciteit\\
\\
$T_{su}$ en $T_{hold}$\> Setup- en hold-tijden van de flipflops
\end{tabbing}
De vertragings-tijden en -co\"effici\"enten worden afzonderlijk gedefinieerd voor een opgaande (LH) en neergaande (HL) flank van het uitgangssignaal. 
De totale vertragingstijd wordt berekend met de formule:
\begin{description}
\item
$T_{\it P_{tot}} = T_{\it P} + \Delta T_{\it P}(C_{\it load}-C_{\it unit})$
\end{description}
waarbij $C_{load}$ de belastingscapaciteit voorstelt en $C_{unit}$ de eenheidsbelasting.\\
Deze belastingscapaciteit is de som van de ingangscapaciteit van de aangestuurde cellen plus de capaciteit van de verbindingsdraden.
 
De eenheid van (equivalent) chip oppervlakte is gedefinieerd als het kleinst mogelijke stukje van het fishbone image en bestaat uit \'e\'en nmos en \'e\'en pmos transistor. De grootte hiervan is ongeveer 800 $\mu$m$^{2}$.

De fanout is de belasting waarboven de totale vertragingstijd {\it afneemt} als een buffer wordt toegevoegd.\\

\clearpage
\subsection{De cellen}
\section{iv110}

Function: Inverter

Terminals: (A, Y, vss, vdd)


IEC-symbol:
%figuur iv110.iec.eps
\begin{figure}[bth]
\callpsfig{iv110.iec.eps}{width=.31\textwidth}
\end{figure}

Function table:
\begin{table}[bth]
\begin{tabular}{|c||c|}
\hline
A	&Y\\
\hline
L	&H\\
H	&L\\
\hline
\end{tabular}

\vspace{1cm}

Propagation and load dependent delays:\\

\begin{tabular}{|c|c|c|c|c|}
\hline
Parameter               &From            &To   &Typ    &Unit\\
\hline
T$_{PLH}$               &A     		&Y      &0.4    &ns\\
T$_{PHL}$               &A    		&Y      &0.3    &ns\\
\hline
$\Delta$T$_{PLH}$       &A           	&Y      &2.2    &ns/pF\\
$\Delta$T$_{PHL}$       &A           	&Y      &2.0    &ns/pF\\
\hline
C$_{in}$                &A	    	&vss    &0.12   &pF\\
\hline
\end{tabular}
\end{table}

Equivalent chip area: 2






%figuur iv110.lay.eps
\begin{figure}[bth]
layout:\\
 
\callpsfig{iv110.lay.eps}{height=5in}

\end{figure}


\clearpage

\subsection{no210}

Function: 2-input nor

Terminals: (A, B, Y, vss, vdd)


IEC-symbol:
%figuur no210.iec.eps
\begin{figure}[bth]
\callpsfig{no210.iec.eps}{width=.31\textwidth}
\end{figure}

\begin{minipage}[t]{0.3\textwidth}
Function table:\\

\begin{tabular}{|c|c||c|}
\hline
A	&B	&Y\\
\hline
L	&L	&H\\
-	&H	&L\\
H	&-	&L\\
\hline
\end{tabular}
\end{minipage}
\hfill
\begin{minipage}[t]{0.6\textwidth}
Timing parameters:\\

\begin{tabular}{|c|cc|ccc|c|}
\hline
Parameter		&From	&To	&Min	&Typ	&Max	&Unit\\
\hline
T$_{PLH}$               &A     	&Y      &0.23	&0.34	&0.59   &ns\\
T$_{PHL}$               &A    	&Y      &0.10	&0.25	&0.38   &ns\\
T$_{PLH}$               &B     	&Y      &0.21	&0.32	&0.59   &ns\\
T$_{PHL}$               &B    	&Y      &0.10	&0.22	&0.40   &ns\\
\hline
$\Delta$T$_{PLH}$       &any    &Y      &-	&-	&1.8   &ns/pF\\
$\Delta$T$_{PHL}$       &any    &Y      &-	&-	&0.8   &ns/pF\\
\hline
C$_{in}$                &any	&vss    &-	&0.12	&0.18   &pF\\
\hline
\end{tabular}
\end{minipage}
\\

Equivalent chip area: 3

Fanout: 2.4pF

\makebox[0.98\textwidth]{
%figuur no210.cir.eps
\begin{minipage}[t]{0.5\textwidth}{
circuit:\\
\\

{\callpsfig{no210.cir.eps}{height=3in}}
}
\end{minipage}
\hfill
\begin{minipage}[t]{0.45\textwidth}{
layout:\\
\\

%figuur no210.lay.eps
{\callpsfig{no210.lay.eps}{height=3in}}
}
\end{minipage}
}

\clearpage

\section{no310}

Function: 3-input nor

Terminals: (A, B, C, Y, vss, vdd)


IEC-symbol:
%figuur no310.iec.eps
\begin{figure}[bth]
\callpsfig{no310.iec.eps}{width=.31\textwidth}
\end{figure}

Function table:
\begin{table}[bth]
\begin{tabular}{|c|c|c||c|}
\hline
A	&B	&C	&Y\\
\hline
L	&L	&L	&H\\
-	&-	&H	&L\\
-	&H	&-	&L\\
H	&-	&-	&L\\
\hline
\end{tabular}
\vspace{1cm}

Propagation and load dependent delays:\\

\begin{tabular}{|c|c|c|c|c|}
\hline
Parameter               &From            &To   &Typ    &Unit\\
\hline
T$_{PLH}$               &A     		&Y      &1.4    &ns\\
T$_{PHL}$               &A    		&Y      &0.5    &ns\\
T$_{PLH}$               &B     		&Y      &1.3    &ns\\
T$_{PHL}$               &B    		&Y      &0.4    &ns\\
T$_{PLH}$               &C     		&Y      &1.1    &ns\\
T$_{PHL}$               &C    		&Y      &0.3    &ns\\
\hline
$\Delta$T$_{PLH}$       &any          	&Y      &5.0    &ns/pF\\
$\Delta$T$_{PHL}$       &any           	&Y      &10.0    &ns/pF\\
\hline
C$_{in}$                &any	    	&vss    &0.12   &pF\\
\hline
\end{tabular}
\end{table}


Equivalent chip area: 4





%figuur no310.lay.eps
\begin{figure}[bth]
layout:\\

\callpsfig{no310.lay.eps}{height=5in}
\end{figure}


\clearpage

\subsection{na210}

Function: 2-input nand

Terminals: (A, B, Y, vss, vdd)


IEC-symbol:
%figuur na210.iec.eps
\begin{figure}[bth]
\callpsfig{na210.iec.eps}{width=.31\textwidth}
\end{figure}

\begin{minipage}[t]{0.3\textwidth}
Function table:\\

\begin{tabular}{|c|c||c|}
\hline
A	&B	&Y\\
\hline
-	&L	&H\\
L	&-	&H\\
H	&H	&L\\
\hline
\end{tabular}
\end{minipage}
\hfill
\begin{minipage}[t]{0.6\textwidth}
Timing parameters:\\

\begin{tabular}{|c|cc|ccc|c|}
\hline
Parameter               &From            &To	&Min   	&Typ	&Max	&Unit\\
\hline
T$_{PLH}$               &A     		&Y      &0.08	&0.33	&0.50    &ns\\
T$_{PHL}$               &A    		&Y      &0.19	&0.29	&0.52    &ns\\
T$_{PLH}$               &B     		&Y      &0.08	&0.20	&0.36    &ns\\
T$_{PHL}$               &B    		&Y      &0.20	&0.30	&0.52    &ns\\
\hline
$\Delta$T$_{PLH}$       &any          	&Y      &-	&-	&1.0    &ns/pF\\
$\Delta$T$_{PHL}$       &any           	&Y      &-	&-	&1.1    &ns/pF\\
\hline
C$_{in}$                &any	    	&vss    &-	&0.12	&0.18   &pF\\
\hline
\end{tabular}
\end{minipage}
\\

Equivalent chip area: 3

Fanout: 2.9pF

\makebox[0.98\textwidth]{
%figuur na210.cir.eps
\begin{minipage}[t]{0.5\textwidth}{
circuit:\\
\\

{\callpsfig{na210.cir.eps}{height=3in}}
}
\end{minipage}
\hfill
\begin{minipage}[t]{0.45\textwidth}{
layout:\\
\\

%figuur na210.lay.eps
{\callpsfig{na210.lay.eps}{height=3in}}
}
\end{minipage}
}

\clearpage

\subsubsection{na310}

Functie: 3-input nand

Terminals: (A, B, C, Y, vss, vdd)


IEC-symbool:
%figuur na310.iec.eps
\begin{figure}[bth]
\callpsfig{na310.iec.eps}{width=.25\textwidth}
\end{figure}

\begin{minipage}[t]{0.3\textwidth}
Waarheidstabel:\\

\begin{tabular}{|c|c|c||c|}
\hline
A	&B	&C	&Y\\
\hline
-	&-	&L	&H\\
-	&L	&-	&H\\
L	&-	&-	&H\\
H	&H	&H	&L\\
\hline
\end{tabular}
\end{minipage}
\hfill
\begin{minipage}[t]{0.6\textwidth}
Timing parameters:\\

\begin{tabular}{|c|cc|ccc|c|}
\hline
Parameter               &From            &To	&Min	&Typ	&Max    &Unit\\
\hline
T$_{PLH}$               &A     		&Y      &0.06	&0.37	&0.61    &ns\\
T$_{PHL}$               &A    		&Y      &0.26	&0.39	&0.77    &ns\\
T$_{PLH}$               &B     		&Y      &0.06	&0.35	&0.58    &ns\\
T$_{PHL}$               &B    		&Y      &0.27	&0.40	&0.77    &ns\\
T$_{PLH}$               &C     		&Y      &0.06	&0.21	&0.37    &ns\\
T$_{PHL}$               &C    		&Y      &0.26	&0.39	&0.77    &ns\\
\hline
$\Delta$T$_{PLH}$       &any          	&Y      &-	&-	&0.9    &ns/pF\\
$\Delta$T$_{PHL}$       &any           	&Y      &-	&-	&1.4    &ns/pF\\
\hline
C$_{in}$                &any	    	&vss    &-	&0.12   &0.18	&pF\\
\hline
\end{tabular}
\end{minipage}
\\

Equivalent chip oppervlak: 4

Fanout: 2.5 pF

\makebox[0.98\textwidth]{
%figuur na310.cir.eps
\begin{minipage}[t]{0.5\textwidth}{
Circuit:\\
\\
{\callpsfig{na310.cir.eps}{height=2.5in}}
}
\end{minipage}
\hfill
\begin{minipage}[t]{0.45\textwidth}{
%figuur na310.lay.eps
Layout:\\
\\

{\callpsfig{na310.lay.eps}{height=2.5in}}
}
\end{minipage}
}
\clearpage

\section{ex210}

Function: Exclusive or

Terminals: (A, B, Y, vss, vdd)


IEC-symbol:
%figuur ex210.iec.eps
\begin{figure}[bth]
\callpsfig{ex210.iec.eps}{width=.31\textwidth}
\end{figure}

Function table:
\begin{table}[bth]
\begin{tabular}{|c|c||c|}
\hline
A	&B	&Y\\
\hline
L	&L	&L\\
L	&H	&H\\
H	&L	&H\\
H	&H	&L\\
\hline
\end{tabular}
\vspace{1cm}



Propagation and load dependent delays:\\

\begin{tabular}{|c|c|c|c|c|}
\hline
Parameter               &From            &To   &Typ    &Unit\\
\hline
T$_{PLH}$               &A     		&Y      &0.7    &ns\\
T$_{PHL}$               &A    		&Y      &0.3    &ns\\
T$_{PLH}$               &B     		&Y      &0.6    &ns\\
T$_{PHL}$               &B    		&Y      &0.2    &ns\\
\hline
$\Delta$T$_{PLH}$       &any          	&Y      &7.0    &ns/pF\\
$\Delta$T$_{PHL}$       &any           	&Y      &5.0    &ns/pF\\
\hline
C$_{in}$                &any	    	&vss    &0.12   &pF\\
\hline
\end{tabular}
\end{table}

Equivalent chip area: 7

%figuur ex210.cir.eps
\begin{figure}[bth]
circuit:\\

\callpsfig{ex210.cir.eps}{width=.65\textwidth}
\end{figure}




%figuur ex210.lay.eps
\begin{figure}[bth]
layout:\\

\callpsfig{ex210.lay.eps}{height=5in}
\end{figure}


\clearpage

\subsection{buf40}
\index{buffer!buf40}
Function: Buffer (4x-drive)

Terminals: (A, Y, vss, vdd)


IEC-symbol:
%figuur buf40.iec.eps
\begin{figure}[bth]
\callpsfig{buf40.iec.eps}{width=.31\textwidth}
\end{figure}

\begin{minipage}[t]{0.3\textwidth}
Function table:\\

\begin{tabular}{|c||c|}
\hline
A	&Y\\
\hline
L	&L\\
H	&H\\
\hline
\end{tabular}
\end{minipage}
\hfill
\begin{minipage}[t]{0.6\textwidth}
Timing parameters:\\

\begin{tabular}{|c|cc|ccc|c|}
\hline
Parameter               &From            &To   	&Min	&Typ	&Max    &Unit\\
\hline
T$_{PLH}$               &A     		&Y      &0.46	&0.69	&1.0    &ns\\
T$_{PHL}$               &A    		&Y      &0.56	&0.84	&1.3    &ns\\
\hline
$\Delta$T$_{PLH}$       &A           	&Y      &-	&-	&0.3    &ns/pF\\
$\Delta$T$_{PHL}$       &A           	&Y      &-	&-	&0.4    &ns/pF\\
\hline
C$_{in}$                &A	    	&vss    &-	&0.12	&0.18   &pF\\
\hline
\end{tabular}
\end{minipage}
\\

Equivalent chip area: 6

Fanout: 12pF

\vspace{1cm}
\makebox[0.98\textwidth]{
%figuur buf40.cir.eps
\begin{minipage}[t]{0.70\textwidth}{
circuit:\\
\\

{\callpsfig{buf40.cir.eps}{height=2in}}
}
\end{minipage}
\hfill
\hspace{1cm}
\begin{minipage}[t]{0.3\textwidth}{
layout:\\
\\

{\callpsfig{buf40.lay.eps}{height=3in}}
}
\end{minipage}
}
\clearpage

\subsubsection{mu111}

Functie: 2-line-to-1-line data selector/multiplexer

Terminals: (A, B, S, Y, vss, vdd)


IEC-symbool:
%figuur mu111.iec.eps
\begin{figure}[bth]
\callpsfig{mu111.iec.eps}{width=.25\textwidth}
\end{figure}

\begin{minipage}[t]{0.3\textwidth}
Waarheidstabel:\\

\begin{tabular}{|c|cc||c|}
\hline
S	&A	&B	&Y\\
\hline
L	&L	&-	&L\\
L	&H	&-	&H\\
H	&-	&L	&L\\
H	&-	&H	&H\\
\hline
\end{tabular}
\end{minipage}
\hfill
\begin{minipage}[t]{0.6\textwidth}
Timing parameters:\\

\begin{tabular}{|c|cc|ccc|c|}
\hline
Parameter               &From            &To   	&Min	&Typ	&Max    &Unit\\
\hline
T$_{PLH}$               &A, B     	&Y      &0.31	&0.50	&0.81    &ns\\
T$_{PHL}$               &A, B    	&Y      &0.37	&0.60	&0.98    &ns\\
T$_{PLH}$               &S         	&Y      &0.37	&0.60	&0.95    &ns\\
T$_{PHL}$               &S         	&Y      &0.37	&0.65	&1.1    &ns\\
\hline
$\Delta$T$_{PLH}$       &any           &Y      &-	&-	&1.2    &ns/pF\\
$\Delta$T$_{PHL}$       &any           &Y      &-	&-	&1.0    &ns/pF\\
\hline
C$_{in}$                &A, B    	&vss   &-	&0.12	&0.18   &pF\\
C$_{in}$                &S             	&vss   &-	&0.24	&0.36   &pF\\
\hline
\end{tabular}
\end{minipage}
\\

Equivalent chip oppervlak: 7

Fanout: 3.0 pF

\makebox[0.98\textwidth]{
%figuur mu111.cir.eps
\begin{minipage}[t]{0.5\textwidth}{
Circuit:\\
\\

{\callpsfig{mu111.cir.eps}{height=2.5in}}
}
\end{minipage}
\hfill
\begin{minipage}[t]{0.45\textwidth}{
%figuur mu111.lay.eps
Layout:\\
\\

{\callpsfig{mu111.lay.eps}{height=2.5in}}
}
\end{minipage}
}
\clearpage

\subsection{mu210}

Function: 4-line-to-1-line data selector/multiplexer

Terminals: (S1, S2, A, B, C, D, Y, vss,vdd)


IEC-symbol:
%figuur mu210.iec.eps
\begin{figure}[bth]
\callpsfig{mu210.iec.eps}{width=.31\textwidth}
\end{figure}

Function table:
\begin{table}[bth]
\begin{tabular}{|cc|cccc||c|}
\hline
S2	&S1	&A 	&B	&C	&D	&Y\\
\hline
L	&L	&L	&-	&-	&-	&L\\
L	&L	&H	&-	&-	&-	&H\\
L	&H	&- 	&L	&-	&-	&L\\
L	&H	&-	&H	&-	&-	&H\\
H	&L	&-	&-	&L	&-	&L\\
H	&L	&-	&-	&H	&-	&H\\
H	&H	&-	&-	&-	&L	&L\\
H	&H	&-	&-	&-	&H	&H\\
\hline
\end{tabular}
\vspace{.5cm}

	
Timing parameters:\\

\begin{tabular}{|c|cc|ccc|c|}
\hline
Parameter		&From		&To 	&Min	&Typ	&Max	&Unit\\
\hline
T$_{PLH}$		&A, B, C, D, S1	&Y	&0.88	&1.3	&2.0	&ns\\
T$_{PHL}$		&A, B, C, D, S1	&Y	&1.4	&2.1	&3.2	&ns\\
T$_{PLH}$		&S2		&Y	&0.77	&1.2	&1.7	&ns\\
T$_{PHL}$		&S2		&Y	&0.57	&0.86	&1.3	&ns\\
\hline
$\Delta$T$_{PLH}$	&any		&Y	&-	&-	&1.3	&ns/pF\\
$\Delta$T$_{PHL}$	&any		&Y	&-	&-	&1.4	&ns/pF\\
\hline	
C$_{in}$		&A, B, C, D	&vss	&-	&0.12	&0.18	&pF\\
C$_{in}$		&S1		&vss	&-	&0.36	&0.54	&pF\\
C$_{in}$		&S2		&vss	&-	&0.24	&0.36	&pF\\
\hline
\end{tabular}
\end{table}

Equivalent chip area: 16

Fanout: 3.0pF



%figuur mu210.cir.eps
\begin{figure}[bth]
circuit:\\

\callpsfig{mu210.cir.eps}{height=4in}
\end{figure}



%figuur mu210.lay.eps
\begin{figure}[bth]
layout:\\

\callpsfig{mu210.lay.eps}{height=4in}
\end{figure}


\clearpage







\subsubsection{de211}

Functie: 2-to-4 decoder/demultiplexer

Terminals: (A, B, Y0, Y1, Y2, Y3, vss, vdd)


IEC-symbool:
%figuur de211.iec.eps
\begin{figure}[bth]
\callpsfig{de211.iec.eps}{width=.31\textwidth}
\end{figure}

Waarheidstabel:
\begin{table}[bth]
\begin{tabular}{|cc||cccc|}
\hline
B	&A	&Y0	&Y1	&Y2	&Y3\\
\hline
L	&L	&H	&L	&L	&L\\
L	&H	&L	&H	&L	&L\\
H	&L	&L	&L	&H	&L\\
H	&H	&L	&L	&L	&H\\	
\hline
\end{tabular}
\vspace{1cm}


Timing parameters:\\

\begin{tabular}{|c|cc|ccc|c|}
\hline
Parameter               &From            &To   &Min	&Typ	&Max    &Unit\\
\hline
T$_{PLH}$               &A     		&Y0,Y2  &0.23	&0.34	&0.51    &ns\\
T$_{PHL}$               &A     		&Y0,Y2  &0.17	&0.25	&0.38    &ns\\
T$_{PLH}$               &A     		&Y1,Y3  &0.41	&0.62	&0.93    &ns\\
T$_{PHL}$               &A     		&Y1,Y3  &0.36	&0.54	&0.81    &ns\\
T$_{PLH}$               &B	    	&Y0,Y1  &0.20	&0.30	&0.45    &ns\\
T$_{PHL}$               &B	    	&Y0,Y1  &0.15	&0.22	&0.33    &ns\\
T$_{PLH}$               &B	    	&Y2,Y3  &0.38	&0.57	&0.87    &ns\\
T$_{PHL}$               &B	    	&Y2,Y3  &0.34	&0.51	&0.76    &ns\\
\hline
$\Delta$T$_{PLH}$       &any           &any   &-	&-	&1.8    &ns/pF\\
$\Delta$T$_{PHL}$       &any           &any   &-	&-	&0.8    &ns/pF\\
\hline
C$_{in}$                &A   		&vss    &-	&0.36	&0.54   &pF\\
C$_{in}$                &B             	&vss    &-	&0.30	&0.45   &pF\\
\hline
\end{tabular}
\end{table}

Equivalent chip oppervlak: 12

Fanout: 2.4 pF

%figuur de211.cir.eps
\begin{figure}[bth]
Circuit:\\

\callpsfig{de211.cir.eps}{height=3in}
\end{figure}



%figuur de211.lay.eps
\begin{figure}[bth]
Layout:\\

\callpsfig{de211.lay.eps}{height=5in}
\end{figure}


\clearpage


\subsubsection{dfn10}
\index{flipflop!dfn10}
Functie: D-type positive edge-triggered flipflop

Terminals: (D, CK, Q, vss, vdd)


IEC-symbool:
%figuur dfn10.iec.eps
\begin{figure}[h]
\callpsfig{dfn10.iec.eps}{width=.31\textwidth}
\end{figure}

Waarheidstabel:
\begin{table}[h]
\begin{tabular}{|c|c||c|}
\hline
D	&CK	&Q\\
\hline
L	&$\uparrow$	&L\\
H	&$\uparrow$	&H\\
\hline
\end{tabular}
\vspace{1cm}

Timing parameters:\\

\begin{tabular}{|c|cc|ccc|c|}
\hline
Parameter               &From            &To   &Min	&Typ	&Max    &Unit\\
\hline
T$_{PLH}$               &CK     	&Q     &1.4	&2.1	&3.2    &ns\\
T$_{PHL}$               &CK    		&Q     &1.3	&1.9	&2.9    &ns\\
\hline
$\Delta$T$_{PLH}$       &CK          	&Q	&-	&-	&1.1    &ns/pF\\
$\Delta$T$_{PHL}$       &CK           	&Q    	&-	&-	&0.8    &ns/pF\\
\hline
T$_{su}$		&D		&CK	&-	&-	&1.4	&ns\\
T$_{hold}$		&D		&CK	&-	&-	&0.9	&ns\\
\hline
C$_{in}$                &D	    	&vss    &-	&0.12	&0.18   &pF\\
C$_{in}$                &CK	    	&vss    &-	&0.18	&0.27   &pF\\
\hline
\end{tabular}
\end{table}

Equivalent chip oppervlak: 17

Fanout: 3.0 pF

\clearpage
%figuur dfn10.cir.eps
\begin{figure}[h]
Circuit:\\

\callpsfig{dfn10.cir.eps}{height=3.6in}

\vspace{0.5cm}
Layout:\\

\callpsfig{dfn10.lay.eps}{height=3.6in}
\end{figure}
\clearpage

\section{dfr11}

Function: D-type positive edge triggered flipflop with synchronous reset

Terminals: (D, R, CK, Q, QN, vss, vdd)


IEC-symbol:
%figuur dfr11.iec.eps
\begin{figure}[bth]
\callpsfig{dfr11.iec.eps}{width=.31\textwidth}
\end{figure}

Function table:
\begin{table}[bth]
\begin{tabular}{|c|c|c||c|c|}
\hline
R	&D	&CK		&Q	&QN\\
\hline
L	&L	&$\uparrow$	&L	&H\\
L	&H	&$\uparrow$	&H	&L\\
H	&-	&$\uparrow$	&L	&H\\
\hline
\end{tabular}
\vspace{1cm}

Propagation and load dependent delays:\\

\begin{tabular}{|c|c|c|c|c|}
\hline
Parameter               &From            &To   &Typ    &Unit\\
\hline
T$_{PLH}$               &CK     	&Q      &1.8    &ns\\
T$_{PHL}$               &CK    		&Q      &2.3    &ns\\
T$_{PLH}$               &CK     	&QN     &1.9    &ns\\
T$_{PHL}$               &CK    		&QN     &2.3    &ns\\
\hline
$\Delta$T$_{PLH}$       &CK          	&any    &8.0    &ns/pF\\
$\Delta$T$_{PHL}$       &CK           	&any    &7.0    &ns/pF\\
\hline
C$_{in}$                &D,R	    	&vss    &0.12   &pF\\
C$_{in}$                &CK	    	&vss    &0.43   &pF\\
\hline
\end{tabular}
\end{table}

Equivalent chip area: 15


\begin{figure}[t]
circuit:\\

\callpsfig{dfr11.cir.eps}{width=1\textwidth}
\end{figure}

\begin{figure}[bth]
layout:\\

\callpsfig{dfr11.lay.eps}{height=4in}
\end{figure}

\clearpage

%\input{lib/dfa11}
\subsubsection{osc10}

Functie: Crystal oscillator with enable

Terminals: (E, F, XI, XO, vss, vdd)


IEC-symbool:
%figuur osc10.iec.eps
\begin{figure}[bth]
\callpsfig{osc10.iec.eps}{width=.4\textwidth}
\end{figure}

Frequentie bereik: 1 MHz t/m 20 MHz

Equivalent chip oppervlak: 16



Beschrijving:

XI en XO zijn de aansluitingen voor het kristal. XI moet gebufferd worden met een inputbuffer, XO dient niet gebufferd te worden.\\
Als E (enable) hoog is, is de oscillator aktief. Als E laag is, is de uitgang F hoog.\\
\\
\\

\makebox[0.98\textwidth]{
%figuur osc10.cir.eps
\begin{minipage}[t]{0.45\textwidth}{
Circuit:\\
\\

{\callpsfig{osc10.cir.eps}{width=0.9\textwidth}}
}
\end{minipage}
\hfill
\begin{minipage}[t]{0.55\textwidth}{
%figuur osc10.lay.eps
Layout:\\
\\

{\callpsfig{osc10.lay.eps}{width=1\textwidth}}
}
\end{minipage}
}
\clearpage





\subsubsection {ln3x3}

Functie: NMOS compoundtransistor, basiscel voor analoge schakelingen

Terminals: (g, d, s, vss, vdd)

Equivalent chip oppervlak: 10

Beschrijving:

De NMOS compoundtransistor ln3x3 bestaat uit 9 NMOS basistransistoren
die in een matrix van 3x3 geschakeld zijn.\\
\\
\\

Circuit: \hspace{0.5\textwidth} Layout:\\


\makebox[0.98\textwidth]{
%figuur ln3x3.eps
{\callpsfig{ln3x3.cir.eps}{width=0.4\textwidth}}
\hfill
%figuur ln3x3.lay.eps
{\callpsfig{ln3x3.lay.eps}{width=0.4\textwidth}}
}

\clearpage

\subsubsection {lp3x3}

Functie: PMOS compoundtransistor, basiscel voor analoge schakelingen

Terminals: (g, d, s, vss, vdd)

Equivalent chip oppervlak: 10

Beschrijving:

De PMOS compoundtransistor lp3x3 bestaat uit 9 PMOS basistransistoren
die in een matrix van 3x3 geschakeld zijn.\\
\\
\\

Circuit: \hspace{0.5\textwidth} Layout:\\


\makebox[0.98\textwidth]{
%figuur lp3x3.eps
{\callpsfig{lp3x3.cir.eps}{width=0.4\textwidth}}
\hfill
%figuur lp3x3.lay.eps
{\callpsfig{lp3x3.lay.eps}{width=0.4\textwidth}}
}
\clearpage

\subsubsection {mir\_nin, mir\_nout}
\index{stroomspiegel!basiscel}
Functie: NMOS spiegel, basiscel voor stroomspiegel

Terminals mir\_nin: (i, g, vss, vdd)

Terminals mir\_nout: (i, g, o, vss, vdd)

Equivalent chip oppervlak: 19

Beschrijving:

De NMOS spiegel is een gecascodeerde stroomspiegel. 
Hij bestaat uit een ingang mir\_nin en een uitgang
mir\_nout.
Beide kunnen in een spiegel meerdere keren gerepeteerd
worden om schaling te verkrijgen.
De gebruikte transistoren zijn zgn. compound transistoren.\\

Circuit:

(a) mir\_nin \hspace{0.2\textwidth} (b) mir\_nout\\

\medskip
%figuur mir_n.eps
\begin{figure}[h]
{\callpsfig{mir_n.eps}{width=.5\textwidth}}
%\caption{}
\end{figure}
\newpage
Layout mir\_nin:
%figuur mir_nin.eps
\begin{figure}[h]
\centerline{\callpsfig{mir_nin.eps}{height=3.2in}}
%\caption{}
\end{figure}

Layout mir\_nout:
%figuur mir_nout.eps
\begin{figure}[h]
\centerline{\callpsfig{mir_nout.eps}{height=3.2in}}
%\caption{}
\end{figure}
\clearpage

\subsubsection {mir\_pin, mir\_pout}
\index{stroomspiegel!basiscel}
Functie: PMOS spiegel, basiscel voor stroomspiegel

Terminals mir\_pin: (i, g, vss, vdd)

Terminals mir\_pout: (i, g, o, vss, vdd)

Equivalent chip oppervlak: 19

Beschrijving:

De PMOS spiegel is een gecascodeerde stroomspiegel. 
Hij bestaat uit een ingang mir\_pin en een uitgang
mir\_pout.
Beide kunnen in een spiegel meerdere keren gerepeteerd
worden om schaling te verkrijgen.
De gebruikte transistoren zijn zgn. compound transistoren.\\

Circuit:

(a) mir\_pin \hspace{0.2\textwidth} (b) mir\_pout\\

\medskip
%figuur mir_p.eps
\begin{figure}[h]
{\callpsfig{mir_p.eps}{width=.5\textwidth}}
%\caption{}
\end{figure}
\newpage
Layout mir\_pin:
%figuur mir_pin.eps
\begin{figure}[h]
\centerline{\callpsfig{mir_pin.eps}{height=3.2in}}
%\caption{}
\end{figure}

Layout mir\_pout:
%figuur mir_pout.eps
\begin{figure}[h]
\centerline{\callpsfig{mir_pout.eps}{height=3.2in}}
%\caption{}
\end{figure}

\clearpage

\subsubsection{bond\_leer}

Functie: Basiscel voor kleine SoG-schakelingen

Terminals: (vss, bf1, bf2, bf3, vdd, bf4, bf5, bf6)

Equivalent chip oppervlak: 800

Maximum grootte schakeling: 350

Aansluitingen: 
\begin{itemize}
\item
Aantal: 8, waarvan 6 vrij te kiezen, 2 voeding (vss, vdd)
\item
Soort beveiliging: protectiedioden, geen buffers
\end{itemize}

Meetmogelijkheden: 8-pins probe-kaart

%figuur bond_leer.lay.eps
\begin{figure}[bth]
Layout:\\

\callpsfig{bond_leer.lay.eps}{width=1\textwidth}
\end{figure}

\clearpage


\subsection{bond\_bar}

Functiion: virtual bonding pattern for one quarter of the chip.

Terminals: (t1, t2, t3, t4, t5, t6, t7, t8, t9, t10, t11, t12, 
                  t13, t14, t15, t16, t17, t18, t19, t20, t21, t22, t23, t24, 
                  t25, t26, t27, t28, t29, t30, t31, t32)

Equivalent chip area: 25000

Maximum size of the circuit: 20000

Pins:
\begin{itemize}
\item
2 power pins (vss, vdd)
\item
2 power buffers
\item
32 user-configurable pins (buffered input, buffered output, buffered
bi-directional, unbuffered direct).
\end{itemize}

Four cells of this type can be placed on one chip. The pin positions
connect to an outer ring of interconnect wires, which connects the
pins to the bonding pads on the border of the chip. The pins of one
cell bond\_bar are spread out over the 144 pins in such a way that
they connect to every 4th bonding pad. Each of the 4 virtual chips
has its own vdd power supply, which is also used to drive the
buffers in the bonding pads. In this way the 4 circuits on the chip
are completely independent. All circuits share one common vss ground
supply connection. For larger circuits, multiple instances of
bond\_bar can be used.

%figuur bond_bar.lay.eps
\begin{figure}[bth]
Layout:\\

\callpsfig{bond_bar.lay.eps}{width=1\textwidth}
\end{figure}

\clearpage


\cleardoublepage

