\subsection{Restricties ontwerpsoftware}
In deze sectie zullen de restricties worden behandeld, die in acht
moeten worden genomen bij het gebruik van de ontwerpsoftware.
\begin{itemize}
\item De 'ports' van entities van \smc{vhdl}-beschrijvingen die moeten worden
      gesynthetiseerd mogen {\it alleen} van het type {\it std\_logic} of
      {\it std\_logic\_vector} zijn.\\
      Dit om te voorkomen dat allerlei conversies moeten worden gemaakt,
      waarbij wellicht niet steeds van dezelfde testbench gebruik kan
      worden gemaakt, en ook conversies van \smc{vhdl} naar sls kunnen mislukken.
\item Gebruik voor de namen van signalen, ports en entities {\it kleine letters}
      Bij de diverse data-omzettingen, die gedurende het ontwerp plaatsvinden
      kunnen anders fouten ontstaan.
\item De namen die worden gebruikt voor terminals en
      circuits dienen onderscheidbaar te zijn in de eerste 14 letters.
\item Voor entities, architectures en configurations moeten aparte files
      worden gemaakt.
\item Geef de layout cellen dezelfde naam als de overeenkomstige circuit cel.
      Dit om bij extractie van
      \smc{vhdl}-code uit het layout-gedeelte naam-problemen te voorkomen.
\item signalen van het type 'inout' mogen alleen worden gebruikt voor
      'analoge' signalen, dus signalen die aan de aangesloten moeten worden
      op een 'direct' buffer aan de rand.
\item Aan ports mogen geen gedeelten van een std\_logic\_vector worden
      aangesloten.
\end{itemize}
De synthese-software voor het genereren van de circuits stelt ook
beperkingen aan de \smc{vhdl}-beschrijvingen.
Naast de algemene aanwijzingen in appendix \ref{good_vhdl} voor het maken van ''synthetiseerbare''
\smc{vhdl}-beschrijvingen noemen we nog de volgende aandachtspunten:
\begin{itemize}
\item Initialisaties van signalen bij hun declaratie worden genegeerd.\\
      Dus bijv. {\it signal a: std\_logic := '0';} wordt gelezen
      als {\it signal a: std\_logic;}.
\item After-clauses in statements worden genegeerd.\\
      Dus bijv. {\it a $<$= '1' after 2 ns;} wordt gelezen als {\it a $<$= '1';}.
\item Process-statements moeten worden gemaakt met een process-list
      waarin, bij een register beschrijving, alleen het klok signaal voorkomt,
      of, bij een beshcrijving van een combinatorische schakeling, {\it alle} 
      signalen voorkomen die tijdens de uitvoering van
      het process worden gelezen.
\item In case- en select-statements moeten {\it alle} mogelijke waarden
      worden behandeld.
\item In iedere tak van case- en if-statements moeten {\it alle}
      uitgangs-signalen een waarde krijgen.
\end{itemize}
Om de tijdvertragingen van de schakelingen te kunnen bekijken is de
nauwkeurigheid van de sls-simulatie default op 100 ps ingesteld.
De maximale tijsduur die dan nog kan worden gesimuleerd, zonder
'overflow'-problemen te krijgen is 100 ms.
{\it Hou hier met het simuleren van de \smc{vhdl} code rekening mee en pas zonodig
de klok-frequentie van de schakeling aan.}
