\selectlanguage{british}
\title{Automatic Routing with Trout}
\maketitle
\label{routeman}
\index{trout@\tool{trout}|(bold}
\index{automatic!routing|see{trout}}
Making all connections by hand using the editor is a nasty and error-sensitive
job. \tool{Trout}, your automatic router, can do this job for you much quicker
and without errors.  One of the most interesting features of \tool{trout} is
that it allows any amount of pre-routed nets and obstacles.  For instance, you
can route some critical part of your circuit by hand, and then let the router
complete the remaining parts. You can also ask the router to make only a part
of the design. Many creative combinations of automatic and manual design are
possible.

Another special property of \tool{trout} is that it routes {\em through} the
cells, rather than around them. This approach is especially suitable for
Sea-of-Gates layout.

\section{What is required before I can click \protect\button{Trout}?}
The router is a very complicated tool,
which requires all kinds of data before
it can start its job.
If any of this data is missing,
it will fail miserably.
Basically it needs the same as \tool{madonna}: a proper netlist which
specifies the desired connectivity of the modules.
But apart from this,
the router expects a placement layout,
in which all instances are placed.
This placement can be made by hand or using \tool{madonna} (or both).
It should contain all instances which were mentioned in the netlist description.
It is not necessary to place the terminals,
because the router will do it itself if they are missing.
It is also not necessary to give the instances the proper instance names.

You can expect an error- or warning-window in the following cases:
\begin{itemize}
\item
One or more instances are missing in the placement. In this
case the circuit will be routed without the terminals on
the missing instances. This will most likely lead to an
incorrect layout.
\item
There are unknown instances in the placement, that is, the
layout contains instances which do not appear in the
circuit description. These instances will be treated as
dummies: apart from the power rails they will not be
connected.  They will only be treated as obstacles.  To
prevent unknown instances, use the \button{set instance name} in
\index{seadali!instance name}
the instances menu.
\item
The placement contains a terminal which is not in the net
list.
\item
If the router fails to route certain nets.
\end{itemize}

\section{Running \protect\tool{trout}}
\section{Calling \protect\tool{trout} from \protect\tool{seadali}}
After pressing \button{Trout}, \tool{seadali} will ask
for the name of the circuit. Then you'll see a small menu.
Just press \button{DO IT !} to start the router. After a short while
the routed circuit should appear on your screen.
Optionally you can set the following parameters in the
menu:
\begin{description}
\item[\button{no power}]
If selected, the horizontal power rails of the circuit will not be connected.
By default all vdd (and vss) rails will be connected to each other by a
vertical wires in \mask{ins}. \tool{Trout} will do this even if the power nets
are not present in the netlist.
\item[\button{erase wires}]
Erase all wires (boxes) and terminals which are present in the layout before
routing. By default the router will only add wires to the existing layout.
\item[\button{border terminals}]
Place all unspecified terminals on the boundary the cell. In this way it is
ensured that all terminals can be reached on a higher hierarchical level. This
button is enabled by default.
\item[\button{route box only}]
After pressing this button you can drag a box over the layout. \tool{Trout}
will only route the nets of which the terminal patterns are in this box. All
wires which are generated by the router (except power) will only be inside the
box. In this way you can locally re-route a large design. On large empty
designs routing in a number of smaller boxes can be considerably faster than
routing in one piece.

Notice that the connectivity verifier (which is run
automatically after \tool{trout}) is very likely to report
errors for unconnected nets because some nets cannot be
routed inside the box.
\item[\button{single pass routing}]
Prevent the trout from retrying in case the routing was not complete.
\item[\button{Option menu}]
Opens a second menu with additional options for the router.
See section~\ref{troutopt} for more details.
\item[\button{DO IT!}]
Start the router. Depending on the size of the circuit and
the load of the system, this will take 5 seconds up to 20
minutes.  The routing time increases quadratically with the
size of the layout and linearly with the number of nets.
\end{description}

\subsection{Stopping the router}
There are two ways to stop \tool{trout} while it is executing from
within \tool{seadali}:
\begin{enumerate}
\item
By clicking \button{KILL} in the menu. This is the brute-force way.
It will immediately kill \tool{trout} and leave the layout
unchanged.
\item
By clicking \button{STOP} in the menu. This will request
\tool{trout} to interrupt the routing as soon as possible and
return the current layout. This will most likely be an incompletely
routed circuit. This button can be useful if routing takes too
long, but you are still interested in the (partial) solution.
\end{enumerate}

\section{Running \protect\tool{trout} from the command line}
The non-interactive way to call \tool{trout} is similar to the way
\index{sea@\tool{sea}}
\tool{madonna} is run: \type{sea -r name} in which {\sl name} is 
the name of the circuit cell. It is assumed that a layout cell {\sl name}
with all instances already exists in the layout view. See
section~\ref{m-sea} for more details on \tool{sea}.

\section{What does \protect\tool{trout} do for me?}
Apart from just routing the nets according to the netlist description, the
router performs a number of other tasks. Most of these features were added just
to make the 'default behavior' of \tool{trout} is as intuitive and useful as
possible. \tool{Trout} performs the following automatically:
\paragraph{Orphan instances}
\index{trout!orphan instances}
The router tries to assign unknown instances ('orphans') to missing instances
if they belong to the same cell type.  For instance: if an instance 'x' of
cell 'nand2' was missing in the placement, but an unknown instance 'y' of cell
'nand2' was found (that is, 'y' is not in the circuit description) then
instance 'y' will be renamed to 'x'. Notice that this does not necessarily
result in a good placement for the instances.  This feature was added for your
convenience during manual placement.
\paragraph{Father terminals}
The terminals of the father cell can be placed automatically by \tool{trout}.
The router automatically places any missing terminal of the father circuit on
the border of the cell or on an optimal position. No message will be printed in
this case. It is better not to place father terminals unless you really want to
force them on a specific position. See section~\ref{routeguidelines} for more
details.
\paragraph{Power terminals}
\index{power terminals}
The router automatically adds the 'vdd' and 'vss' terminals to your layout,
in case they are not yet present yet. \tool{Trout} will do this even
if the circuit does not contain power nets.  
\paragraph{Power rails}
The horizontal vdd- and vss power rails (in metal1) are connected by vertical
wires in metal2. In this way all power terminals are connected to the vdd or
vss terminals, even if the design contains many rows of transistors.  Click
\button{no power} to disable power rails connection.
\paragraph{Connectivity check}
\index{check nets}
\tool{Trout} runs the connectivity verifier automatically after routing. 
If there is any short or unconnect a warning window will pop up and markers
will be placed in the layout. This is the same verifier as under the button
\button{Check nets} (see~\ref{verify} on page~\pageref{verify}).
\section{Guidelines for successful routing}
\label{routeguidelines}
Trout is doing its best to route all nets for you. It can
happen, however, that \tool{trout} is unable to route them
all. Some of the problems may be cause by a design error at
an earlier phase. To ensure the routability of the nets,
you must design de layout of your instances in a certain
way. Since the cell which you are constructing now may also be
used as a leaf cell on a higher lever, it is wise to comply
with the following guidelines:
\begin{itemize}
\item
Make sure that all terminals can be reached by a wire from
the 'outside' of cell. A terminal which cannot be connected
is not useful. The best way to ensure the 'accessibility'
is by placing the terminals on the border of the cell. The
router helps you with that.
\item
Let \tool{trout} take care of the placement of the terminals (the button
\button{border terminals}). If you do not specify the terminal 
positions (e.g. if you pressed \button{erase wires}), \tool{trout} will place
them on the border of the cell. The router will automatically find the shortest
(and most clever) route to the border. In special cases you can always enforce
a terminal placement by placing terminals manually.
\item
Try to make the cell as transparent as possible. Avoid using horizontal wires
in metal2. You will notice that you router itself does not like to generate
this kind of wires.  If \tool{trout} does generate them, that is a sure
indication that area is very congested (close to its maximum routing
capacity).
\item
Leave some space in critical areas by shifting the modules apart. Manually
update your \tool{madonna} placement: she is not too clever.
\item
\index{trout!handling of short circuits}
\tool{Trout} does not make short-circuits by itself, it can only make
\index{short circuits}
existing short-circuit bigger.
If the automatic checker detects a short-circuit after
routing, this generally indicates the presence of an error (short-circuit) in
one of the son-cells.  In many of the cases a terminal in a sub-cell makes a
short circuit with the terminal of another net in that sub-cell. Since
\tool{trout} expands the entire pattern of the terminal, including the entire
short-circuit, connecting more terminals of the net will just make the short
bigger. Unfortunately finding the cause of these short is not too easy, since
the exact position of the cause of the problem cannot be indicated in the
layout.
\end{itemize}

\section{What should I do in the case of incomplete routing?}
Despite all precautions, incomplete routing may still occur.  
\index{trout!incomplete routing}
Basically we can
distinguish two situations for this case:
\begin{enumerate}
\item
Many nets are unconnected. With 'many' we mean more than 5
or more than a few percent.
\item
A few shattered nets are unconnected.
\end{enumerate}
The strategies for these situations will be discussed in
the following sections.

\section{Many nets are unconnected}
In this case the conclusion is simple: the cell must be enlarged.  We have to
live with the fact that the routing capacity on our Sea-of-Gates chip is
limited.  In almost all cases we will have to sacrifice some transistors to
create space for the wires. You can do this by:
\begin{enumerate}
\item
Running \tool{madonna} again in a larger box. 
\item
Running \tool{madonna} again in a box with a different width/height ratio.  In
'flat' horizontal cells the capacity for horizontal wires is easily too small.
In narrow vertical cells there might not be enough space for vertical wires.
Have a look at the density of the wires in each of the layers so see which is
the case. Use
\button{X-expand} or \button{Y-expand} to guide
\tool{madonna}.
\item
Manually replacing some blocks. Try to distribute the
wiring density over the entire cell area.
\end{enumerate}
Prevent spending too much time squeezing the size of your
placement.  Always try the easiest and quickest way first: use \tool{madonna}.

\section{A small number of nets is unconnected}
At this point it is useful to get a little feeling for way in which the router
works. \tool{Trout} routes the nets one at a time in a 'greedy' way.  For each
net he tries to find the best wires which connect the terminals.  The router
is very clever in finding this path.  He uses all tricks in the image (such as
using the polysilicon gates for wires).  You can be sure that if the router
reports that it cannot connect a net, you will also not be able to do it
manually.  Unfortunately the router does not have much overview over the
entire set of nets and wires. Therefore it is possible that the wires of a net
block a terminal of a still unrouted net. This was the cause of the incomplete
routing in the case of a few shattered unconnects.  Notice that this is very
dependent on the order in which the nets were routed. 

The problem can be solved in the following ways:
\begin{enumerate}
\item
Trout itself tries to solve the problem by ripping up 
wires and re-routing. It will only do this with a small number
of unconnected nets and if the \smc{cpu}-consumption is not too high.
\item
Make a new placement by running \tool{madonna} again in the
same or a slightly larger box. This will create a new one
which - with a little bit of luck - will be routable. Any time you press
\button{Madonna}, you'll get a different placement.
\item
Delete a part of the congested area and run \tool{trout}
again.  You can specifically delete some pieces of wire
which you suspect to be the cause of the problem.  The
router will try to glue the nets back together again, in a
different order.  You can use the \button{route box only}-button
to focus the router.  Have a look at which spot the
unconnects are located.
\item
Route the cell in a few overlapping boxes using the
\button{route box only}-button.
\item
Modify the layout manually. At any case you can call the
router again to finish the result automatically.
\end{enumerate}

\section{The run-time of \protect\tool{trout}}
The router will automatically keep you informed about its progress. If it
takes a longer time, \tool{trout} will also print a rough estimate of the time
it needs to finish. The cpu-time consumption of \tool{trout} depends on the
following factors:
\begin{itemize}
\item
The dominating factor is the {\bf size} of your placement (in square
grid points).  The run time increases quadratically with the size: if a layout
is twice as large, it takes four times as much time to route it.
\item
The run time increases approximately linear with the number of nets.
\item
The router is at is best for large complicated layouts with many wires. It is
quite slow, however, if there is a lot of empty space in your placement. This
is because in the latter case there are many different ways of making a wire.
\tool{Trout}'s algorithm evaluates many routes for each wire.
\item
For small placements, the overhead time for database IO is 
determining the time you have to wait.
\item
If \tool{trout} fails to route a wire, it may spend some time trying to
complete the routing in a different way. It might also spend some time trying
to solve problems by rip-up and re-routing.
\end{itemize}
The worst-case \smc{cpu}-consumption should be in the order of an hour for 200 nets
spanning the entire 200,000 transistor sea-of-gates chip. The typical cases are
much better: generally it is much less than a minute.

\section{Additional options of \protect\tool{trout}}
\label{troutopt}
Clicking \button{Option menu} opens a menu with a few more optional features
of the router. They are only useful for the final assembly of your layout, and
they should (in general) not be used to make leaf cells.
\begin{description}
\item[\button{make capacitors}]
\index{trout!decoupling capacitors|bold}
This is the most powerful option of all. It causes trout to convert all unused
transistors into capacitors between the power lines. In this way the noise
on the power lines decreases. This option creates many contacts and
therefore makes your layout very big. Generally you should only use it at the 
highest level of hierarchy.
\item[\button{dangling transistors}]
Connect all unused (dangling) transistors to the appropriate power wire. In
this way the state of all transistors is known and the capacity of the power
nets increases.
\item[\button{substrate contacts}]
\index{design rules!substrate contacts}
Create substrate contacts under the horizontal power rails.  The substrate
contacts are required by the design rules of the process.
\item[\button{make fat power lines}]
Make the horizontal power rails as big as possible. In this way the resistance
of the power wires decreases and the capacitance increases.
\tool{Trout} will also attempt to make as many as possible vertical
connections between the horizontal power rails.  The latter
can be turned off by clicking \button{no power}.  In a
large hierarchical design we advise you to click
\button{flood mesh}] as well.
\item[\button{use borders}]
Normally the router doesn't use the upper- and lower-most row, which is on top
of a power rail. In this way it prevents short-circuits with adjacent cells.
In some cases you might wish to switch off this feature by pressing this
button.
\item[\button{no routing}]
No routing of the signal nets. Use this option to add the
special features to the layout (fat power, substrate
contacts, etc.).  This option turns off the automatic
verification of the connectivity.
\item[\button{flood mesh}]
In a hierarchical design some fat wires may contain some
holes. Especially after using \button{make fat power lines} this could
be the case.  Pressing \button{flood mesh} causes the
router to fill all holes in your design. Please note that
pressing \button{Fish} again will undo the result of this
option, because \tool{fish} does not have a hierarchical
view of your design.
\item[\button{overlap wires}] 
On the Sea-of-Gates chip a metal2 wire (\mask{ins}) has a
higher probability to crack if it is running in parallel
with a metal1 (\mask{in}) wire. These 'co-incident edges',
however, are hard to avoid. To reduce the strain the button
\button{overlap wires} will makes the \mask{in} wires wider
at the spots where they overlap with \mask{ins} wires. In
this way the edges are less co-incident.  Please note that
pressing \button{Fish} again will undo the result of this
option.
\item[\button{options}]
Set any other (unofficial) option which should be
propagated to \tool{trout}.
\end{description}
\warning{
The options \button{make capacitors},
\button{make fat power lines},
\button{substrate contacts} and\\
\button{dangling transistors} will reduce the transparency of the
cell considerably.
This makes it almost impossible to route
new wires through the cell.
These options should therefore
only be used on the highest level of hierarchy,
or for critical cells.
A secondary effect of these options is
that the router is likely to generate a huge amount of boxes,
which makes \tool{seadali} slower and your database very big.}
\warning{ 
Pressing \button{Fish} or \button{Check nets} will undo the
results of the options \button{flood mesh} and
\button{overlap wires}. Therefore you should not use
\tool{fish} anymore after you've used these options.  }

\index{trout@\tool{trout}|)}

\cleardoublepage
