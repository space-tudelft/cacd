% conclusion.tex

%% Acceptatie van P.v.E. aantonen.
\chapter{Conclusion}\label{chap:conclusion}
In this final chapter we will compare the resulting application with the
requirements stated in Chapter \ref{chap:demands}.

\subsection*{Platform independence}
%% Prove platfrom independence
By using Qt, the STL (standard template library) and the \verb=gcc= compiler we
have made sure that the tools necessary to compile the application are
available on all target platforms. Both Qt and \verb=gcc= are freely available
for all required platforms. All recent versions of \verb=gcc= include a good
implementation of the STL.

These are not sufficient conditions to \emph{guarantee} platform independence.
However, the areas covered by these tools are usually the areas where platform
independcies arise. The application has been successfully tested on both Linux
and Solaris. Some tests were also performed on HP-UX and no problems were
encountered.

The \verb=makefile= generator tool \verb=tmake= also takes away the need to
manually configure the \verb=makefiles= for each target platform.

\subsection*{Integration with SPACE}
%% Prove integration
Integration with SPACE is available. The application can read the file
describing the processes (\verb=processlist=) in the SPACE process tree and the
user can integrate the generated technology files into the SPACE process tree
if desired. The \verb=processlist= file describing the available processes is
automatically updated to reflect the changes that were made.

Unfortunately, the application cannot read the technology files in the SPACE
process tree, it can only read its own file format.

\subsection*{Technology file generation}
%% Prove file generation
The technology files specified in the requirements can all be generated. These
are:
\begin{itemize}
\item \emph{maskdata}, which defines the layers present in the process and the
colors used to represent them in the programs and their output.
\item \emph{space.xxx.s}, the element definition files used by SPACE.
\item \emph{space.xxx.p}, the parameter files used by space.
\item \emph{bmlist.gds}, which provides a mapping between the GDS layer format and
the format used by SPACE.
\item \emph{xspicerc}, a control file that specifies which models are used for
the devices.
\end{itemize}
It should be noted that the space parameter file \verb=space.xxx.p= cannot be
completely generated yet. Not all of the parameters have been included in the
first version of the configuration file.

\subsection*{The configuration file language}
The configuration file language provides the basic functionality needed. The
language can be made more effective if additional language constructions are
implemented. The language constructions for ``chopping'' and macro definitions
can appear mangled to new users. A different syntax could be considered.

\subsection*{Flexibility}
%% Prove flexibility
Another requirement is the possibility to add new files to the list above or to
change the format of these files in a flexible way. This flexibility has been
achieved by making use of a configuration file. The configuration file contains
the specification of the user interface and the generators that use this user
interface to generate the technology files. These files can be changed without
the need to recompile the application.

Flexibility with regard to the source code has been achieved by using a good
source code documentation application and by using some well-known design
patterns that make understanding the code easier.

\bigskip \noindent
The final conclusion is that the developed application meets the requirements
specified in Chapter \ref{chap:demands}.
