% environment.tex

\chapter{Development environment} \label{chap:environment}
Developing an application is usually done with many tools. Some tools are
required for the development (like the compiler). Other tools increase the
programmers productivity.

\bigskip \noindent
The main goal of this chapter is to present the environment in which the
application was developed. This means the used tools and their interaction will
briefly be discussed.

\bigskip \noindent
First of all, the compiler and the platform will be described. This is followed
by a section on \verb=Makefile= generation. The \verb=Makefile= can be a major
headache to developers, especially if multiple platforms are involved. The
source code has been documented using a tool called Doxygen. A short
description of doxygen is given in Section \ref{chap:env:doxygen}.
%%%%%%%%%%%%%%%%%%%%%%%%%%%%%%%%%%%%%%%%%%%%%%%%%%%%%%%%%%%%%%%%%%%%%%%%%%%%%%
\section{Platform and compiler}
The platforms used most actively during the development of the application are
Solaris and Linux. A few compilations have been done on HP-UX as well and these
presented no problems.

The compiler used was \verb=gcc 2.95.2=. The \verb=gcc= compiler is a free
compiler that is available on practically every platform.
%%%%%%%%%%%%%%%%%%%%%%%%%%%%%%%%%%%%%%%%%%%%%%%%%%%%%%%%%%%%%%%%%%%%%%%%%%%%%%
\section{Makefile generation}
Makefiles are a necessary evil when developing applications on Unix or Linux
platforms. This is because there is no good integrated development environment
(like Visual Studio for Windows) available that works on all the platforms that
need to be supported. This is mainly due to the peculiarities of each platform.

A solution to this problem is \verb=tmake=. \verb=tmake= is a small utility
written by the creators of Qt, Trolltech \cite{Qt}. This utility uses templates
and a small input file containing some options and the sources to generate a
\verb=Makefile=. The templates hide the peculiarities of each platform, thus
establishing a platform independent interface.

Another possibility is to use \verb=autoconf=. The overhead involved with
\verb=autoconf= is larger, though.

As was mentioned above, the \verb=tmake= tool uses templates for the generation
of makefiles. There are templates for recursive makes and projects. It is also
possible to use custom templates. For the development of the application a
custom template was created that supports building library and test
applications in a single run as well as support for lex/flex yacc/bison parser
compilation.

%%%%%%%%%%%%%%%%%%%%%%%%%%%%%%%%%%%%%%%%%%%%%%%%%%%%%%%%%%%%%%%%%%%%%%%%%%%%%%
\section{Doxygen} \label{chap:env:doxygen}
Source code documentation can be generated with \verb=doxygen= \cite{doxygen}.
Another candidate for this position was \verb=kdoc=. However, \verb=kdoc=
forces the documentation completely into the header files, making the header
files unreadable. \verb=Doxygen= allows documentation virtually everywhere,
resulting in clean well-readable source files.

\subsection{Possible output formats}
As an added bonus, \verb=doxygen= can generate the documentation in many
formats:
\begin{itemize}
\item HTML format
\item \LaTeX
\item Man page format
\item Rich Text Format (RTF) which can be read by Microsoft Word
\item XML
\end{itemize}
It can even generate a cgi script that can be used as a search engine to search
through the HTML documentation.

\subsection{Doxygen options}
The output of \verb=doxygen=  can be influenced with the many features present.
It is possible to extract everything, even if the code is undocumented (this
shows you the structure of the source code). It is also possible to extract
only the documented code. Private and static members can be ignored if desired.

To keep the printed source code reference a reasonable size and to save some
trees in general, the documentation was created for documented items only,
leaving out the private methods and members.

\subsection{Collaboration diagrams}
If the \verb=graphviz= package is installed, \verb=doxygen= can make use of
\verb=dot=, which is a part of that package. With \verb=dot= it is possible to
create collaboration diagrams. Some of the diagrams created by \verb=dot= are
used in this report, for example Figure \ref{fig:design:guibuilder_partial}.

%%%%%%%%%%%%%%%%%%%%%%%%%%%%%%%%%%%%%%%%%%%%%%%%%%%%%%%%%%%%%%%%%%%%%%%%%%%%%%
\section{Qt Designer}
Qt Designer became available near the end of the project. It is a user
interface editor like X-Designer. It is possible to click together a dialog box
and generate the source code that builds the dialog box. The technology file
generation dialog box was created with Qt Designer as an experiment.

To use the generated source a new class has to be derived from the generated
class. This derived class implements the desired functionality. In the case of
the technology file generation dialog box this is the \verb=CGenerateDlg=
class.

\bigskip \noindent
The ease with which a user interface can be created with Qt Designer is
remarkable. Qt supports automatic widget layout and with Qt Designer this
becomes even easier then before.

Functionality can easily be added to the dialog boxes designed with Qt
Designer. By using inheritance, the only code that needs to be written is the
code that must provide the needed functionality. This makes Qt Designer a true
rapid application development tool.

The only downside of Qt Designer is that is (currently) only possible to create
widgets and dialog boxes. It is not possible to create an application's
framework.
