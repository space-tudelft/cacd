%% Introduction
%%
% Onderwerp : Ontwikkeling van een programma waarmee de technologie bestanden van SPACE
%             op een snelle manier gemaakt kunnen worden.
%
%
% Hoofdvraag: Wat zijn de belangrijkste problemen bij de ontwikkeling en hoe zijn deze opgelost?
%
% Achtergrondvragen:
%   - Wat is de omgeving waarin het programma is ontwikkeld?
%   - Welke ontwerpmethodes zijn toegepast?
%   - Waarom helpt dit programma bij de ontwikkeling van technologie bestanden? (Waarom optimaal)
%  
% Kernvragen:
%   - technische haalbaarheid
%   - flexibiliteit
%   - betrouwbaarheid
%%%%%%%%%%%%%%%%%%%%%%%%%%%%%%%%%%%%%%%%%%%%%%%%%%%%%%%%%%%%%%%%%%%%%%%%%%%%%%

\chapter{Introduction}
\label{chap:introduction}

A layout-to-circuit extractor like SPACE\footnote{\emph{SPACE} is the 
layout--to--circuit extractor developed at the Circuits and Systems group of 
the Electrical Engineering faculty. SPACE has been quite successful so far. It 
is used by several companies and universities, including Agilent (Hewlett 
Packard) and Level One (Intel).}  must (of course) be able to handle all kinds 
of processes. In SPACE, each of these processes is described by a set of 
\emph{technology files}. The contents of these files vary from simple layer 
specification to complex capacitance tables.

Until now, the technology files were maintained by hand. This means that for 
each process all the technology files had to be entered manually, a 
time-consuming and error-prone task. Also, the user must be familiar with the 
file format of each of the technology files, which makes process entry and 
maintenance a specialized task.

\emph{SPOCK} is to alleviate the problems described above by presenting the 
user with a graphical user interface in which the required data can be entered.

\bigskip \noindent
The main goal of this thesis is to identify the problems encountered during 
the design and implementation of SPOCK and to describe the solutions that were 
used to solve these problems.

\bigskip \noindent
Chapter \ref{chap:demands} presents the demanded functionality of the 
application as well as some imposed constraints. Chapter \ref{chap:design} 
discusses and explains the final design. The user interface plays an important 
role in this application. Some standard and custom user interface elements are 
therefore presented in Chapter \ref{chap:uidesign}. Chapter 
\ref{chap:language} describes the configuration file language. As will become 
clear later, the configuration file describes the technology files supported 
by the application.
%Chapter \ref{chap:environment}) describes the tools used during development. Some 
%comments on the testing and debugging process will be given in Chapter 
%\ref{chap:testing}. 
The final chapter will contain some concluding remarks.
