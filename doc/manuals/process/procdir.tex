\chapter{The Process Directory}

The technology files for a process are stored in a separate (process) directory.
This way, all design projects that use the process information can refer to
the same directory.

\section{Default Location of a Process Directory} 

The default location of a process directory is under
\texttt{\$ICDPATH/share/lib/process},
where \texttt{\$ICDPATH} is the installation directory of the software.
The standard software distribution contains several process directories
that are all stored under this directory
(e.g. c3tu, dimes03, fishbone, scmos\_n, tsmc025).
The file \texttt{\$ICDPATH/share/lib/process/processlist} contains a list of these
process directories and assigns an id (number) to each process.
\begin{alltt}
% cat /usr/cacd/share/lib/process/processlist
# process list
# proc_id proc_name
1  nmos      # TUD demo nmos process
3  scmos_n   # scalable cmos process
6  gatearray # sea-of-gates gatearray process
18 c3tu      # 1.6 micron cmos process
23 dimes01   # first (bipolar) process of DIMES
40 octagon   # sea-of-gates octagon process (modified c3tu)
41 fishbone  # sea-of-gates fishbone process (1.6 micron)
44 scmos-orb-2 # Orbit 2.0 micron cmos process
45 ami-c5n   # AMI 0.5 micron cmos process
46 dimes03   # current bipolar process of DIMES, TU Delft
60 tsmc025   # MOSIS TSMC CMOS025 (0.25u) process
\end{alltt}

When creating a design project with 
\texttt{mkpr}
you can select the process that you want to use by its number.
When you want to add a process to the default location for process
directories, you have to place the directory as a subdirectory 
under \texttt{\$ICDPATH/share/lib/process} and you have to add a new line
with the (unique) process id and process name to the file
\texttt{\$ICDPATH/share/lib/process/processlist}.

\section{Other Locations of a Process Directory} 

A process directory may also reside at an arbitrary place on the 
file system.
In that case, when you want to create a design project that refers
to the process information, you have to use the option -p with
\texttt{mkpr}
to specify the path to the directory.
