%%%%%%%%%%%%%%%%%
% Main document.
%%%%%%%%%%%%%%%%%

\documentclass[11pt]{book}
\usepackage{cacdmanual}
% \usepackage{alltt}
% \usepackage{fancyhdr}
% \usepackage{graphics}
% \usepackage{calc}

\usepackage{a4wide,times,latexsym,calc,ifthen,float}
\usepackage{epsfig}
\usepackage{fancyvrb}
\usepackage[dvips,colorlinks]{hyperref}

\newcommand{\io}[1]{{\ttfamily #1}}
\newcommand{\CACDTOP}[1]{\io{\$ICDPATH/share/#1}}
\newcommand{\manualpage}[1]{\io{icdman\ #1}}

%%% Various Description List Types
%%% Adapted from pp60-66 The Latex Companion
%%%
\newcommand{\entrylabel}[1]{\mbox{\texttt{#1:}}\hfil}

\newlength{\Mylen}

\newenvironment{fentry}
    {\begin{list}{}%
        {\renewcommand{\makelabel}{\entrylabel}%
            \setlength{\labelwidth}{4cm}%
	    \setlength{\itemsep}{0.0cm}%
            \setlength{\leftmargin}{\labelwidth+\labelsep}%
        }}%
    {\end{list}}

\newenvironment{oentry}
    {\begin{list}{}%
        {\renewcommand{\makelabel}{\entrylabel}%
            \setlength{\labelwidth}{2.8em}%
	    \setlength{\itemsep}{0.0cm}%
            \setlength{\leftmargin}{\labelwidth+\labelsep+1em}%
        }}%
    {\end{list}}


\newcommand{\Oentrylabel}[1]{%
    \settowidth{\Mylen}{\texttt{#1:}}%
    \ifthenelse{\lengthtest{\Mylen > \labelwidth}}%
        {\parbox[b]{\labelwidth}%        term > labelwidth
            {\makebox[0pt][l]{\texttt{\small  #1:}}\\}}%
        {\texttt{\small#1:}}%                  term < labelwidth
    \hfil\relax}

\newenvironment{optionlist}
    {\renewcommand{\entrylabel}{\Oentrylabel}%
     \begin{oentry}}
    {\end{oentry}}

\newenvironment{filelist}
    {\renewcommand{\entrylabel}{\Oentrylabel}%
     \begin{fentry}}
    {\end{fentry}}


\def\floatpagefraction{0.9}
\def\textfraction{0.1}

\begin{document}

%%%%%%%%%%%%%%%%%%%%%
% Define title page.
%%%%%%%%%%%%%%%%%%%%%

\cacdreport{ET-ENS 2006.01}
\cacdauthors{Nick van der Meijs, Simon de Graaf}
\cacddate{June 29, 2006}
\cacdyears{2006}
\cacdtitle{SPACE EXAMPLES}
\cacdheadertitle{Space Examples}
\cacdmaketitle

\tableofcontents

%%%%%%%%%%%%%%%%%%%%
% Define Remainder.
%%%%%%%%%%%%%%%%%%%%

\chapter{Introduction}

This document explains something about the space demo examples.
Trying these demo's
can be a good starting point to learn more about the Space System.

Before you can use the demo's, you must know the space software installation path.
When the space software is installed in your home directory, then the installation
path (or \io{ICDPATH}) is \io{\$HOME/cacd}.
But you can also rename 'cacd' into something else, for example 'cacd\_new'.
The only thing you need to do, is to set your \io{PATH} environment variable to the \io{ICDPATH}.
See also the file \io{\$ICDPATH/Installation.txt} for more information.
Give for example the following commands to set your \io{PATH}:
\begin{Verbatim}
    % setenv ICDPATH $HOME/cacd
    % setenv PATH $ICDPATH/bin:$PATH
\end{Verbatim}
Go now to the demo directory, for example type:
\begin{Verbatim}
    % cd $ICDPATH/share/demo
    % ls
    attenua  invert      ny9t    README    sub3term   switchbox
    crand    multiplier  poly5   sram      suboscil
\end{Verbatim}
Each demo contains a \io{README} file, which explains what you must do.
When you don't have a private software installation or don't have write permission,
you must copy the demo files to a private directory to do the demo.
For example:
\begin{Verbatim}
    % mkdir mydemo ; cd mydemo
    % cp $ICDPATH/share/demo/attenua/* .
    % cat README
\end{Verbatim}
When you don't want to type the space commands yourself,
you can use a shell script to do it for you.
In that case, type:
\begin{Verbatim}
    % ./script.sh
 or
    % sh script.sh
\end{Verbatim}
Read the following chapters, to know more about the demo examples (if included).
We wish you success.
We hope that you appreciate our software and want to use it.
If you have any questions or comments, they are welcome.
\begin{Verbatim}
    The SPACE team.
\end{Verbatim}

\chapter{crand Example of Extraction/Switch-Level Simulation}
\section{Introduction}
\label{PEintro}
In this example, we will be studying a random counter circuit.
We will see how Space is used for circuit extraction.
And how you can do a switch-level simulation of the circuit.
\\[1 ex]
The layout looks as follows, using the layout editor \io{dali} (see \manualpage{dali}):

\begin{figure}[h]
\centerline{\epsfig{figure=crand/crand.eps, width=15cm}}
\end{figure}

\section{Files}
This tutorial is located in the directory \CACDTOP{demo/crand}.
Initially, it contains the following files:
\begin{filelist}
\item[README] A file containing information about the demo.
\item[crand.cmd] Command file for circuit simulation.
\item[crand.gds] GDS2 file of the layout of the crand design.
\item[script.sh] Batch file for running all commands of the demo in sequence.
\end{filelist}

\section{Running the Extractor}
First, use the following command to change the current working directory '.' into a project directory:
\small
\begin{Verbatim}
% mkpr -p scmos_n -l 0.2 .
\end{Verbatim}
\normalsize
The command specifies the \io{scmos\_n} process from the technology library
and a lambda (design unit) of $0.2 \mu m$.
We use the mask names as defined in the \io{maskdata} file of the library.
And we are using the default technology file \io{space.def.s}
and parameter file \io{space.def.p} of the library.
\small
\begin{Verbatim}
% cgi crand.gds
\end{Verbatim}
\normalsize
Now, we can extract a circuit description for the layout of the \io{crand} cell, as follows:
\small
\begin{Verbatim}
% space -vFc crand
\end{Verbatim}
\normalsize
\small \begin{Verbatim}[frame=single]
Version 5.3.1, compiled on Fri Feb 03 12:45:53 GMT 2006
See http://www.space.tudelft.nl
parameter file: $ICDPATH/share/lib/process/scmos_n/space.def.p
technology file: $ICDPATH/share/lib/process/scmos_n/space.def.t
preprocessing crand (phase 1 - flattening layout)
preprocessing crand (phase 2 - removing overlap)
extracting crand

extraction statistics for layout crand:
	capacitances        : 221
	resistances         : 0
	nodes               : 222
	mos transistors     : 419
	bipolar vertical    : 0
	bipolar lateral     : 0
	substrate nodes     : 0

overall resource utilization:
	memory allocation  : 0.287 Mbyte
	user time          :         0.0
	system time        :         0.0
	real time          :         1.5   5%

space: --- Finished ---
\end{Verbatim}
\normalsize
You can show the resulting circuit with one of the circuit listing tools.
For example, to list the circuit in a SLS description, use \io{xsls} (see \io{icdman}).
\small
\begin{Verbatim}
% xsls crand
\end{Verbatim}
\normalsize
The output is default going to "stdout", a part is shown below:
\small \begin{Verbatim}[frame=single]
   ...
network crand (terminal out_7, out_6, out_5, out_4, out_3, out_2, out_1, out_0,
               inpar_7, inpar_6, inpar_5, inpar_4, inpar_3, inpar_2, inpar_1,
               inpar_0, serial, vss_lb, vss_lo, sc_l, nsc_l, vdd_lb, vdd_lo,
               nphi1_l, phi1_l, nphi2_l, phi2_l, phi1_r, phi2_r, nphi2_r,
               nphi1_r, vss_ro, vss_rb, vdd_rb, nsc_r, sc_r, vdd_ro)
{
    net {vdd_lo, vdd_ro};
    net {phi1_l, phi1_r};
    net {phi2_l, phi2_r};
    net {nphi2_l, nphi2_r};
    net {nphi1_l, nphi1_r};
    net {sc_l, sc_r};
    net {vdd_lb, vdd_rb};
    net {nsc_l, nsc_r};
    net {SUBSTR, vss_lb};
    net {SUBSTR, vss_rb};
    net {SUBSTR, vss_ro};
    net {SUBSTR, vss_lo};
    cap 2.8f (1, GND);
    nenh w=4u l=1.2u (14, 1, 14);
    cap 11.44f (2, GND);
    penh w=6.8u l=1.2u (14, 2, 14);
    cap 11.44f (3, GND);
    penh w=6.8u l=1.2u (vdd_lb, 3, 12);
    cap 2.8f (4, GND);
    nenh w=4u l=1.2u (SUBSTR, 4, SUBSTR);
    nenh w=4u l=1.2u (phi1_l, 13, 14);
    penh w=6.8u l=1.2u (nphi1_l, 13, 14);
    penh w=6.8u l=1.2u (serial, 12, vdd_lb);
    nenh w=4u l=1.2u (serial, 5, SUBSTR);
    nenh w=4u l=1.2u (phi2_l, 10, 13);
    penh w=6.8u l=1.2u (nphi2_l, 10, 13);
    penh w=6.8u l=1.2u (nsc_l, 12, vdd_lb);
    cap 3.2f (5, GND);
    nenh w=4u l=1.2u (nsc_l, 5, 14);
    cap 72.16f (6, GND);
    penh w=6.8u l=1.2u (vdd_lo, 6, 8);
    cap 16f (7, GND);
    nenh w=4u l=1.2u (SUBSTR, 7, 9);
    nenh w=4u l=1.2u (10, 10, 10);
    penh w=6.8u l=1.2u (10, 10, 10);
    penh w=6.8u l=1.2u (inpar_0, 12, 14);
    nenh w=4u l=1.2u (inpar_0, 11, 14);
    nenh w=7.2u l=1.2u (inpar_0, SUBSTR, SUBSTR);
    ...
    ...
}
\end{Verbatim}
\normalsize

\section{Running the Switch-Level Simulation}
For this simulation you are using the switch-level simulator \io{sls}.
See the "SLS: Switch-Level Simulator User's Manual" and for the manual page \manualpage{sls}.
This simulator is started from the simulation GUI \io{simeye} and
the results are shown in the output window (see \manualpage{simeye}).
\\[1 ex]
First, start the simulation GUI \io{simeye}.
\small
\begin{Verbatim}
% simeye
\end{Verbatim}
\normalsize
Second, prepare the simulation:
\\[1 ex]
Click on the "Simulate" menu and choice the "Prepare" item.
Select in the "Circuit:" field cell name "crand" and
in the "Stimuli:" field file name "crand.cmd" (click on it).
To inspect or edit the input signals, click on the "Edit" button.
\\[1 ex]
Third, start the switch-level simulation:
\\[1 ex]
Go back to the "Simulate" menu and choice the "Prepare" dialog item again:

\begin{figure}[h]
\centerline{\epsfig{figure=crand/dialog.eps, width=9cm}}
\end{figure}

\noindent
In the dialog window, choice simulation "Type: sls-timing" and for "Read: Analog".
Now, start the switch-level timing simulation by clicking on the "Run" button and wait for simulation results.
Below, you see the output waveforms.

\begin{figure}[h]
\centerline{\epsfig{figure=crand/simeye.eps, width=14cm}}
\end{figure}

Note, to exit \io{simeye},
go to the "File" menu and click on "Exit" and "Yes".

\chapter{crand Example of Extraction/Switch-Level Simulation}
\section{Introduction}
\label{PEintro}
In this example, we will be studying a random counter circuit.
We will see how Space is used for circuit extraction.
And how you can do a switch-level simulation of the circuit.
\\[1 ex]
The layout looks as follows, using the layout editor \io{dali} (see \manualpage{dali}):

\begin{figure}[h]
\centerline{\epsfig{figure=crand/crand.eps, width=15cm}}
\end{figure}

\section{Files}
This tutorial is located in the directory \CACDTOP{demo/crand}.
Initially, it contains the following files:
\begin{filelist}
\item[README] A file containing information about the demo.
\item[crand.cmd] Command file for circuit simulation.
\item[crand.gds] GDS2 file of the layout of the crand design.
\item[script.sh] Batch file for running all commands of the demo in sequence.
\end{filelist}

\section{Running the Extractor}
First, use the following command to change the current working directory '.' into a project directory:
\small
\begin{Verbatim}
% mkpr -p scmos_n -l 0.2 .
\end{Verbatim}
\normalsize
The command specifies the \io{scmos\_n} process from the technology library
and a lambda (design unit) of $0.2 \mu m$.
We use the mask names as defined in the \io{maskdata} file of the library.
And we are using the default technology file \io{space.def.s}
and parameter file \io{space.def.p} of the library.
\small
\begin{Verbatim}
% cgi crand.gds
\end{Verbatim}
\normalsize
Now, we can extract a circuit description for the layout of the \io{crand} cell, as follows:
\small
\begin{Verbatim}
% space -vFc crand
\end{Verbatim}
\normalsize
\small \begin{Verbatim}[frame=single]
Version 5.3.1, compiled on Fri Feb 03 12:45:53 GMT 2006
See http://www.space.tudelft.nl
parameter file: $ICDPATH/share/lib/process/scmos_n/space.def.p
technology file: $ICDPATH/share/lib/process/scmos_n/space.def.t
preprocessing crand (phase 1 - flattening layout)
preprocessing crand (phase 2 - removing overlap)
extracting crand

extraction statistics for layout crand:
	capacitances        : 221
	resistances         : 0
	nodes               : 222
	mos transistors     : 419
	bipolar vertical    : 0
	bipolar lateral     : 0
	substrate nodes     : 0

overall resource utilization:
	memory allocation  : 0.287 Mbyte
	user time          :         0.0
	system time        :         0.0
	real time          :         1.5   5%

space: --- Finished ---
\end{Verbatim}
\normalsize
You can show the resulting circuit with one of the circuit listing tools.
For example, to list the circuit in a SLS description, use \io{xsls} (see \io{icdman}).
\small
\begin{Verbatim}
% xsls crand
\end{Verbatim}
\normalsize
The output is default going to "stdout", a part is shown below:
\small \begin{Verbatim}[frame=single]
   ...
network crand (terminal out_7, out_6, out_5, out_4, out_3, out_2, out_1, out_0,
               inpar_7, inpar_6, inpar_5, inpar_4, inpar_3, inpar_2, inpar_1,
               inpar_0, serial, vss_lb, vss_lo, sc_l, nsc_l, vdd_lb, vdd_lo,
               nphi1_l, phi1_l, nphi2_l, phi2_l, phi1_r, phi2_r, nphi2_r,
               nphi1_r, vss_ro, vss_rb, vdd_rb, nsc_r, sc_r, vdd_ro)
{
    net {vdd_lo, vdd_ro};
    net {phi1_l, phi1_r};
    net {phi2_l, phi2_r};
    net {nphi2_l, nphi2_r};
    net {nphi1_l, nphi1_r};
    net {sc_l, sc_r};
    net {vdd_lb, vdd_rb};
    net {nsc_l, nsc_r};
    net {SUBSTR, vss_lb};
    net {SUBSTR, vss_rb};
    net {SUBSTR, vss_ro};
    net {SUBSTR, vss_lo};
    cap 2.8f (1, GND);
    nenh w=4u l=1.2u (14, 1, 14);
    cap 11.44f (2, GND);
    penh w=6.8u l=1.2u (14, 2, 14);
    cap 11.44f (3, GND);
    penh w=6.8u l=1.2u (vdd_lb, 3, 12);
    cap 2.8f (4, GND);
    nenh w=4u l=1.2u (SUBSTR, 4, SUBSTR);
    nenh w=4u l=1.2u (phi1_l, 13, 14);
    penh w=6.8u l=1.2u (nphi1_l, 13, 14);
    penh w=6.8u l=1.2u (serial, 12, vdd_lb);
    nenh w=4u l=1.2u (serial, 5, SUBSTR);
    nenh w=4u l=1.2u (phi2_l, 10, 13);
    penh w=6.8u l=1.2u (nphi2_l, 10, 13);
    penh w=6.8u l=1.2u (nsc_l, 12, vdd_lb);
    cap 3.2f (5, GND);
    nenh w=4u l=1.2u (nsc_l, 5, 14);
    cap 72.16f (6, GND);
    penh w=6.8u l=1.2u (vdd_lo, 6, 8);
    cap 16f (7, GND);
    nenh w=4u l=1.2u (SUBSTR, 7, 9);
    nenh w=4u l=1.2u (10, 10, 10);
    penh w=6.8u l=1.2u (10, 10, 10);
    penh w=6.8u l=1.2u (inpar_0, 12, 14);
    nenh w=4u l=1.2u (inpar_0, 11, 14);
    nenh w=7.2u l=1.2u (inpar_0, SUBSTR, SUBSTR);
    ...
    ...
}
\end{Verbatim}
\normalsize

\section{Running the Switch-Level Simulation}
For this simulation you are using the switch-level simulator \io{sls}.
See the "SLS: Switch-Level Simulator User's Manual" and for the manual page \manualpage{sls}.
This simulator is started from the simulation GUI \io{simeye} and
the results are shown in the output window (see \manualpage{simeye}).
\\[1 ex]
First, start the simulation GUI \io{simeye}.
\small
\begin{Verbatim}
% simeye
\end{Verbatim}
\normalsize
Second, prepare the simulation:
\\[1 ex]
Click on the "Simulate" menu and choice the "Prepare" item.
Select in the "Circuit:" field cell name "crand" and
in the "Stimuli:" field file name "crand.cmd" (click on it).
To inspect or edit the input signals, click on the "Edit" button.
\\[1 ex]
Third, start the switch-level simulation:
\\[1 ex]
Go back to the "Simulate" menu and choice the "Prepare" dialog item again:

\begin{figure}[h]
\centerline{\epsfig{figure=crand/dialog.eps, width=9cm}}
\end{figure}

\noindent
In the dialog window, choice simulation "Type: sls-timing" and for "Read: Analog".
Now, start the switch-level timing simulation by clicking on the "Run" button and wait for simulation results.
Below, you see the output waveforms.

\begin{figure}[h]
\centerline{\epsfig{figure=crand/simeye.eps, width=14cm}}
\end{figure}

Note, to exit \io{simeye},
go to the "File" menu and click on "Exit" and "Yes".

\chapter{crand Example of Extraction/Switch-Level Simulation}
\section{Introduction}
\label{PEintro}
In this example, we will be studying a random counter circuit.
We will see how Space is used for circuit extraction.
And how you can do a switch-level simulation of the circuit.
\\[1 ex]
The layout looks as follows, using the layout editor \io{dali} (see \manualpage{dali}):

\begin{figure}[h]
\centerline{\epsfig{figure=crand/crand.eps, width=15cm}}
\end{figure}

\section{Files}
This tutorial is located in the directory \CACDTOP{demo/crand}.
Initially, it contains the following files:
\begin{filelist}
\item[README] A file containing information about the demo.
\item[crand.cmd] Command file for circuit simulation.
\item[crand.gds] GDS2 file of the layout of the crand design.
\item[script.sh] Batch file for running all commands of the demo in sequence.
\end{filelist}

\section{Running the Extractor}
First, use the following command to change the current working directory '.' into a project directory:
\small
\begin{Verbatim}
% mkpr -p scmos_n -l 0.2 .
\end{Verbatim}
\normalsize
The command specifies the \io{scmos\_n} process from the technology library
and a lambda (design unit) of $0.2 \mu m$.
We use the mask names as defined in the \io{maskdata} file of the library.
And we are using the default technology file \io{space.def.s}
and parameter file \io{space.def.p} of the library.
\small
\begin{Verbatim}
% cgi crand.gds
\end{Verbatim}
\normalsize
Now, we can extract a circuit description for the layout of the \io{crand} cell, as follows:
\small
\begin{Verbatim}
% space -vFc crand
\end{Verbatim}
\normalsize
\small \begin{Verbatim}[frame=single]
Version 5.3.1, compiled on Fri Feb 03 12:45:53 GMT 2006
See http://www.space.tudelft.nl
parameter file: $ICDPATH/share/lib/process/scmos_n/space.def.p
technology file: $ICDPATH/share/lib/process/scmos_n/space.def.t
preprocessing crand (phase 1 - flattening layout)
preprocessing crand (phase 2 - removing overlap)
extracting crand

extraction statistics for layout crand:
	capacitances        : 221
	resistances         : 0
	nodes               : 222
	mos transistors     : 419
	bipolar vertical    : 0
	bipolar lateral     : 0
	substrate nodes     : 0

overall resource utilization:
	memory allocation  : 0.287 Mbyte
	user time          :         0.0
	system time        :         0.0
	real time          :         1.5   5%

space: --- Finished ---
\end{Verbatim}
\normalsize
You can show the resulting circuit with one of the circuit listing tools.
For example, to list the circuit in a SLS description, use \io{xsls} (see \io{icdman}).
\small
\begin{Verbatim}
% xsls crand
\end{Verbatim}
\normalsize
The output is default going to "stdout", a part is shown below:
\small \begin{Verbatim}[frame=single]
   ...
network crand (terminal out_7, out_6, out_5, out_4, out_3, out_2, out_1, out_0,
               inpar_7, inpar_6, inpar_5, inpar_4, inpar_3, inpar_2, inpar_1,
               inpar_0, serial, vss_lb, vss_lo, sc_l, nsc_l, vdd_lb, vdd_lo,
               nphi1_l, phi1_l, nphi2_l, phi2_l, phi1_r, phi2_r, nphi2_r,
               nphi1_r, vss_ro, vss_rb, vdd_rb, nsc_r, sc_r, vdd_ro)
{
    net {vdd_lo, vdd_ro};
    net {phi1_l, phi1_r};
    net {phi2_l, phi2_r};
    net {nphi2_l, nphi2_r};
    net {nphi1_l, nphi1_r};
    net {sc_l, sc_r};
    net {vdd_lb, vdd_rb};
    net {nsc_l, nsc_r};
    net {SUBSTR, vss_lb};
    net {SUBSTR, vss_rb};
    net {SUBSTR, vss_ro};
    net {SUBSTR, vss_lo};
    cap 2.8f (1, GND);
    nenh w=4u l=1.2u (14, 1, 14);
    cap 11.44f (2, GND);
    penh w=6.8u l=1.2u (14, 2, 14);
    cap 11.44f (3, GND);
    penh w=6.8u l=1.2u (vdd_lb, 3, 12);
    cap 2.8f (4, GND);
    nenh w=4u l=1.2u (SUBSTR, 4, SUBSTR);
    nenh w=4u l=1.2u (phi1_l, 13, 14);
    penh w=6.8u l=1.2u (nphi1_l, 13, 14);
    penh w=6.8u l=1.2u (serial, 12, vdd_lb);
    nenh w=4u l=1.2u (serial, 5, SUBSTR);
    nenh w=4u l=1.2u (phi2_l, 10, 13);
    penh w=6.8u l=1.2u (nphi2_l, 10, 13);
    penh w=6.8u l=1.2u (nsc_l, 12, vdd_lb);
    cap 3.2f (5, GND);
    nenh w=4u l=1.2u (nsc_l, 5, 14);
    cap 72.16f (6, GND);
    penh w=6.8u l=1.2u (vdd_lo, 6, 8);
    cap 16f (7, GND);
    nenh w=4u l=1.2u (SUBSTR, 7, 9);
    nenh w=4u l=1.2u (10, 10, 10);
    penh w=6.8u l=1.2u (10, 10, 10);
    penh w=6.8u l=1.2u (inpar_0, 12, 14);
    nenh w=4u l=1.2u (inpar_0, 11, 14);
    nenh w=7.2u l=1.2u (inpar_0, SUBSTR, SUBSTR);
    ...
    ...
}
\end{Verbatim}
\normalsize

\section{Running the Switch-Level Simulation}
For this simulation you are using the switch-level simulator \io{sls}.
See the "SLS: Switch-Level Simulator User's Manual" and for the manual page \manualpage{sls}.
This simulator is started from the simulation GUI \io{simeye} and
the results are shown in the output window (see \manualpage{simeye}).
\\[1 ex]
First, start the simulation GUI \io{simeye}.
\small
\begin{Verbatim}
% simeye
\end{Verbatim}
\normalsize
Second, prepare the simulation:
\\[1 ex]
Click on the "Simulate" menu and choice the "Prepare" item.
Select in the "Circuit:" field cell name "crand" and
in the "Stimuli:" field file name "crand.cmd" (click on it).
To inspect or edit the input signals, click on the "Edit" button.
\\[1 ex]
Third, start the switch-level simulation:
\\[1 ex]
Go back to the "Simulate" menu and choice the "Prepare" dialog item again:

\begin{figure}[h]
\centerline{\epsfig{figure=crand/dialog.eps, width=9cm}}
\end{figure}

\noindent
In the dialog window, choice simulation "Type: sls-timing" and for "Read: Analog".
Now, start the switch-level timing simulation by clicking on the "Run" button and wait for simulation results.
Below, you see the output waveforms.

\begin{figure}[h]
\centerline{\epsfig{figure=crand/simeye.eps, width=14cm}}
\end{figure}

Note, to exit \io{simeye},
go to the "File" menu and click on "Exit" and "Yes".

\chapter{crand Example of Extraction/Switch-Level Simulation}
\section{Introduction}
\label{PEintro}
In this example, we will be studying a random counter circuit.
We will see how Space is used for circuit extraction.
And how you can do a switch-level simulation of the circuit.
\\[1 ex]
The layout looks as follows, using the layout editor \io{dali} (see \manualpage{dali}):

\begin{figure}[h]
\centerline{\epsfig{figure=crand/crand.eps, width=15cm}}
\end{figure}

\section{Files}
This tutorial is located in the directory \CACDTOP{demo/crand}.
Initially, it contains the following files:
\begin{filelist}
\item[README] A file containing information about the demo.
\item[crand.cmd] Command file for circuit simulation.
\item[crand.gds] GDS2 file of the layout of the crand design.
\item[script.sh] Batch file for running all commands of the demo in sequence.
\end{filelist}

\section{Running the Extractor}
First, use the following command to change the current working directory '.' into a project directory:
\small
\begin{Verbatim}
% mkpr -p scmos_n -l 0.2 .
\end{Verbatim}
\normalsize
The command specifies the \io{scmos\_n} process from the technology library
and a lambda (design unit) of $0.2 \mu m$.
We use the mask names as defined in the \io{maskdata} file of the library.
And we are using the default technology file \io{space.def.s}
and parameter file \io{space.def.p} of the library.
\small
\begin{Verbatim}
% cgi crand.gds
\end{Verbatim}
\normalsize
Now, we can extract a circuit description for the layout of the \io{crand} cell, as follows:
\small
\begin{Verbatim}
% space -vFc crand
\end{Verbatim}
\normalsize
\small \begin{Verbatim}[frame=single]
Version 5.3.1, compiled on Fri Feb 03 12:45:53 GMT 2006
See http://www.space.tudelft.nl
parameter file: $ICDPATH/share/lib/process/scmos_n/space.def.p
technology file: $ICDPATH/share/lib/process/scmos_n/space.def.t
preprocessing crand (phase 1 - flattening layout)
preprocessing crand (phase 2 - removing overlap)
extracting crand

extraction statistics for layout crand:
	capacitances        : 221
	resistances         : 0
	nodes               : 222
	mos transistors     : 419
	bipolar vertical    : 0
	bipolar lateral     : 0
	substrate nodes     : 0

overall resource utilization:
	memory allocation  : 0.287 Mbyte
	user time          :         0.0
	system time        :         0.0
	real time          :         1.5   5%

space: --- Finished ---
\end{Verbatim}
\normalsize
You can show the resulting circuit with one of the circuit listing tools.
For example, to list the circuit in a SLS description, use \io{xsls} (see \io{icdman}).
\small
\begin{Verbatim}
% xsls crand
\end{Verbatim}
\normalsize
The output is default going to "stdout", a part is shown below:
\small \begin{Verbatim}[frame=single]
   ...
network crand (terminal out_7, out_6, out_5, out_4, out_3, out_2, out_1, out_0,
               inpar_7, inpar_6, inpar_5, inpar_4, inpar_3, inpar_2, inpar_1,
               inpar_0, serial, vss_lb, vss_lo, sc_l, nsc_l, vdd_lb, vdd_lo,
               nphi1_l, phi1_l, nphi2_l, phi2_l, phi1_r, phi2_r, nphi2_r,
               nphi1_r, vss_ro, vss_rb, vdd_rb, nsc_r, sc_r, vdd_ro)
{
    net {vdd_lo, vdd_ro};
    net {phi1_l, phi1_r};
    net {phi2_l, phi2_r};
    net {nphi2_l, nphi2_r};
    net {nphi1_l, nphi1_r};
    net {sc_l, sc_r};
    net {vdd_lb, vdd_rb};
    net {nsc_l, nsc_r};
    net {SUBSTR, vss_lb};
    net {SUBSTR, vss_rb};
    net {SUBSTR, vss_ro};
    net {SUBSTR, vss_lo};
    cap 2.8f (1, GND);
    nenh w=4u l=1.2u (14, 1, 14);
    cap 11.44f (2, GND);
    penh w=6.8u l=1.2u (14, 2, 14);
    cap 11.44f (3, GND);
    penh w=6.8u l=1.2u (vdd_lb, 3, 12);
    cap 2.8f (4, GND);
    nenh w=4u l=1.2u (SUBSTR, 4, SUBSTR);
    nenh w=4u l=1.2u (phi1_l, 13, 14);
    penh w=6.8u l=1.2u (nphi1_l, 13, 14);
    penh w=6.8u l=1.2u (serial, 12, vdd_lb);
    nenh w=4u l=1.2u (serial, 5, SUBSTR);
    nenh w=4u l=1.2u (phi2_l, 10, 13);
    penh w=6.8u l=1.2u (nphi2_l, 10, 13);
    penh w=6.8u l=1.2u (nsc_l, 12, vdd_lb);
    cap 3.2f (5, GND);
    nenh w=4u l=1.2u (nsc_l, 5, 14);
    cap 72.16f (6, GND);
    penh w=6.8u l=1.2u (vdd_lo, 6, 8);
    cap 16f (7, GND);
    nenh w=4u l=1.2u (SUBSTR, 7, 9);
    nenh w=4u l=1.2u (10, 10, 10);
    penh w=6.8u l=1.2u (10, 10, 10);
    penh w=6.8u l=1.2u (inpar_0, 12, 14);
    nenh w=4u l=1.2u (inpar_0, 11, 14);
    nenh w=7.2u l=1.2u (inpar_0, SUBSTR, SUBSTR);
    ...
    ...
}
\end{Verbatim}
\normalsize

\section{Running the Switch-Level Simulation}
For this simulation you are using the switch-level simulator \io{sls}.
See the "SLS: Switch-Level Simulator User's Manual" and for the manual page \manualpage{sls}.
This simulator is started from the simulation GUI \io{simeye} and
the results are shown in the output window (see \manualpage{simeye}).
\\[1 ex]
First, start the simulation GUI \io{simeye}.
\small
\begin{Verbatim}
% simeye
\end{Verbatim}
\normalsize
Second, prepare the simulation:
\\[1 ex]
Click on the "Simulate" menu and choice the "Prepare" item.
Select in the "Circuit:" field cell name "crand" and
in the "Stimuli:" field file name "crand.cmd" (click on it).
To inspect or edit the input signals, click on the "Edit" button.
\\[1 ex]
Third, start the switch-level simulation:
\\[1 ex]
Go back to the "Simulate" menu and choice the "Prepare" dialog item again:

\begin{figure}[h]
\centerline{\epsfig{figure=crand/dialog.eps, width=9cm}}
\end{figure}

\noindent
In the dialog window, choice simulation "Type: sls-timing" and for "Read: Analog".
Now, start the switch-level timing simulation by clicking on the "Run" button and wait for simulation results.
Below, you see the output waveforms.

\begin{figure}[h]
\centerline{\epsfig{figure=crand/simeye.eps, width=14cm}}
\end{figure}

Note, to exit \io{simeye},
go to the "File" menu and click on "Exit" and "Yes".

\chapter{crand Example of Extraction/Switch-Level Simulation}
\section{Introduction}
\label{PEintro}
In this example, we will be studying a random counter circuit.
We will see how Space is used for circuit extraction.
And how you can do a switch-level simulation of the circuit.
\\[1 ex]
The layout looks as follows, using the layout editor \io{dali} (see \manualpage{dali}):

\begin{figure}[h]
\centerline{\epsfig{figure=crand/crand.eps, width=15cm}}
\end{figure}

\section{Files}
This tutorial is located in the directory \CACDTOP{demo/crand}.
Initially, it contains the following files:
\begin{filelist}
\item[README] A file containing information about the demo.
\item[crand.cmd] Command file for circuit simulation.
\item[crand.gds] GDS2 file of the layout of the crand design.
\item[script.sh] Batch file for running all commands of the demo in sequence.
\end{filelist}

\section{Running the Extractor}
First, use the following command to change the current working directory '.' into a project directory:
\small
\begin{Verbatim}
% mkpr -p scmos_n -l 0.2 .
\end{Verbatim}
\normalsize
The command specifies the \io{scmos\_n} process from the technology library
and a lambda (design unit) of $0.2 \mu m$.
We use the mask names as defined in the \io{maskdata} file of the library.
And we are using the default technology file \io{space.def.s}
and parameter file \io{space.def.p} of the library.
\small
\begin{Verbatim}
% cgi crand.gds
\end{Verbatim}
\normalsize
Now, we can extract a circuit description for the layout of the \io{crand} cell, as follows:
\small
\begin{Verbatim}
% space -vFc crand
\end{Verbatim}
\normalsize
\small \begin{Verbatim}[frame=single]
Version 5.3.1, compiled on Fri Feb 03 12:45:53 GMT 2006
See http://www.space.tudelft.nl
parameter file: $ICDPATH/share/lib/process/scmos_n/space.def.p
technology file: $ICDPATH/share/lib/process/scmos_n/space.def.t
preprocessing crand (phase 1 - flattening layout)
preprocessing crand (phase 2 - removing overlap)
extracting crand

extraction statistics for layout crand:
	capacitances        : 221
	resistances         : 0
	nodes               : 222
	mos transistors     : 419
	bipolar vertical    : 0
	bipolar lateral     : 0
	substrate nodes     : 0

overall resource utilization:
	memory allocation  : 0.287 Mbyte
	user time          :         0.0
	system time        :         0.0
	real time          :         1.5   5%

space: --- Finished ---
\end{Verbatim}
\normalsize
You can show the resulting circuit with one of the circuit listing tools.
For example, to list the circuit in a SLS description, use \io{xsls} (see \io{icdman}).
\small
\begin{Verbatim}
% xsls crand
\end{Verbatim}
\normalsize
The output is default going to "stdout", a part is shown below:
\small \begin{Verbatim}[frame=single]
   ...
network crand (terminal out_7, out_6, out_5, out_4, out_3, out_2, out_1, out_0,
               inpar_7, inpar_6, inpar_5, inpar_4, inpar_3, inpar_2, inpar_1,
               inpar_0, serial, vss_lb, vss_lo, sc_l, nsc_l, vdd_lb, vdd_lo,
               nphi1_l, phi1_l, nphi2_l, phi2_l, phi1_r, phi2_r, nphi2_r,
               nphi1_r, vss_ro, vss_rb, vdd_rb, nsc_r, sc_r, vdd_ro)
{
    net {vdd_lo, vdd_ro};
    net {phi1_l, phi1_r};
    net {phi2_l, phi2_r};
    net {nphi2_l, nphi2_r};
    net {nphi1_l, nphi1_r};
    net {sc_l, sc_r};
    net {vdd_lb, vdd_rb};
    net {nsc_l, nsc_r};
    net {SUBSTR, vss_lb};
    net {SUBSTR, vss_rb};
    net {SUBSTR, vss_ro};
    net {SUBSTR, vss_lo};
    cap 2.8f (1, GND);
    nenh w=4u l=1.2u (14, 1, 14);
    cap 11.44f (2, GND);
    penh w=6.8u l=1.2u (14, 2, 14);
    cap 11.44f (3, GND);
    penh w=6.8u l=1.2u (vdd_lb, 3, 12);
    cap 2.8f (4, GND);
    nenh w=4u l=1.2u (SUBSTR, 4, SUBSTR);
    nenh w=4u l=1.2u (phi1_l, 13, 14);
    penh w=6.8u l=1.2u (nphi1_l, 13, 14);
    penh w=6.8u l=1.2u (serial, 12, vdd_lb);
    nenh w=4u l=1.2u (serial, 5, SUBSTR);
    nenh w=4u l=1.2u (phi2_l, 10, 13);
    penh w=6.8u l=1.2u (nphi2_l, 10, 13);
    penh w=6.8u l=1.2u (nsc_l, 12, vdd_lb);
    cap 3.2f (5, GND);
    nenh w=4u l=1.2u (nsc_l, 5, 14);
    cap 72.16f (6, GND);
    penh w=6.8u l=1.2u (vdd_lo, 6, 8);
    cap 16f (7, GND);
    nenh w=4u l=1.2u (SUBSTR, 7, 9);
    nenh w=4u l=1.2u (10, 10, 10);
    penh w=6.8u l=1.2u (10, 10, 10);
    penh w=6.8u l=1.2u (inpar_0, 12, 14);
    nenh w=4u l=1.2u (inpar_0, 11, 14);
    nenh w=7.2u l=1.2u (inpar_0, SUBSTR, SUBSTR);
    ...
    ...
}
\end{Verbatim}
\normalsize

\section{Running the Switch-Level Simulation}
For this simulation you are using the switch-level simulator \io{sls}.
See the "SLS: Switch-Level Simulator User's Manual" and for the manual page \manualpage{sls}.
This simulator is started from the simulation GUI \io{simeye} and
the results are shown in the output window (see \manualpage{simeye}).
\\[1 ex]
First, start the simulation GUI \io{simeye}.
\small
\begin{Verbatim}
% simeye
\end{Verbatim}
\normalsize
Second, prepare the simulation:
\\[1 ex]
Click on the "Simulate" menu and choice the "Prepare" item.
Select in the "Circuit:" field cell name "crand" and
in the "Stimuli:" field file name "crand.cmd" (click on it).
To inspect or edit the input signals, click on the "Edit" button.
\\[1 ex]
Third, start the switch-level simulation:
\\[1 ex]
Go back to the "Simulate" menu and choice the "Prepare" dialog item again:

\begin{figure}[h]
\centerline{\epsfig{figure=crand/dialog.eps, width=9cm}}
\end{figure}

\noindent
In the dialog window, choice simulation "Type: sls-timing" and for "Read: Analog".
Now, start the switch-level timing simulation by clicking on the "Run" button and wait for simulation results.
Below, you see the output waveforms.

\begin{figure}[h]
\centerline{\epsfig{figure=crand/simeye.eps, width=14cm}}
\end{figure}

Note, to exit \io{simeye},
go to the "File" menu and click on "Exit" and "Yes".

\chapter{crand Example of Extraction/Switch-Level Simulation}
\section{Introduction}
\label{PEintro}
In this example, we will be studying a random counter circuit.
We will see how Space is used for circuit extraction.
And how you can do a switch-level simulation of the circuit.
\\[1 ex]
The layout looks as follows, using the layout editor \io{dali} (see \manualpage{dali}):

\begin{figure}[h]
\centerline{\epsfig{figure=crand/crand.eps, width=15cm}}
\end{figure}

\section{Files}
This tutorial is located in the directory \CACDTOP{demo/crand}.
Initially, it contains the following files:
\begin{filelist}
\item[README] A file containing information about the demo.
\item[crand.cmd] Command file for circuit simulation.
\item[crand.gds] GDS2 file of the layout of the crand design.
\item[script.sh] Batch file for running all commands of the demo in sequence.
\end{filelist}

\section{Running the Extractor}
First, use the following command to change the current working directory '.' into a project directory:
\small
\begin{Verbatim}
% mkpr -p scmos_n -l 0.2 .
\end{Verbatim}
\normalsize
The command specifies the \io{scmos\_n} process from the technology library
and a lambda (design unit) of $0.2 \mu m$.
We use the mask names as defined in the \io{maskdata} file of the library.
And we are using the default technology file \io{space.def.s}
and parameter file \io{space.def.p} of the library.
\small
\begin{Verbatim}
% cgi crand.gds
\end{Verbatim}
\normalsize
Now, we can extract a circuit description for the layout of the \io{crand} cell, as follows:
\small
\begin{Verbatim}
% space -vFc crand
\end{Verbatim}
\normalsize
\small \begin{Verbatim}[frame=single]
Version 5.3.1, compiled on Fri Feb 03 12:45:53 GMT 2006
See http://www.space.tudelft.nl
parameter file: $ICDPATH/share/lib/process/scmos_n/space.def.p
technology file: $ICDPATH/share/lib/process/scmos_n/space.def.t
preprocessing crand (phase 1 - flattening layout)
preprocessing crand (phase 2 - removing overlap)
extracting crand

extraction statistics for layout crand:
	capacitances        : 221
	resistances         : 0
	nodes               : 222
	mos transistors     : 419
	bipolar vertical    : 0
	bipolar lateral     : 0
	substrate nodes     : 0

overall resource utilization:
	memory allocation  : 0.287 Mbyte
	user time          :         0.0
	system time        :         0.0
	real time          :         1.5   5%

space: --- Finished ---
\end{Verbatim}
\normalsize
You can show the resulting circuit with one of the circuit listing tools.
For example, to list the circuit in a SLS description, use \io{xsls} (see \io{icdman}).
\small
\begin{Verbatim}
% xsls crand
\end{Verbatim}
\normalsize
The output is default going to "stdout", a part is shown below:
\small \begin{Verbatim}[frame=single]
   ...
network crand (terminal out_7, out_6, out_5, out_4, out_3, out_2, out_1, out_0,
               inpar_7, inpar_6, inpar_5, inpar_4, inpar_3, inpar_2, inpar_1,
               inpar_0, serial, vss_lb, vss_lo, sc_l, nsc_l, vdd_lb, vdd_lo,
               nphi1_l, phi1_l, nphi2_l, phi2_l, phi1_r, phi2_r, nphi2_r,
               nphi1_r, vss_ro, vss_rb, vdd_rb, nsc_r, sc_r, vdd_ro)
{
    net {vdd_lo, vdd_ro};
    net {phi1_l, phi1_r};
    net {phi2_l, phi2_r};
    net {nphi2_l, nphi2_r};
    net {nphi1_l, nphi1_r};
    net {sc_l, sc_r};
    net {vdd_lb, vdd_rb};
    net {nsc_l, nsc_r};
    net {SUBSTR, vss_lb};
    net {SUBSTR, vss_rb};
    net {SUBSTR, vss_ro};
    net {SUBSTR, vss_lo};
    cap 2.8f (1, GND);
    nenh w=4u l=1.2u (14, 1, 14);
    cap 11.44f (2, GND);
    penh w=6.8u l=1.2u (14, 2, 14);
    cap 11.44f (3, GND);
    penh w=6.8u l=1.2u (vdd_lb, 3, 12);
    cap 2.8f (4, GND);
    nenh w=4u l=1.2u (SUBSTR, 4, SUBSTR);
    nenh w=4u l=1.2u (phi1_l, 13, 14);
    penh w=6.8u l=1.2u (nphi1_l, 13, 14);
    penh w=6.8u l=1.2u (serial, 12, vdd_lb);
    nenh w=4u l=1.2u (serial, 5, SUBSTR);
    nenh w=4u l=1.2u (phi2_l, 10, 13);
    penh w=6.8u l=1.2u (nphi2_l, 10, 13);
    penh w=6.8u l=1.2u (nsc_l, 12, vdd_lb);
    cap 3.2f (5, GND);
    nenh w=4u l=1.2u (nsc_l, 5, 14);
    cap 72.16f (6, GND);
    penh w=6.8u l=1.2u (vdd_lo, 6, 8);
    cap 16f (7, GND);
    nenh w=4u l=1.2u (SUBSTR, 7, 9);
    nenh w=4u l=1.2u (10, 10, 10);
    penh w=6.8u l=1.2u (10, 10, 10);
    penh w=6.8u l=1.2u (inpar_0, 12, 14);
    nenh w=4u l=1.2u (inpar_0, 11, 14);
    nenh w=7.2u l=1.2u (inpar_0, SUBSTR, SUBSTR);
    ...
    ...
}
\end{Verbatim}
\normalsize

\section{Running the Switch-Level Simulation}
For this simulation you are using the switch-level simulator \io{sls}.
See the "SLS: Switch-Level Simulator User's Manual" and for the manual page \manualpage{sls}.
This simulator is started from the simulation GUI \io{simeye} and
the results are shown in the output window (see \manualpage{simeye}).
\\[1 ex]
First, start the simulation GUI \io{simeye}.
\small
\begin{Verbatim}
% simeye
\end{Verbatim}
\normalsize
Second, prepare the simulation:
\\[1 ex]
Click on the "Simulate" menu and choice the "Prepare" item.
Select in the "Circuit:" field cell name "crand" and
in the "Stimuli:" field file name "crand.cmd" (click on it).
To inspect or edit the input signals, click on the "Edit" button.
\\[1 ex]
Third, start the switch-level simulation:
\\[1 ex]
Go back to the "Simulate" menu and choice the "Prepare" dialog item again:

\begin{figure}[h]
\centerline{\epsfig{figure=crand/dialog.eps, width=9cm}}
\end{figure}

\noindent
In the dialog window, choice simulation "Type: sls-timing" and for "Read: Analog".
Now, start the switch-level timing simulation by clicking on the "Run" button and wait for simulation results.
Below, you see the output waveforms.

\begin{figure}[h]
\centerline{\epsfig{figure=crand/simeye.eps, width=14cm}}
\end{figure}

Note, to exit \io{simeye},
go to the "File" menu and click on "Exit" and "Yes".

\chapter{crand Example of Extraction/Switch-Level Simulation}
\section{Introduction}
\label{PEintro}
In this example, we will be studying a random counter circuit.
We will see how Space is used for circuit extraction.
And how you can do a switch-level simulation of the circuit.
\\[1 ex]
The layout looks as follows, using the layout editor \io{dali} (see \manualpage{dali}):

\begin{figure}[h]
\centerline{\epsfig{figure=crand/crand.eps, width=15cm}}
\end{figure}

\section{Files}
This tutorial is located in the directory \CACDTOP{demo/crand}.
Initially, it contains the following files:
\begin{filelist}
\item[README] A file containing information about the demo.
\item[crand.cmd] Command file for circuit simulation.
\item[crand.gds] GDS2 file of the layout of the crand design.
\item[script.sh] Batch file for running all commands of the demo in sequence.
\end{filelist}

\section{Running the Extractor}
First, use the following command to change the current working directory '.' into a project directory:
\small
\begin{Verbatim}
% mkpr -p scmos_n -l 0.2 .
\end{Verbatim}
\normalsize
The command specifies the \io{scmos\_n} process from the technology library
and a lambda (design unit) of $0.2 \mu m$.
We use the mask names as defined in the \io{maskdata} file of the library.
And we are using the default technology file \io{space.def.s}
and parameter file \io{space.def.p} of the library.
\small
\begin{Verbatim}
% cgi crand.gds
\end{Verbatim}
\normalsize
Now, we can extract a circuit description for the layout of the \io{crand} cell, as follows:
\small
\begin{Verbatim}
% space -vFc crand
\end{Verbatim}
\normalsize
\small \begin{Verbatim}[frame=single]
Version 5.3.1, compiled on Fri Feb 03 12:45:53 GMT 2006
See http://www.space.tudelft.nl
parameter file: $ICDPATH/share/lib/process/scmos_n/space.def.p
technology file: $ICDPATH/share/lib/process/scmos_n/space.def.t
preprocessing crand (phase 1 - flattening layout)
preprocessing crand (phase 2 - removing overlap)
extracting crand

extraction statistics for layout crand:
	capacitances        : 221
	resistances         : 0
	nodes               : 222
	mos transistors     : 419
	bipolar vertical    : 0
	bipolar lateral     : 0
	substrate nodes     : 0

overall resource utilization:
	memory allocation  : 0.287 Mbyte
	user time          :         0.0
	system time        :         0.0
	real time          :         1.5   5%

space: --- Finished ---
\end{Verbatim}
\normalsize
You can show the resulting circuit with one of the circuit listing tools.
For example, to list the circuit in a SLS description, use \io{xsls} (see \io{icdman}).
\small
\begin{Verbatim}
% xsls crand
\end{Verbatim}
\normalsize
The output is default going to "stdout", a part is shown below:
\small \begin{Verbatim}[frame=single]
   ...
network crand (terminal out_7, out_6, out_5, out_4, out_3, out_2, out_1, out_0,
               inpar_7, inpar_6, inpar_5, inpar_4, inpar_3, inpar_2, inpar_1,
               inpar_0, serial, vss_lb, vss_lo, sc_l, nsc_l, vdd_lb, vdd_lo,
               nphi1_l, phi1_l, nphi2_l, phi2_l, phi1_r, phi2_r, nphi2_r,
               nphi1_r, vss_ro, vss_rb, vdd_rb, nsc_r, sc_r, vdd_ro)
{
    net {vdd_lo, vdd_ro};
    net {phi1_l, phi1_r};
    net {phi2_l, phi2_r};
    net {nphi2_l, nphi2_r};
    net {nphi1_l, nphi1_r};
    net {sc_l, sc_r};
    net {vdd_lb, vdd_rb};
    net {nsc_l, nsc_r};
    net {SUBSTR, vss_lb};
    net {SUBSTR, vss_rb};
    net {SUBSTR, vss_ro};
    net {SUBSTR, vss_lo};
    cap 2.8f (1, GND);
    nenh w=4u l=1.2u (14, 1, 14);
    cap 11.44f (2, GND);
    penh w=6.8u l=1.2u (14, 2, 14);
    cap 11.44f (3, GND);
    penh w=6.8u l=1.2u (vdd_lb, 3, 12);
    cap 2.8f (4, GND);
    nenh w=4u l=1.2u (SUBSTR, 4, SUBSTR);
    nenh w=4u l=1.2u (phi1_l, 13, 14);
    penh w=6.8u l=1.2u (nphi1_l, 13, 14);
    penh w=6.8u l=1.2u (serial, 12, vdd_lb);
    nenh w=4u l=1.2u (serial, 5, SUBSTR);
    nenh w=4u l=1.2u (phi2_l, 10, 13);
    penh w=6.8u l=1.2u (nphi2_l, 10, 13);
    penh w=6.8u l=1.2u (nsc_l, 12, vdd_lb);
    cap 3.2f (5, GND);
    nenh w=4u l=1.2u (nsc_l, 5, 14);
    cap 72.16f (6, GND);
    penh w=6.8u l=1.2u (vdd_lo, 6, 8);
    cap 16f (7, GND);
    nenh w=4u l=1.2u (SUBSTR, 7, 9);
    nenh w=4u l=1.2u (10, 10, 10);
    penh w=6.8u l=1.2u (10, 10, 10);
    penh w=6.8u l=1.2u (inpar_0, 12, 14);
    nenh w=4u l=1.2u (inpar_0, 11, 14);
    nenh w=7.2u l=1.2u (inpar_0, SUBSTR, SUBSTR);
    ...
    ...
}
\end{Verbatim}
\normalsize

\section{Running the Switch-Level Simulation}
For this simulation you are using the switch-level simulator \io{sls}.
See the "SLS: Switch-Level Simulator User's Manual" and for the manual page \manualpage{sls}.
This simulator is started from the simulation GUI \io{simeye} and
the results are shown in the output window (see \manualpage{simeye}).
\\[1 ex]
First, start the simulation GUI \io{simeye}.
\small
\begin{Verbatim}
% simeye
\end{Verbatim}
\normalsize
Second, prepare the simulation:
\\[1 ex]
Click on the "Simulate" menu and choice the "Prepare" item.
Select in the "Circuit:" field cell name "crand" and
in the "Stimuli:" field file name "crand.cmd" (click on it).
To inspect or edit the input signals, click on the "Edit" button.
\\[1 ex]
Third, start the switch-level simulation:
\\[1 ex]
Go back to the "Simulate" menu and choice the "Prepare" dialog item again:

\begin{figure}[h]
\centerline{\epsfig{figure=crand/dialog.eps, width=9cm}}
\end{figure}

\noindent
In the dialog window, choice simulation "Type: sls-timing" and for "Read: Analog".
Now, start the switch-level timing simulation by clicking on the "Run" button and wait for simulation results.
Below, you see the output waveforms.

\begin{figure}[h]
\centerline{\epsfig{figure=crand/simeye.eps, width=14cm}}
\end{figure}

Note, to exit \io{simeye},
go to the "File" menu and click on "Exit" and "Yes".

\chapter{crand Example of Extraction/Switch-Level Simulation}
\section{Introduction}
\label{PEintro}
In this example, we will be studying a random counter circuit.
We will see how Space is used for circuit extraction.
And how you can do a switch-level simulation of the circuit.
\\[1 ex]
The layout looks as follows, using the layout editor \io{dali} (see \manualpage{dali}):

\begin{figure}[h]
\centerline{\epsfig{figure=crand/crand.eps, width=15cm}}
\end{figure}

\section{Files}
This tutorial is located in the directory \CACDTOP{demo/crand}.
Initially, it contains the following files:
\begin{filelist}
\item[README] A file containing information about the demo.
\item[crand.cmd] Command file for circuit simulation.
\item[crand.gds] GDS2 file of the layout of the crand design.
\item[script.sh] Batch file for running all commands of the demo in sequence.
\end{filelist}

\section{Running the Extractor}
First, use the following command to change the current working directory '.' into a project directory:
\small
\begin{Verbatim}
% mkpr -p scmos_n -l 0.2 .
\end{Verbatim}
\normalsize
The command specifies the \io{scmos\_n} process from the technology library
and a lambda (design unit) of $0.2 \mu m$.
We use the mask names as defined in the \io{maskdata} file of the library.
And we are using the default technology file \io{space.def.s}
and parameter file \io{space.def.p} of the library.
\small
\begin{Verbatim}
% cgi crand.gds
\end{Verbatim}
\normalsize
Now, we can extract a circuit description for the layout of the \io{crand} cell, as follows:
\small
\begin{Verbatim}
% space -vFc crand
\end{Verbatim}
\normalsize
\small \begin{Verbatim}[frame=single]
Version 5.3.1, compiled on Fri Feb 03 12:45:53 GMT 2006
See http://www.space.tudelft.nl
parameter file: $ICDPATH/share/lib/process/scmos_n/space.def.p
technology file: $ICDPATH/share/lib/process/scmos_n/space.def.t
preprocessing crand (phase 1 - flattening layout)
preprocessing crand (phase 2 - removing overlap)
extracting crand

extraction statistics for layout crand:
	capacitances        : 221
	resistances         : 0
	nodes               : 222
	mos transistors     : 419
	bipolar vertical    : 0
	bipolar lateral     : 0
	substrate nodes     : 0

overall resource utilization:
	memory allocation  : 0.287 Mbyte
	user time          :         0.0
	system time        :         0.0
	real time          :         1.5   5%

space: --- Finished ---
\end{Verbatim}
\normalsize
You can show the resulting circuit with one of the circuit listing tools.
For example, to list the circuit in a SLS description, use \io{xsls} (see \io{icdman}).
\small
\begin{Verbatim}
% xsls crand
\end{Verbatim}
\normalsize
The output is default going to "stdout", a part is shown below:
\small \begin{Verbatim}[frame=single]
   ...
network crand (terminal out_7, out_6, out_5, out_4, out_3, out_2, out_1, out_0,
               inpar_7, inpar_6, inpar_5, inpar_4, inpar_3, inpar_2, inpar_1,
               inpar_0, serial, vss_lb, vss_lo, sc_l, nsc_l, vdd_lb, vdd_lo,
               nphi1_l, phi1_l, nphi2_l, phi2_l, phi1_r, phi2_r, nphi2_r,
               nphi1_r, vss_ro, vss_rb, vdd_rb, nsc_r, sc_r, vdd_ro)
{
    net {vdd_lo, vdd_ro};
    net {phi1_l, phi1_r};
    net {phi2_l, phi2_r};
    net {nphi2_l, nphi2_r};
    net {nphi1_l, nphi1_r};
    net {sc_l, sc_r};
    net {vdd_lb, vdd_rb};
    net {nsc_l, nsc_r};
    net {SUBSTR, vss_lb};
    net {SUBSTR, vss_rb};
    net {SUBSTR, vss_ro};
    net {SUBSTR, vss_lo};
    cap 2.8f (1, GND);
    nenh w=4u l=1.2u (14, 1, 14);
    cap 11.44f (2, GND);
    penh w=6.8u l=1.2u (14, 2, 14);
    cap 11.44f (3, GND);
    penh w=6.8u l=1.2u (vdd_lb, 3, 12);
    cap 2.8f (4, GND);
    nenh w=4u l=1.2u (SUBSTR, 4, SUBSTR);
    nenh w=4u l=1.2u (phi1_l, 13, 14);
    penh w=6.8u l=1.2u (nphi1_l, 13, 14);
    penh w=6.8u l=1.2u (serial, 12, vdd_lb);
    nenh w=4u l=1.2u (serial, 5, SUBSTR);
    nenh w=4u l=1.2u (phi2_l, 10, 13);
    penh w=6.8u l=1.2u (nphi2_l, 10, 13);
    penh w=6.8u l=1.2u (nsc_l, 12, vdd_lb);
    cap 3.2f (5, GND);
    nenh w=4u l=1.2u (nsc_l, 5, 14);
    cap 72.16f (6, GND);
    penh w=6.8u l=1.2u (vdd_lo, 6, 8);
    cap 16f (7, GND);
    nenh w=4u l=1.2u (SUBSTR, 7, 9);
    nenh w=4u l=1.2u (10, 10, 10);
    penh w=6.8u l=1.2u (10, 10, 10);
    penh w=6.8u l=1.2u (inpar_0, 12, 14);
    nenh w=4u l=1.2u (inpar_0, 11, 14);
    nenh w=7.2u l=1.2u (inpar_0, SUBSTR, SUBSTR);
    ...
    ...
}
\end{Verbatim}
\normalsize

\section{Running the Switch-Level Simulation}
For this simulation you are using the switch-level simulator \io{sls}.
See the "SLS: Switch-Level Simulator User's Manual" and for the manual page \manualpage{sls}.
This simulator is started from the simulation GUI \io{simeye} and
the results are shown in the output window (see \manualpage{simeye}).
\\[1 ex]
First, start the simulation GUI \io{simeye}.
\small
\begin{Verbatim}
% simeye
\end{Verbatim}
\normalsize
Second, prepare the simulation:
\\[1 ex]
Click on the "Simulate" menu and choice the "Prepare" item.
Select in the "Circuit:" field cell name "crand" and
in the "Stimuli:" field file name "crand.cmd" (click on it).
To inspect or edit the input signals, click on the "Edit" button.
\\[1 ex]
Third, start the switch-level simulation:
\\[1 ex]
Go back to the "Simulate" menu and choice the "Prepare" dialog item again:

\begin{figure}[h]
\centerline{\epsfig{figure=crand/dialog.eps, width=9cm}}
\end{figure}

\noindent
In the dialog window, choice simulation "Type: sls-timing" and for "Read: Analog".
Now, start the switch-level timing simulation by clicking on the "Run" button and wait for simulation results.
Below, you see the output waveforms.

\begin{figure}[h]
\centerline{\epsfig{figure=crand/simeye.eps, width=14cm}}
\end{figure}

Note, to exit \io{simeye},
go to the "File" menu and click on "Exit" and "Yes".

\chapter{crand Example of Extraction/Switch-Level Simulation}
\section{Introduction}
\label{PEintro}
In this example, we will be studying a random counter circuit.
We will see how Space is used for circuit extraction.
And how you can do a switch-level simulation of the circuit.
\\[1 ex]
The layout looks as follows, using the layout editor \io{dali} (see \manualpage{dali}):

\begin{figure}[h]
\centerline{\epsfig{figure=crand/crand.eps, width=15cm}}
\end{figure}

\section{Files}
This tutorial is located in the directory \CACDTOP{demo/crand}.
Initially, it contains the following files:
\begin{filelist}
\item[README] A file containing information about the demo.
\item[crand.cmd] Command file for circuit simulation.
\item[crand.gds] GDS2 file of the layout of the crand design.
\item[script.sh] Batch file for running all commands of the demo in sequence.
\end{filelist}

\section{Running the Extractor}
First, use the following command to change the current working directory '.' into a project directory:
\small
\begin{Verbatim}
% mkpr -p scmos_n -l 0.2 .
\end{Verbatim}
\normalsize
The command specifies the \io{scmos\_n} process from the technology library
and a lambda (design unit) of $0.2 \mu m$.
We use the mask names as defined in the \io{maskdata} file of the library.
And we are using the default technology file \io{space.def.s}
and parameter file \io{space.def.p} of the library.
\small
\begin{Verbatim}
% cgi crand.gds
\end{Verbatim}
\normalsize
Now, we can extract a circuit description for the layout of the \io{crand} cell, as follows:
\small
\begin{Verbatim}
% space -vFc crand
\end{Verbatim}
\normalsize
\small \begin{Verbatim}[frame=single]
Version 5.3.1, compiled on Fri Feb 03 12:45:53 GMT 2006
See http://www.space.tudelft.nl
parameter file: $ICDPATH/share/lib/process/scmos_n/space.def.p
technology file: $ICDPATH/share/lib/process/scmos_n/space.def.t
preprocessing crand (phase 1 - flattening layout)
preprocessing crand (phase 2 - removing overlap)
extracting crand

extraction statistics for layout crand:
	capacitances        : 221
	resistances         : 0
	nodes               : 222
	mos transistors     : 419
	bipolar vertical    : 0
	bipolar lateral     : 0
	substrate nodes     : 0

overall resource utilization:
	memory allocation  : 0.287 Mbyte
	user time          :         0.0
	system time        :         0.0
	real time          :         1.5   5%

space: --- Finished ---
\end{Verbatim}
\normalsize
You can show the resulting circuit with one of the circuit listing tools.
For example, to list the circuit in a SLS description, use \io{xsls} (see \io{icdman}).
\small
\begin{Verbatim}
% xsls crand
\end{Verbatim}
\normalsize
The output is default going to "stdout", a part is shown below:
\small \begin{Verbatim}[frame=single]
   ...
network crand (terminal out_7, out_6, out_5, out_4, out_3, out_2, out_1, out_0,
               inpar_7, inpar_6, inpar_5, inpar_4, inpar_3, inpar_2, inpar_1,
               inpar_0, serial, vss_lb, vss_lo, sc_l, nsc_l, vdd_lb, vdd_lo,
               nphi1_l, phi1_l, nphi2_l, phi2_l, phi1_r, phi2_r, nphi2_r,
               nphi1_r, vss_ro, vss_rb, vdd_rb, nsc_r, sc_r, vdd_ro)
{
    net {vdd_lo, vdd_ro};
    net {phi1_l, phi1_r};
    net {phi2_l, phi2_r};
    net {nphi2_l, nphi2_r};
    net {nphi1_l, nphi1_r};
    net {sc_l, sc_r};
    net {vdd_lb, vdd_rb};
    net {nsc_l, nsc_r};
    net {SUBSTR, vss_lb};
    net {SUBSTR, vss_rb};
    net {SUBSTR, vss_ro};
    net {SUBSTR, vss_lo};
    cap 2.8f (1, GND);
    nenh w=4u l=1.2u (14, 1, 14);
    cap 11.44f (2, GND);
    penh w=6.8u l=1.2u (14, 2, 14);
    cap 11.44f (3, GND);
    penh w=6.8u l=1.2u (vdd_lb, 3, 12);
    cap 2.8f (4, GND);
    nenh w=4u l=1.2u (SUBSTR, 4, SUBSTR);
    nenh w=4u l=1.2u (phi1_l, 13, 14);
    penh w=6.8u l=1.2u (nphi1_l, 13, 14);
    penh w=6.8u l=1.2u (serial, 12, vdd_lb);
    nenh w=4u l=1.2u (serial, 5, SUBSTR);
    nenh w=4u l=1.2u (phi2_l, 10, 13);
    penh w=6.8u l=1.2u (nphi2_l, 10, 13);
    penh w=6.8u l=1.2u (nsc_l, 12, vdd_lb);
    cap 3.2f (5, GND);
    nenh w=4u l=1.2u (nsc_l, 5, 14);
    cap 72.16f (6, GND);
    penh w=6.8u l=1.2u (vdd_lo, 6, 8);
    cap 16f (7, GND);
    nenh w=4u l=1.2u (SUBSTR, 7, 9);
    nenh w=4u l=1.2u (10, 10, 10);
    penh w=6.8u l=1.2u (10, 10, 10);
    penh w=6.8u l=1.2u (inpar_0, 12, 14);
    nenh w=4u l=1.2u (inpar_0, 11, 14);
    nenh w=7.2u l=1.2u (inpar_0, SUBSTR, SUBSTR);
    ...
    ...
}
\end{Verbatim}
\normalsize

\section{Running the Switch-Level Simulation}
For this simulation you are using the switch-level simulator \io{sls}.
See the "SLS: Switch-Level Simulator User's Manual" and for the manual page \manualpage{sls}.
This simulator is started from the simulation GUI \io{simeye} and
the results are shown in the output window (see \manualpage{simeye}).
\\[1 ex]
First, start the simulation GUI \io{simeye}.
\small
\begin{Verbatim}
% simeye
\end{Verbatim}
\normalsize
Second, prepare the simulation:
\\[1 ex]
Click on the "Simulate" menu and choice the "Prepare" item.
Select in the "Circuit:" field cell name "crand" and
in the "Stimuli:" field file name "crand.cmd" (click on it).
To inspect or edit the input signals, click on the "Edit" button.
\\[1 ex]
Third, start the switch-level simulation:
\\[1 ex]
Go back to the "Simulate" menu and choice the "Prepare" dialog item again:

\begin{figure}[h]
\centerline{\epsfig{figure=crand/dialog.eps, width=9cm}}
\end{figure}

\noindent
In the dialog window, choice simulation "Type: sls-timing" and for "Read: Analog".
Now, start the switch-level timing simulation by clicking on the "Run" button and wait for simulation results.
Below, you see the output waveforms.

\begin{figure}[h]
\centerline{\epsfig{figure=crand/simeye.eps, width=14cm}}
\end{figure}

Note, to exit \io{simeye},
go to the "File" menu and click on "Exit" and "Yes".

\chapter{crand Example of Extraction/Switch-Level Simulation}
\section{Introduction}
\label{PEintro}
In this example, we will be studying a random counter circuit.
We will see how Space is used for circuit extraction.
And how you can do a switch-level simulation of the circuit.
\\[1 ex]
The layout looks as follows, using the layout editor \io{dali} (see \manualpage{dali}):

\begin{figure}[h]
\centerline{\epsfig{figure=crand/crand.eps, width=15cm}}
\end{figure}

\section{Files}
This tutorial is located in the directory \CACDTOP{demo/crand}.
Initially, it contains the following files:
\begin{filelist}
\item[README] A file containing information about the demo.
\item[crand.cmd] Command file for circuit simulation.
\item[crand.gds] GDS2 file of the layout of the crand design.
\item[script.sh] Batch file for running all commands of the demo in sequence.
\end{filelist}

\section{Running the Extractor}
First, use the following command to change the current working directory '.' into a project directory:
\small
\begin{Verbatim}
% mkpr -p scmos_n -l 0.2 .
\end{Verbatim}
\normalsize
The command specifies the \io{scmos\_n} process from the technology library
and a lambda (design unit) of $0.2 \mu m$.
We use the mask names as defined in the \io{maskdata} file of the library.
And we are using the default technology file \io{space.def.s}
and parameter file \io{space.def.p} of the library.
\small
\begin{Verbatim}
% cgi crand.gds
\end{Verbatim}
\normalsize
Now, we can extract a circuit description for the layout of the \io{crand} cell, as follows:
\small
\begin{Verbatim}
% space -vFc crand
\end{Verbatim}
\normalsize
\small \begin{Verbatim}[frame=single]
Version 5.3.1, compiled on Fri Feb 03 12:45:53 GMT 2006
See http://www.space.tudelft.nl
parameter file: $ICDPATH/share/lib/process/scmos_n/space.def.p
technology file: $ICDPATH/share/lib/process/scmos_n/space.def.t
preprocessing crand (phase 1 - flattening layout)
preprocessing crand (phase 2 - removing overlap)
extracting crand

extraction statistics for layout crand:
	capacitances        : 221
	resistances         : 0
	nodes               : 222
	mos transistors     : 419
	bipolar vertical    : 0
	bipolar lateral     : 0
	substrate nodes     : 0

overall resource utilization:
	memory allocation  : 0.287 Mbyte
	user time          :         0.0
	system time        :         0.0
	real time          :         1.5   5%

space: --- Finished ---
\end{Verbatim}
\normalsize
You can show the resulting circuit with one of the circuit listing tools.
For example, to list the circuit in a SLS description, use \io{xsls} (see \io{icdman}).
\small
\begin{Verbatim}
% xsls crand
\end{Verbatim}
\normalsize
The output is default going to "stdout", a part is shown below:
\small \begin{Verbatim}[frame=single]
   ...
network crand (terminal out_7, out_6, out_5, out_4, out_3, out_2, out_1, out_0,
               inpar_7, inpar_6, inpar_5, inpar_4, inpar_3, inpar_2, inpar_1,
               inpar_0, serial, vss_lb, vss_lo, sc_l, nsc_l, vdd_lb, vdd_lo,
               nphi1_l, phi1_l, nphi2_l, phi2_l, phi1_r, phi2_r, nphi2_r,
               nphi1_r, vss_ro, vss_rb, vdd_rb, nsc_r, sc_r, vdd_ro)
{
    net {vdd_lo, vdd_ro};
    net {phi1_l, phi1_r};
    net {phi2_l, phi2_r};
    net {nphi2_l, nphi2_r};
    net {nphi1_l, nphi1_r};
    net {sc_l, sc_r};
    net {vdd_lb, vdd_rb};
    net {nsc_l, nsc_r};
    net {SUBSTR, vss_lb};
    net {SUBSTR, vss_rb};
    net {SUBSTR, vss_ro};
    net {SUBSTR, vss_lo};
    cap 2.8f (1, GND);
    nenh w=4u l=1.2u (14, 1, 14);
    cap 11.44f (2, GND);
    penh w=6.8u l=1.2u (14, 2, 14);
    cap 11.44f (3, GND);
    penh w=6.8u l=1.2u (vdd_lb, 3, 12);
    cap 2.8f (4, GND);
    nenh w=4u l=1.2u (SUBSTR, 4, SUBSTR);
    nenh w=4u l=1.2u (phi1_l, 13, 14);
    penh w=6.8u l=1.2u (nphi1_l, 13, 14);
    penh w=6.8u l=1.2u (serial, 12, vdd_lb);
    nenh w=4u l=1.2u (serial, 5, SUBSTR);
    nenh w=4u l=1.2u (phi2_l, 10, 13);
    penh w=6.8u l=1.2u (nphi2_l, 10, 13);
    penh w=6.8u l=1.2u (nsc_l, 12, vdd_lb);
    cap 3.2f (5, GND);
    nenh w=4u l=1.2u (nsc_l, 5, 14);
    cap 72.16f (6, GND);
    penh w=6.8u l=1.2u (vdd_lo, 6, 8);
    cap 16f (7, GND);
    nenh w=4u l=1.2u (SUBSTR, 7, 9);
    nenh w=4u l=1.2u (10, 10, 10);
    penh w=6.8u l=1.2u (10, 10, 10);
    penh w=6.8u l=1.2u (inpar_0, 12, 14);
    nenh w=4u l=1.2u (inpar_0, 11, 14);
    nenh w=7.2u l=1.2u (inpar_0, SUBSTR, SUBSTR);
    ...
    ...
}
\end{Verbatim}
\normalsize

\section{Running the Switch-Level Simulation}
For this simulation you are using the switch-level simulator \io{sls}.
See the "SLS: Switch-Level Simulator User's Manual" and for the manual page \manualpage{sls}.
This simulator is started from the simulation GUI \io{simeye} and
the results are shown in the output window (see \manualpage{simeye}).
\\[1 ex]
First, start the simulation GUI \io{simeye}.
\small
\begin{Verbatim}
% simeye
\end{Verbatim}
\normalsize
Second, prepare the simulation:
\\[1 ex]
Click on the "Simulate" menu and choice the "Prepare" item.
Select in the "Circuit:" field cell name "crand" and
in the "Stimuli:" field file name "crand.cmd" (click on it).
To inspect or edit the input signals, click on the "Edit" button.
\\[1 ex]
Third, start the switch-level simulation:
\\[1 ex]
Go back to the "Simulate" menu and choice the "Prepare" dialog item again:

\begin{figure}[h]
\centerline{\epsfig{figure=crand/dialog.eps, width=9cm}}
\end{figure}

\noindent
In the dialog window, choice simulation "Type: sls-timing" and for "Read: Analog".
Now, start the switch-level timing simulation by clicking on the "Run" button and wait for simulation results.
Below, you see the output waveforms.

\begin{figure}[h]
\centerline{\epsfig{figure=crand/simeye.eps, width=14cm}}
\end{figure}

Note, to exit \io{simeye},
go to the "File" menu and click on "Exit" and "Yes".


% \appendix

\end{document}
