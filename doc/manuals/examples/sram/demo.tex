\chapter{sram Example of 3D Capacitance Extraction}
\section{Introduction}
\label{SRintro}
In this tutorial, we will be studying a CMOS static RAM example.
We will see how Space is used to compute 3D capacitances.
See also the "Space 3D Capacitance Extraction User's Manual".
\\[1 ex]
This is a more serious example than the 5 parallel conductors example \io{poly5}.
The layout looks as follows, using the layout editor \io{dali} (see \manualpage{dali}):

\begin{figure}[h]
\centerline{\epsfig{figure=sram/sram.eps, width=12cm}}
\end{figure}

\section{Files}
This tutorial is located in the directory \CACDTOP{demo/sram}.
Initially, it contains the following files:
\begin{filelist}
\item[README] A file containing information about the demo.
\item[sram.gds] The layout of the sram design.
\item[sram.s] The technology file for Space. See Section \ref{SRtech}.
\item[sram.p] The parameter settings file for Space. See Section \ref{SRparam}.
\item[sram.cmd] A command file with stimuli to simulate the circuit.
\item[script.sh] A file containing the commands for executing all
steps of the demo in sequence.
\end{filelist}

\section{Technology File}
\label{SRtech}
The technology file that we will use is the file \io{sram.s},
of which a number of parts are listed below.

\small \begin{Verbatim}[frame=single]
listing of file sram.s
       ...
 22  unit vdimension    1e-6  # um
 23  unit shape         1e-6  # um
       ...
 27  colors :
 28     cpg   red
 29     caa   green
 30     cmf   blue
 31     cms   gold
 32     cca   black
 33     ccp   black
 34     cva   black
 35     cwn   glass
 36     csn   glass
 37     cog   glass
 38
 39  conductors :
 40  # name    : condition     : mask : resistivity : type
 41    cond_mf : cmf           : cmf  : 0.045       : m   # first metal
 42    cond_ms : cms           : cms  : 0.030       : m   # second metal
 43    cond_pg : cpg           : cpg  : 40          : m   # poly interc.
 44    cond_pa : caa !cpg !csn : caa  : 70          : p   # p+ active area
 45    cond_na : caa !cpg  csn : caa  : 50          : n   # n+ active area
 46
 47  fets :
 48  # name : condition    : gate d/s
 49    nenh : cpg caa  csn : cpg  caa     # nenh MOS
 50    penh : cpg caa !csn : cpg  caa     # penh MOS
 51
 52  contacts :
 53  # name   : condition        : lay1 lay2 : resistivity
 54    cont_s : cva cms cmf      : cms  cmf  :   1    # metal to metal2
 55    cont_p : ccp cmf cpg      : cmf  cpg  : 100    # metal to poly
 56    cont_a : cca cmf caa !cpg : cmf  caa  : 100    # metal to act. area
 57
 58  capacitances :
       ...
       ...
 96
 97  vdimensions :
 98     ver_caa_on_all : caa !cpg   : caa : 0.30 0.00
 99     ver_cpg_of_caa : cpg !caa   : cpg : 0.60 0.50
100     ver_cpg_on_caa : cpg caa    : cpg : 0.35 0.70
101     ver_cmf        : cmf        : cmf : 1.70 0.70
102     ver_cms        : cms        : cms : 2.80 0.70
103
104  dielectrics :
105     # Dielectric consists of 5 micron thick SiO2
106     # (epsilon = 3.9) on a conducting plane.
107     SiO2   3.9   0.0
108     air    1.0   5.0
109
110  #EOF
\end{Verbatim}
\normalsize
Please note that the masks to be used in this file have already been defined
in the \io{maskdata} file in the process library \CACDTOP{lib/process}.
In this example, we will only use the \io{vdimensions} of 4 masks.

When we look to the file, we first see on line \io{22} an \io{unit} specification
for the \io{vdimensions} definitions starting on line \io{97}.
An unit of \io{1e-6} $m$ is chosen for the values.
Thus, the values must be specified in $\mu m$.
On line \io{23} is an unit specified for possible \io{eshapes} definitions.
For shape definitions, see sections 3.4 and 3.5 of the "Space 3D Capacitance User's Manual".

Starting on line \io{27} we find the \io{colors} specification.
Note that this information is only used by the visualization program \io{Xspace}.
For a picture, see section 5.2 of the User's Manual.
When the color is \io{glass}, then the mask is invisible.

Starting on line \io{39} we find the \io{conductors} definitions.
This is an important section, because without this we can not specify
element pins and even a vdimension.
The conductor values are not important,
because we don't extract resistances in this example.
On line \io{44} and \io{45} are two different conductors specified for the \io{caa} mask.
These are the drain/source regions, also called diffused conductors.
These conductors have a conductor type, respectivily \io{p} and \io{n}.

We skip all 2D \io{capacitances} definitions, starting on line \io{58}.
Because, for 3D capacitance extraction, we need only the following two sections.
First, the \io{vdimensions} definitions for building the 3D conductor mesh.
This section specifies respectivily the bottom (against the ground plane) and the thickness
of each conductor.
You find, on line \io{98}, only one specification for the diffused conductors.
And these conductors are modeled with a zero thickness (see section 3.8 of the User's Manual).
On lines \io{99} and \io{100}, you see, that the gate conductor \io{cpg} is modeled with
two different heights and thicknesses.

Second, the \io{dielectrics} definitions specify the dielectric properties of the media
around the conductors, respectivily the relative permitivity $\epsilon_r$ and the bottom in $\mu m$.
Note that the thickness of the last medium (\io{air}) extends to infinity.

\section{Parameter File}
\label{SRparam}
The parameter file \io{sram.p} is controlling the extraction, see listing below.

\small \begin{Verbatim}[frame=single]
listing of file sram.p
        ...
   3
   4    BEGIN cap3d
   5    be_mode                 0c
   6    be_window               2.0
   7    max_be_area             1.0
   8    omit_gate_ds_cap        on
   9    END cap3d
\end{Verbatim}
\normalsize
In the above listing, we see a \io{cap3d} block from line \io{4} to line \io{9}.
These parameters are only used for controlling capacitance 3D extraction.
The \io{max\_be\_area} (in $\mu m^2$) and \io{be\_window} (width in $\mu m$) parameters
needs always to be specified.
On line \io{5}, you find also the \io{be\_mode} parameter.
However, this boundary element mode is the default (see the 3D Cap. User's Manual).
On line \io{8} you see the \io{omit\_gate\_ds\_cap} parameter.
With this parameter, you can skip the couple capacitances between gate and drain/source regions.
Possibly, these capacitances are already present in the used SPICE device model
(see section 3.9 of the User's Manual).

\section{Running the Extractor}
For example, you have copied this demo files to a directory \io{exam2}:
\small
\begin{Verbatim}
% cd exam2
% cp $ICDPATH/share/demo/sram/* .
\end{Verbatim}
\normalsize
The file \io{script.sh} is a batch file for running all the commands
for this example.
First, it changes the current working directory '.' into a project directory:
\small
\begin{Verbatim}
% mkpr -p scmos_n -l 0.25 .
\end{Verbatim}
\normalsize
The static RAM cell is in $0.5 \mu$ technology.
Thus, we use a lambda value (option \io{-l}) of $0.25 \mu m$.
For the CMOS technology, we use the \io{scmos\_n} process from the technology library.
We use the mask names as defined in the \io{maskdata} file of the library.
But, we are not using the default technology
file \io{space.def.s} and parameter file \io{space.def.p} of the library.
Thus, the local technology file \io{sram.s} needs to be compiled with \io{tecc} to the format
(a \io{.t} file) which is used by the extractor:
\small
\begin{Verbatim}
% tecc sram.s
\end{Verbatim}
\normalsize
Second, the layout description is put into the project database.
The layout is supplied in a GDS2 file, which can be converted to
internal database format with the \io{cgi} program:
\small
\begin{Verbatim}
% cgi sram.gds
\end{Verbatim}
\normalsize
Now, we can extract a circuit description for the layout of the \io{sram} cell, as follows:
\small
\begin{Verbatim}
% space3d -C3 -E sram.t -P sram.p sram
\end{Verbatim}
\normalsize
In this case, the capacitances are calculated, using a 3D mesh method (BEM).
The method calculates couple capacitances between the conductors and between the conductors
and a ground plane.
\\[1 ex]
The extracted circuit can be inspected using the \io{xspice} program.
We use option \io{-a} to get alpha-numeric node names.
\small
\begin{Verbatim}
% xspice -a sram
\end{Verbatim}
\normalsize
The output of the SPICE circuit is listed below.

\small \begin{Verbatim}[frame=single]
  1   sram
  2
  3   * Generated by: xspice 2.39 25-Jan-2006
  4   * Date: 20-Jun-06 10:40:34 GMT
  5   * Path: /users/simon/exam2
  6   * Language: SPICE
  7
  8   * circuit sram pbulk nbulk word vdd b_vss t_vss c1 c2 bit notbit
  9   m1 vdd c1 c2 pbulk penh_0 w=500n l=500n
 10   m2 vdd c2 c1 pbulk penh_0 w=500n l=500n
 11   m3 b_vss c1 c2 nbulk nenh_0 w=500n l=500n
 12   m4 t_vss c2 c1 nbulk nenh_0 w=500n l=500n
 13   m5 notbit word c2 nbulk nenh_0 w=500n l=500n
 14   m6 bit word c1 nbulk nenh_0 w=500n l=500n
 15   c1 b_vss word 114.8418e-18
 16   c2 b_vss vdd 43.84559e-18
 17   c3 b_vss c2 74.079e-18
 18   c4 b_vss c1 212.2091e-18
 19   c5 b_vss notbit 610.2115e-18
 20   c6 b_vss GND 1.80899f
 21   c7 notbit bit 107.3522e-18
 22   c8 notbit word 115.7189e-18
 23   c9 notbit vdd 6.609528e-18
 24   c10 notbit c2 360.549e-18
 25   c11 notbit c1 66.39728e-18
 26   c12 notbit GND 2.597814f
 27   c13 t_vss word 114.8457e-18
 28   c14 t_vss vdd 43.82952e-18
 29   c15 t_vss bit 610.3344e-18
 30   c16 t_vss c2 212.2316e-18
 31   c17 t_vss c1 73.78773e-18
 32   c18 t_vss GND 1.808788f
 33   c19 word bit 115.8338e-18
 34   c20 word c2 113.7943e-18
 35   c21 word c1 111.6566e-18
 36   c22 word GND 344.449e-18
 37   c23 vdd bit 6.60235e-18
 38   c24 vdd c2 56.77325e-18
 39   c25 vdd c1 58.65703e-18
 40   c26 vdd GND 9.5875f
 41   c27 bit c1 360.2213e-18
 42   c28 bit c2 67.05254e-18
 43   c29 bit GND 2.597536f
 44   c30 c2 c1 998.1198e-18
 45   c31 c2 GND 5.426256f
 46   c32 c1 GND 5.425883f
 47   * end sram
 48
 49   .model penh_0 pmos(level=2 ld=0 tox=25n nsub=50e15 vto=-1.1 uo=200 uexp=100m
 50   + ucrit=10k delta=200m xj=500n vmax=50k neff=1 rsh=0 nfs=0 js=10u cj=500u
 51   + cjsw=600p mj=500m mjsw=300m pb=800m cgdo=300p cgso=300p)
 52   .model nenh_0 nmos(level=2 ld=0 tox=25n nsub=20e15 vto=700m uo=600 uexp=100m
 53   + ucrit=10k delta=200m xj=500n vmax=50k neff=1 rsh=0 nfs=0 js=2u cj=100u
 54   + cjsw=300p mj=500m mjsw=300m pb=800m cgdo=300p cgso=300p)
 55
 56
 57   vpbulk pbulk 0 5
 58   rpbulk pbulk 0 100meg
 59
 60   vnbulk nbulk 0 0
 61   rnbulk nbulk 0 100meg
\end{Verbatim}
\normalsize
On lines \io{49} to \io{54}, you find the
device models for \io{penh\_0} and \io{nenh\_0} fets.
Lines \io{57} to \io{61} are automatically added,
and are the \io{pbulk} and \io{nbulk} voltage sources and resistors.
Below, you find a drawing of the listing of the extracted circuit.

\begin{figure}[h]
\centerline{\epsfig{figure=sram/circuit.eps, width=11cm}}
\end{figure}

Note that there are couple capacitances between almost all nodes of the circuit.
This happens, because the chosen \io{be\_window} is $2 \mu m$ width.
Each node has also a couple capacitance to the ground plane (node \io{GND}).

\section{Running a Circuit Simulation}
With the simulation GUI \io{simeye} you can run a SLS or SPICE circuit simulation.
With SLS, you can choice between a logic and timing simulation.
The timing simulation is also using the extracted couple capacitances.
An accurate analog simulation is performed with the SPICE simulator.
The following command file with stimuli is used for both simulators:

\small \begin{Verbatim}[frame=single]
listing of file sram.cmd
      ...
  3   /*
  4   /* Command file for simulating the sram circuit with simeye/nspice.
  5   */
  6
  7   print vdd t_vss b_vss word bit notbit c1 c2
  8   plot  vdd t_vss b_vss word bit notbit c1 c2
  9   option simperiod = 30
 10   option sigunit = 0.75e-10
 11   option outacc  = 10p
 12   option level = 3
 13
 14   /*  spice commands:
 15
 16   *%
 17   trise 0.075n
 18   tfall 0.075n
 19   tstep 0.0125n
 20   *%
 21   .options nomod
 22   .options limpts=601
 23   .options cptime=30
 24   *%
 25
 26   */
 27   set vdd = h*25
 28   set t_vss = l*25
 29   set b_vss = l*25
 30   set word = l*5 h*5 l*5 h*5 l*5 h
 31   set notbit = l*10 h*10 l*5
 32   set bit = h*10 l*10 h*5
\end{Verbatim}
\normalsize
For a SPICE simulation, \io{simeye} is using the script \io{nspice}.
First, the script contains commands to prepare the circuit and to translate
the stimuli to pwl voltage sources.
Note that these stimuli are connected between the nodes and the ground node \io{0}.
On line \io{27} you see node \io{vdd}, the positive power supply node,
which gets a high voltage (default $5 V$).
On lines \io{28} and \io{29} you see the \io{vss} nodes,
which get a low voltage of $0 V$ and are connected to the ground node.
Second, the script tries to run the Berkeley \io{spice3} simulator.
For more information, see \manualpage{nspice}.
\\[1 ex]
Below, you see the \io{simeye} Simulate Prepare Dialog.
\begin{figure}[h]
\centerline{\epsfig{figure=sram/simdialog.eps, width=10cm}}
\end{figure}
\\[1 ex]
On the next page, you find the results of a logic SLS and analog SPICE simulation.
\newpage
Below, you see the result of a logic SLS simulation.
\begin{figure}[h]
\centerline{\epsfig{figure=sram/simeye1.eps, height=9cm}}
\end{figure}

Below, you see the result of a analog SPICE simulation.
\begin{figure}[h]
\centerline{\epsfig{figure=sram/simeye2.eps, height=9cm}}
\end{figure}
