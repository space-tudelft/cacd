\chapter{poly5 Example of 3D Capacitance Extraction}
\section{Introduction}
\label{PEintro}
In this tutorial, we will be studying the poly5 example.
We will see how Space is used to compute 3D capacitances.
See also the "Space 3D Capacitance Extraction User's Manual".
\\[1 ex]
The example consists out of a configuration of 5 parallel conductors.
The conductors have a length of 5 micron, a width of 0.5 micron,
a height of 0.5 micron and their separation is also 0.5 micron.
Each conductor is assigned a terminal (port).
Thus, each conductor can easy be found in the extracted netlist.
The layout looks as follows (such a figure can be printed using \io{getepslay}, see
\manualpage{getepslay}):

\begin{figure}[h]
\centerline{\epsfig{figure=poly5/poly5.eps, width=8cm}}
\end{figure}

\section{Files}
This tutorial is located in the directory \CACDTOP{demo/poly5}.
Initially, it contains the following files:
\begin{filelist}
\item[README] A file containing information about the demo.
\item[poly5.gds] The layout of the poly5 design.
\item[tech.s] The technology file for Space. See Section \ref{PEtech}.
\item[param.p] The parameter settings file for Space. See Section \ref{PEparam}.
\item[script.sh] A file containing the commands for executing all
steps of the demo in sequence.
\end{filelist}

\section{Technology File}
\label{PEtech}
The technology file that we will use is the file \io{tech.s}, see listing below.

\small \begin{Verbatim}[frame=single]
listing of file tech.s
        ...
   3
   4    unit vdimension  1e-6  # micron
   5
   6    colors :
   7        cpg  red
   8
   9    conductors :
  10        resP : cpg : cpg : 0.0
  11
  12    vdimensions :
  13        dimP : cpg : cpg : 0.5 0.5
  14
  15    dielectrics :
  16        # Dielectric consists of 5 micron thick SiO2
  17        # (epsilon = 3.9) on a conducting plane.
  18        SiO2   3.9   0.0
  19        air    1.0   5.0
\end{Verbatim}
\normalsize
Please note that the masks to be used in this file have already been defined
in the \io{maskdata} file in the process library \CACDTOP{lib/process}.
In this example, we will only use the poly-silicon mask (\io{cpg}).

When we look to the file, we first see on line \io{4} an \io{unit} specification
for the \io{vdimensions} definitions starting on line \io{12}.
An unit of \io{1e-6} $m$ is chosen for the values.
Thus, the values must be specified in $\mu m$.

Starting on line \io{6} we find the \io{colors} specification.
The color of the \io{cpg} mask is chosen to be \io{red}.
Note that this info is only used by the visualization program \io{Xspace}.

Starting on line \io{9} we find the \io{conductors} specification.
This is an important section, because without this we can not specify
element pins and even a vdimension.
The conductor value is for this example unimportant,
because we don't extract resistances.

For 3D capacitance extraction, we need only two additional specifications.
First, the \io{vdimensions} definitions for building the 3D conductor mesh.
This section specifies respectivily the bottom (against the ground plane) and the thickness
of each conductor.
Second, the \io{dielectrics} definitions specify the dielectric properties of the media
around the conductors, respectivily the relative permitivity $\epsilon_r$ and the bottom in $\mu m$.
Note that the thickness of the last medium (\io{air}) extends to infinity.

\section{Parameter File}
\label{PEparam}
The parameter file \io{param.p} is controlling the extraction, see listing below.

\small \begin{Verbatim}[frame=single]
listing of file param.p
        ...
   3
   4    BEGIN cap3d
   5    be_mode           0c
   6    max_be_area       0.5
   7    be_window         1
   8    END cap3d
\end{Verbatim}
\normalsize
In the above listing, we see a \io{cap3d} block from line \io{4} to line \io{8}.
These parameters are only used for controlling capacitance 3D extraction.
The \io{max\_be\_area} (in $\mu m^2$) and \io{be\_window} (width in $\mu m$) parameters
needs always to be specified.
On line \io{5}, you find also the \io{be\_mode} parameter.
However, this boundary element mode is the default (see 3D Cap. User's Manual).

\section{Running the Extractor}
The file \io{script.sh} is a batch file for running all the commands
for this example.
First, it changes the current working directory '.' into a project directory:
\small
\begin{Verbatim}
% mkpr -p scmos_n -l 0.05 .
\end{Verbatim}
\normalsize
It uses the \io{scmos\_n} process from the technology library.
We use the mask names as defined in the \io{maskdata} file of the library.
But, see before, we are not using the default technology
file \io{space.def.s} and parameter file \io{space.def.p} of the library.
Thus, the local technology file \io{tech.s} needs to be compiled with \io{tecc} to the format
(a \io{.t} file) which is used by the extractor:
\small
\begin{Verbatim}
% tecc tech.s
\end{Verbatim}
\normalsize
Second, the layout description is put into the project database.
The layout is supplied in a GDS2 file, which can be converted to
internal database format with the \io{cgi} program:
\small
\begin{Verbatim}
% cgi poly5.gds
\end{Verbatim}
\normalsize
Now, we can extract a circuit description for the layout of the \io{poly5} cell, as follows:
\small
\begin{Verbatim}
% space3d -C3 -E tech.t -P param.p poly5
\end{Verbatim}
\normalsize
In this case, the capacitances are calculated, using a 3D mesh method (BEM).
The method calculates couple capacitances between the conductors and between the conductors
and a ground plane.
\\[1 ex]
The extracted circuit can be inspected using the \io{xspice} program.
We use also the \io{-f} option to get a file \io{poly5.spc}:
\small
\begin{Verbatim}
% xspice -af poly5
\end{Verbatim}
\normalsize
The content of the file is listed below.

\small \begin{Verbatim}[frame=single]
listing of file poly5.spc
   1    poly5
   2
   3    * Generated by: xspice 2.39 25-Jan-2006
   4    * Date: 19-Jun-06 16:03:04 GMT
   5    * Path: /users/simon/poly5
   6    * Language: SPICE
   7
   8    * circuit poly5 e d c b a
   9    c1 a b 253.3136e-18
  10    c2 a GND 624.1936e-18
  11    c3 b c 253.3136e-18
  12    c4 b GND 457.9544e-18
  13    c5 c d 253.3136e-18
  14    c6 c GND 457.9544e-18
  15    c7 e d 253.3136e-18
  16    c8 e GND 624.1936e-18
  17    c9 d GND 457.9544e-18
  18    * end poly5
\end{Verbatim}
\normalsize
We can make a drawing of this listing of the extracted circuit.
For this topology see the figure below.

\begin{figure}[h]
\centerline{\epsfig{figure=poly5/circuit.eps, width=11cm}}
\end{figure}

Note that there are no couple capacitances between conductors (a, c), (b, d) and (c, e).
This happens, because the chosen \io{be\_window} is $1 \mu m$ width, and the distance between these
conductors is more than $1 \mu m$.
Note that the "Space 3D Capacitance User's Manual" shows also the capacitance values in a table
for other values of the \io{be\_window}.
