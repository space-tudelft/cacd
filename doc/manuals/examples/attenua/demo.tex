\chapter{attenua Example of Bipolar Extraction}
\section{Introduction}
\label{PEintro}
This example gives a demonstration of an attenuator circuit in bipolar DIMES-01 process.
See also: A. van Staveren and A.H.M. van Roermund,
"Low-voltage Low-power Controlled Attenuator for Hearing Aids",
Electronic Letters, 22 July 1993, Vol. 29, No. 15, pp. 1355-1356.
\\[1 ex]
The layout looks as follows, using the layout editor \io{dali} (see \manualpage{dali}):

\begin{figure}[h]
\centerline{\epsfig{figure=attenua/attenua.eps, width=11cm}}
\end{figure}

\section{Files}
This tutorial is located in the directory \CACDTOP{demo/attenua}.
Initially, it contains the following files:
\begin{filelist}
\item[README] A file containing information about the demo.
\item[attenua.cmd] Command file for a PSPICE simulation.
\item[attenua.gds] GDS2 file of the layout of the design.
\item[att\_ref.spc] SPICE file of the reference circuit.
\item[npnBW.dev] Device file containing npnBW device.
\item[pnpWP.dev] Device file containing pnpWP device.
\item[script.sh] Batch file for running all commands of the demo in sequence.
\item[spice3f3.lib] Library file for circuit listing.
\item[xspicerc] Init file for circuit listing.
\end{filelist}

\section{Running the Extractor}
First, use the following command to change the current working directory '.' into a project directory:
\small
\begin{Verbatim}
% mkpr -p dimes01 -l 0.1 .
\end{Verbatim}
\normalsize
The command specifies the \io{dimes01} process from the technology library
and a lambda (design unit) of $0.1 \mu m$.
We use the mask names as defined in the \io{maskdata} file of the library.
And we use the default technology file \io{space.def.s} and parameter file \io{space.def.p} of the library.
\\[1 ex]
Second, the layout description is put into the project database.
The layout is supplied in a GDS2 file, which is converted into
an internal database format with the \io{cgi} program:
\small
\begin{Verbatim}
% cgi attenua.gds
\end{Verbatim}
\normalsize
Now, using verbose mode, perform a flat extraction of cell "attenua".
\small
\begin{Verbatim}
% space -vF attenua
\end{Verbatim}
\normalsize
\small \begin{Verbatim}[frame=single]
Version 5.3.1, compiled on Fri Feb 03 12:45:53 GMT 2006
See http://www.space.tudelft.nl
parameter file: $ICDPATH/share/lib/process/dimes01/space.def.p
technology file: $ICDPATH/share/lib/process/dimes01/space.def.t
preprocessing attenua (phase 1 - flattening layout)
preprocessing attenua (phase 2 - removing overlap)
extracting attenua

extraction statistics for layout attenua:
	capacitances        : 0
	resistances         : 0
	nodes               : 31
	mos transistors     : 0
	bipolar vertical    : 39
	bipolar lateral     : 59
	substrate nodes     : 0

overall resource utilization:
	memory allocation  : 0.408 Mbyte
	user time          :         0.0
	system time        :         0.0
	real time          :         2.0   1%

space: --- Finished ---
\end{Verbatim}
\normalsize
You can try out the options \io{-c}, \io{-C}, \io{-r} and \io{-z} for capacitance and resistance extraction.
\\[1 ex]
The extracted circuit can be inspected using the \io{xspice} program.
We use the \io{-a} option to get alpha-numeric node names, type:
\small
\begin{Verbatim}
% xspice -a attenua
\end{Verbatim}
\normalsize
This gives the following output (for an extraction w/o cap and res):
\small \begin{Verbatim}[frame=single]
attenua

* circuit attenua iref comp4 comp1 comp2 comp3 comp6 rshunt comp5 ireg vreg
*                 switch asssymm rschaal2 rschaal1 rcshunt rv ground outneg
*                 Vcc outpos compv comcomp
q1 comp4 1 Vcc wp102c
q2 comp2 1 rschaal1 wp102c
q3 comp5 1 rschaal2 wp102c
q4 1 1 Vcc wp102c
q5 comp2 comp1 ground bw101a
q6 1 iref ground bw101a
   ...
q9 2 2 Vcc wp102c
q10 comp5 comp4 ground bw101a
q11 comp1 1 Vcc wp102c
q12 comp1 comp1 ground bw101a
q13 iref iref ground bw101a
q14 rshunt 2 Vcc wp102c
q15 comp4 comp4 ground bw101a
q16 2 3 ground bw101a
q17 comp1 comp1 ground bw101a
q18 rcshunt rshunt Vcc wp102c
q19 comp4 comp4 ground bw101a
q20 3 3 ground bw101a
q21 switch iref ground bw101a
q22 comp1 comp3 ground bw101a
q23 rcshunt iref ground bw101a
    ...
q26 comp4 comp6 ground bw101a
q27 comp6 rcshunt Vcc wp102c
q28 comp3 comp2 ground bw101a
q29 3 4 Vcc wp102c
q30 rshunt iref ground bw101a
q31 compv switch Vcc wp102c
q32 compv switch Vcc wp102c
q33 comp6 comp5 ground bw101a
q34 4 comcomp ground bw101a
q35 comp3 rcshunt Vcc wp102c
q36 8 8 Vcc wp102c
q37 ireg 8 Vcc wp102c
q38 outneg comp2 vreg bw101a
q39 comcomp comp2 vreg bw101a
q40 comcomp comp5 vreg bw101a
q41 outpos comp5 vreg bw101a
q42 5 switch Vcc wp102c
q43 5 switch Vcc wp102c
q44 5 5 ireg bw101a
q45 compv 5 rv bw101a
q46 6 compv Vcc wp102c
q47 Vcc 6 rv bw101a
q48 switch switch Vcc wp102c
    ...
q81 7 switch Vcc wp102c
q82 7 switch Vcc wp102c
q83 7 7 ground bw101a
    ...
q92 8 7 ground bw101a
q93 comcomp switch Vcc wp102c
q94 comcomp switch Vcc wp102c
q95 9 switch Vcc wp102c
q96 9 switch Vcc wp102c
q97 outneg asssymm Vcc wp102c
q98 outpos asssymm Vcc wp102c
* end attenua

.model wp102c pnp( ... )
.model bw101a npn( ... )
\end{Verbatim}
\normalsize
If you have PSPICE, you can perform a circuit simulation (after
customizing the script \io{nspice}) as follows (see \manualpage{nspice}):
\small
\begin{Verbatim}
% nspice attenua attenua.cmd
\end{Verbatim}
\normalsize

\section{Running a Circuit Comparison}
You can compare the extracted circuit against a reference circuit
with the circuit comparison program \io{match}.
\\[1 ex]
First, add device descriptions for the bipolar transistors to the
database so that the reference circuit can be stored into the database:
\small
\begin{Verbatim}
% putdevmod pnpWP.dev npnBW.dev
% xcontrol -device pnpWP npnBW
\end{Verbatim}
\normalsize
Second, add the reference circuit description for the "attenua" circuit
to the database using the program \io{cspice}:
\small
\begin{Verbatim}
% cspice att_ref.spc
\end{Verbatim}
\normalsize
Now, compare this reference circuit with the extracted circuit, type:
\small
\begin{Verbatim}
% match att_ref attenua
\end{Verbatim}
\normalsize
\small \begin{Verbatim}[frame=single]

match: Succeeded.

\end{Verbatim}
\normalsize
The above result shows that the circuits are identical.
Note that you can use the \io{-bindings} option to get more information.
\\[1 ex]
You can also try to change the layout of the "attenua".
For example, use the layout editor \io{dali} to make an error by deleting some connection.
Before you run the compare tool \io{match} again,
extract the circuit of cell "attenua" again, type:
\small
\begin{Verbatim}
% space -F attenua
% match att_ref attenua
\end{Verbatim}
\normalsize
