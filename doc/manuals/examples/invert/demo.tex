\chapter{invert Example of Junction Capacitance Extraction}
\section{Introduction}
\label{PEintro}
In this example,
we demonstrate the extraction of an inverter with transistor bulk
connections and drain/source regions.
The latter are extracted as either non-linear junction capacitances,
or as drain/source area and perimeter information attached to the MOS transistors.
See also the "Space Tutorial" sections "3.8", "4" and "5".
\\[1 ex]
The layout looks as follows, using the layout editor \io{dali} (see \manualpage{dali}):

\begin{figure}[h]
\centerline{\epsfig{figure=invert/invert.eps, width=11cm}}
\end{figure}

\section{Files}
This tutorial is located in the directory \CACDTOP{demo/invert}.
Initially, it contains the following files:
\begin{filelist}
\item[README] A file containing information about the demo.
\item[invert.cmd] Command file for circuit simulation.
\item[invert.gds] GDS2 file of the layout of the design.
\item[jun.lib] Model library file for circuit listing.
\item[script.sh] Batch file for running all commands of the demo in sequence.
\item[xspicerc] Init file for circuit listing.
\end{filelist}

\section{Running the Extractor}
First, we need a project directory.
To change the current working directory '.' into a project directory, type:
\small
\begin{Verbatim}
% mkpr -p scmos_n -l 0.2 .
\end{Verbatim}
\normalsize
We use the \io{scmos\_n} process from the technology library and a lambda of $0.2 \mu m$.
We are using the default technology
file \io{space.def.s} and parameter file \io{space.def.p} of the library.
\\[1 ex]
Second, the layout description is put into the project database.
The layout is supplied in a GDS2 file, which can be converted to
internal database format with the \io{cgi} program:
\small
\begin{Verbatim}
% cgi invert.gds
\end{Verbatim}
\normalsize
Now, we can perform an extraction of the "invert" cell.
Use option \io{-C} to include coupling capacitances.
The drain/source regions are extracted as non-linear junction capacitances
that have parameters 'area' and 'perimeter' (using option \io{-S}).
The following command is used:
\small
\begin{Verbatim}
% space -C -Sjun_caps=area-perimeter invert
\end{Verbatim}
\normalsize
Look to the listing of file "xspicerc" to see how the junction capacitances are
printed in the netlist output as diodes with parameters '\io{area}' and '\io{pj}':

\small \begin{Verbatim}[frame=single]
listing of file xspicerc
  4    include_library spice3f3.lib
  5    include_library jun.lib
  6
  7    model nenh_0 nenh nmos ()
  8    model penh_0 penh pmos ()
  9    model ndif  ndif  d ()
 10    model nwell nwell d ()
 11    model pdif  pdif  d ()
 12
 13    bulk nmos 0.0
 14    bulk pmos 5.0
 15
 16    params ndif  { area=$area pj=$perim }
 17    params nwell { area=$area pj=$perim }
 18    params pdif  { area=$area pj=$perim }
\end{Verbatim}
\normalsize
Next, retrieve a SPICE circuit description.
Use the \io{xspice} command with options \io{-a} and \io{-u} to e\io{x}stract SPICE from the database:
\small
\begin{Verbatim}
% xspice -au invert
\end{Verbatim}
\normalsize
\small \begin{Verbatim}[frame=single]
invert

* circuit invert b_in t_out l_vdd r_vdd l_vss r_vss
vnet1 l_vdd r_vdd 0
vnet2 l_vss r_vss 0
m1 t_out b_in l_vss l_vss nenh_0 w=6.8u l=1.2u
m2 l_vdd b_in t_out l_vdd penh_0 w=12.4u l=1.2u
c1 l_vdd b_in 795.12e-18
c2 l_vdd t_out 3.04892f
d1 t_out l_vdd pdif area=20.48p pj=14.8u
c3 l_vdd GND 249.6e-18
d2 GND l_vdd nwell area=155.52p pj=50.4u
c4 b_in t_out 495.12e-18
c5 b_in GND 3.21264f
c6 b_in l_vss 282.72e-18
c7 t_out GND 987.68e-18
d3 GND t_out ndif area=16.16p pj=11.2u
c8 l_vss GND 774.4e-18
d4 GND l_vss ndif area=27.92p pj=22.6u
* end invert

.model nenh_0 nmos(level=2 ...)
.model penh_0 pmos(level=2 ...)
.model pdif d(is=10u cjo=500u vj=800m m=500m)
.model nwell d(is=2u cjo=100u vj=800m m=500m)
.model ndif d(is=2u cjo=100u vj=800m m=500m)
\end{Verbatim}
\normalsize
An alternative is to extract the drain/source regions as 'area' and 'perimeter'
information that is attached to the MOS transistors.
This is achieved by modifying the transistor definitions in the element definition file.
Therefore, first copy the element definition file from the process directory to the local file "elem.s":
\small
\begin{Verbatim}
% cp $ICDPATH/share/lib/process/scmos_n/space.def.s elem.s
\end{Verbatim}
\normalsize
\small \begin{Verbatim}[frame=single]
listing of file elem.s
    ...
41  conductors :
42  # name    : condition     : mask : resistivity : type
43    cond_mf : cmf           : cmf  : 0.045       : m   # first metal
44    cond_ms : cms           : cms  : 0.030       : m   # second metal
45    cond_pg : cpg           : cpg  : 40          : m   # poly interconnect
46    cond_pa : caa !cpg !csn : caa  : 70          : p   # p+ active area
47    cond_na : caa !cpg  csn : caa  : 50          : n   # n+ active area
48    cond_well : cwn         : cwn  : 0           : n   # n well
49
50  fets :
51  # name : condition    : gate d/s : bulk
52    nenh : cpg caa  csn : cpg  caa : @sub  # nenh MOS
53    penh : cpg caa !csn : cpg  caa : cwn   # penh MOS
    ...
66  junction capacitances ndif :
67  # name    :  condition                 : mask1 mask2 : capacitivity
68    acap_na :  caa       !cpg  csn !cwn  : @gnd  caa : 100  # n+ bottom
69    ecap_na : !caa -caa !-cpg -csn !-cwn : @gnd -caa : 300  # n+ sidewall
70
71  junction capacitances nwell :
72    acap_cw :  cwn                  : @gnd  cwn : 100  # bottom
73    ecap_cw : !cwn -cwn             : @gnd -cwn : 800  # sidewall
74
75  junction capacitances pdif :
76    acap_pa :  caa       !cpg  !csn cwn      :  caa cwn : 500 # p+ bottom
77    ecap_pa : !caa -caa !-cpg !-csn cwn -cwn : -caa cwn : 600 # p+ sidewall
    ...
\end{Verbatim}
\normalsize
Change in line 46 and in line 47 the resistance value to zero
(conductor elements \io{cond\_pa} and \io{cond\_na}).
This, to avoid complaints by \io{tecc} about the fact the resistance for drain/source
regions should be zero (the resistance is already modeled in the transistor model).
\\[1 ex]
Change line 52 and line 53 of the fets definitions, into:
\small \begin{Verbatim}[frame=single]
52    nenh : cpg caa  csn : cpg caa (!cpg caa csn) : @sub # nenh MOS
53    penh : cpg caa !csn : cpg caa (!cpg caa !csn): cwn  # penh MOS
\end{Verbatim}
\normalsize
This, to extract area/perimeter information of the drain/source regions.
And remove the \io{ndif} and \io{pdif} junction capacitance lists,
lines (68,69) and lines (76,77).
\\[1 ex]
Next, run the technology compiler \io{tecc} to compile the new element definition file:
\small
\begin{Verbatim}
% tecc elem.s
\end{Verbatim}
\normalsize
Now, extract again, using the compiled element definition file:
\small
\begin{Verbatim}
% space -C -Sjun_caps=area-perimeter -E elem.t invert
\end{Verbatim}
\normalsize
And watch the resulting SPICE output again:
\small
\begin{Verbatim}
% xspice -au invert
\end{Verbatim}
\normalsize
\small \begin{Verbatim}[frame=single]
invert

* circuit invert b_in t_out l_vdd r_vdd l_vss r_vss
vnet1 l_vdd r_vdd 0
vnet2 l_vss r_vss 0
m1 t_out b_in l_vss l_vss nenh_0 w=6.8u l=1.2u ad=16.16p as=27.92p pd=11.2u
+  ps=22.6u nrs=0.603806 nrd=0.349481
m2 l_vdd b_in t_out l_vdd penh_0 w=12.4u l=1.2u ad=23.36p as=20.48p pd=10.8u
+  ps=14.8u nrs=0.133195 nrd=0.151925
c1 l_vdd b_in 795.12e-18
c2 l_vdd t_out 3.04892f
c3 l_vdd GND 249.6e-18
d1 GND l_vdd nwell area=155.52p pj=50.4u
c4 b_in t_out 495.12e-18
c5 b_in GND 3.21264f
c6 b_in l_vss 282.72e-18
c7 t_out GND 987.68e-18
c8 l_vss GND 774.4e-18
* end invert

.model nenh_0 nmos(level=2 ...)
.model penh_0 pmos(level=2 ...)
.model nwell d(is=2u cjo=100u vj=800m m=500m)
\end{Verbatim}
\normalsize
