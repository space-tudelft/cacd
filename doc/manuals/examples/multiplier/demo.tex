\chapter{multiplier Example of Functional/Switch-Level Simulation}
\section{Introduction}
\label{PEintro}
This example demonstrates how you can do a combined functional and switch-level simulation.
Read also the following documents: "SLS: Switch-Level Simulator User's Manual"
and "Functional Simulation User's Manual".
\\[1 ex]
In this example a network 'total' is simulated that has a ram and a
multiplier that are described at the functional level.
Pass transistors are used to demultiplex the output
signals of the ram and to multiply two subsequent
words that come out of the ram.
The circuit looks as follows:

\begin{figure}[h]
\centerline{\epsfig{figure=multiplier/total.eps, width=12cm}}
\end{figure}

\section{Files}
This tutorial is located in the directory \CACDTOP{demo/multiplier}.
Initially, it contains the following files:
\begin{filelist}
\item[README] A file containing information about the demo.
\item[mul8x8.fun] Functional description of the mul8x8 block.
\item[ram.fun] Functional description of the ram block.
\item[total.cmd] Command file for the circuit simulation.
\item[total.sls] SLS description of the total network.
\item[script.sh] Batch file for running all commands of the demo in sequence.
\end{filelist}

\section{Compiling the Descriptions}
To compile the descriptions and to simulate, we need to work in a project directory.
First, we change the current working directory '.' into a project directory:
\small
\begin{Verbatim}
% mkpr -p scmos_n .
\end{Verbatim}
\normalsize
We use the "scmos\_n" technology and a default lambda value.
\\[1 ex]
To compile the function blocks, we use the \io{cfun} program.
To add the functional descriptions to the project database, type the following commands:
\small
\begin{Verbatim}
% cfun ram.fun
% cfun mul8x8.fun
\end{Verbatim}
\normalsize
To compile the SLS network description, we use the \io{csls} program.
Type the following command to add network \io{total} to the database:
\small
\begin{Verbatim}
% csls total.sls
\end{Verbatim}
\normalsize
To get a hierarchical cell listing (of the circuit tree) of the project database,
use the \io{dblist} command:
\small
\begin{Verbatim}
% dblist -h
\end{Verbatim}
\normalsize
\small \begin{Verbatim}[frame=single]

circuit:
1 - total           (18)
    2 - ram               1 (function)
    2 - mul8x8            1 (function)

\end{Verbatim}
\normalsize
You see 2 hierarchical levels.
The top cell "total" has 2 sub cells (function blocks).

\section{Running the Switch-Level Simulation}
For this simulation you are using the switch-level simulator \io{sls}.
See the "SLS: Switch-Level Simulator User's Manual" and for the manual page \manualpage{sls}.
This simulator is started from the simulation GUI \io{simeye} and
the results are shown in the output window (see \manualpage{simeye}).
\\[1 ex]
First, start the simulation GUI \io{simeye}.
\small
\begin{Verbatim}
% simeye
\end{Verbatim}
\normalsize
Second, prepare the simulation:
Click on the "Simulate" menu and choice the "Prepare" item.
Select in the "Circuit:" field cell name "total" and
in the "Stimuli:" field file name "total.cmd" (click on it).
\\[1 ex]
Third, start the switch-level simulation:
Click on the "Run" button and wait for the simulation results.
\\[1 ex]
After that, watch the results:
Zoom-in onto the lowest 3 signals, choice the "ZoomIn" item of the "View" menu.
Click in the results window and move the mouse to specify a zooming area and click again.
Now, choice "Measure" from the "View" menu, and click on the right
mouse button to watch the integer values of the multiplied signals.
\\[1 ex]
At last, to exit \io{simeye} program,
go to the "File" menu and click on "Exit" and "Yes".
