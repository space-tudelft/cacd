% -*- latex -*-

\chapter{Simulating circuits with 'simeye'}
\label{s-simeye}
\index{simeye@\tool{simeye}|(bold}
\index{simulation!interactive|see{\tool{simeye}}}
\tool{Simeye} provides a graphical interface to the circuit simulators.
It supports three levels of circuit simulation: logic, switch-level and spice.
This section discusses the \tool{simeye} command line options and its user
interface.
\section{Description}
{\it Simeye}
is an X-window based simulation user interface for the
{\it sls(1ICD)}
simulator and the
{\it spice(1ICD)}
simulator.
The program
can be used to inspect the graphical representation of the
output signals of both simulators and
it can be used to edit input signals that
are described in the sls command file format 
(see "sls: switch-level simulator user's manual").
Moreover, from 
{\it simeye}
simulations can be run
\index{simulation!on three levels}
by starting either the
{\it sls} 
\index{sls@\tool{sls}}
\index{simulation!switch-level}
simulator or the 
{\it spice}
\index{simulation!spice-level}
\index{spice@\tool{spice}}
simulator
(The latter is done via the script %
\it nspice(1ICD)\rm%
).
\par
The following program argument may be specified:
\begin{description}
\item[{\it cell : }]
This name specifies the name of the cell for
which simulation signals are displayed and/or edited.
\end{description}
The corresponding file from which the signals are read or to which
the signals are written has a name
that is equal to the cell name, with an extension ".res"
for sls logical and timing (output) signals,
an extension ".plt"
for sls waveform (output) signals,
an extension ".ana"
for spice (output) signals,
and an extension ".cmd"
for command file (input) signals.
The spice output file should contain the spice signals
in tabular format.
When running the program 
{\it spice}
directly this is achieved by
using the card .print tran ... in the spice input file.
When running 
{\it nspice,}
this is achieved by using
sls plot commands in the command file.
\section{Commands of simeye}
Commands that can interactively be given are:
\begin{description}
\item[{\it CELL : }]
This field
specifies the name of the cell to which the signals belong.
The cell name is given by the program argument (see above)
or it is specified by the
user while running the program.
To edit it, use the left arrow, right arrow
and backspace and/or delete keys of the keyboard.
To load the signals of the specified cell, click the read button.
\item[{\it SLS-LOGIC\ \ SLS-TIMING \ \ SPICE : }]
These buttons define the type of simulation.
When clicking the run button,
an sls simulation at level 1 or 2 is performed when SLS-LOGIC
is active,
an sls simulation at level 3 is performed when SLS-TIMING
is active,
and a spice simulation is performed when SPICE
is active.
When clicking the read button,
These buttons define the type of simulation.
When clicking the run button,
an sls simulation at level 1 or 2 is performed when SLS-LOGIC
is active,
an sls simulation at level 3 is performed when SLS-TIMING
is active,
and a spice simulation is performed when SPICE
is active.
When clicking the read button,
sls signals are read when SLS-LOGIC or SLS-TIMING is active,
and spice signals are read when SPICE is active.
\item[{\it READ : }]
Load a new signal file.
Normally, current signals are cleared from the program.
However, if the shift key is held down while pressing the
read button the new signals will be appended to the current signals.
\item[{\it RUN : }]
Perform a simulation by using either the
{\it sls}
simulator or the
{\it spice}
simulator.
For both simulators
{\it simeye}
uses the command file %
\it cell\rm%
.cmd.
The spice simulator is run by calling the program 
{\it nspice.} 
If any error message occurs it will be displayed by 
{\it simeye.}
If the simulation succeeds, the simulation output 
will automatically be displayed by
{\it simeye.}
\item[{\it FULL : }]
Draw all present signals from the beginning of the simulation
time till the end of the simulation time.
\item[{\it REDRAW : }]
Redraw the current window.
\item[{\it IN : }]
Zoom in on the current window. You'll have to indicate the rectangle that you
want to zoom in. First click one corner, then the other corner of the zoom-in
area.
\item[{\it OUT : }]
Zoom out on the current window. After clicking \button{OUT} you'll have to
indicate a rectangle on the screen (with two mouse clicks). This is the
rectangle that will contain the {\sl current} window {\sl after} the zoom-out
has been performed. Consequently, for a large zoom-out you indicate a small
window and for a marginal zoom-out you indicate a big window.
\item[{\it LEFT, RIGHT, UP AND DOWN ARROW : }]
Move the current window leftwards, rightwards, upwards or downwards
respectively.
\item[{\it S : }]
Save the current window settings (i.e. start-time,
stop-time and the names of the signals that are
displayed) in a file (The name of this file is specified
in the configuration file).
\item[{\it L : }]
Load the current window settings (i.e. start-time,
stop-time and the names of the signals that are
displayed) from a file (The name of this file is specified
in the configuration file).

\item[{\it MOVE : }]
With this command the order of the signals can be changed
or one signal can be placed over another signal
in order to compare them.
First the signal that is moved or that is placed over another signal
is selected,
and then the new position of the signal is selected.
To place the selected signal over another signal one should hold
the shift key down while selecting the new position.
To remove a signal that is placed over another signal,
initially select the signal with the shift key held down.
\item[{\it VALUE : }]
Move a vertical scanline over the display window,
according to the position of the pointer,
and show the value of the corresponding x position (and y position).
When clicking the middle (or right) button of the mouse, one
may toggle between displaying
in the right margin (if the number of signals is not too large)
the y values of the signals at the selected x position.
To store the value of a particular position, click the left 
button of the mouse.
This position will then be subtracted from
the next positions of the scanline,
so as to measure delay times and/or voltage differences.
\item[{\it PRINT : }]
Generate a hardcopy of the current screen.
\item[{\it SPACE : }]
Run the layout-to-circuit extraction program. A
layout-to-circuit extraction may be aborted by typing Ctrl-C.
\item[{\it QUIT : }]
Quit the program
\item[{\it INPUT : }]
Read the command file %
\it cell\rm%
.cmd and
enable the edit menu. 
If the command file %
\it cell\rm%
.cmd does no exist,
{\it simeye}
will try to read a default command called "simeye.def.d".
First, it tries to read this file from the current directory
and second it tries to read this file from the process directory.
\end{description}

The following commands are part of the edit menu:
\begin{description}
\item[{\it GRID : }]
Specify the smallest unit for the x-axis (= time axis).
\item[{\it NEW : }]
Create a new signal. The user has to enter the name of the new
signal, and the signal will be placed at the bottom of
the window.
\item[{\it DELETE : }]
Delete one or more signals by selecting them with the cursor.
\item[{\it CLEAR : }]
Delete all signals.
\item[{\it COPY : }]
Copy the signal description from one signal to another signal.
\item[{\it EDIT : }]
Edit a particular signal by inserting a new logical level
for a particular time interval (t1, t2).
The signal description may eventually already be defined for this interval.
The insertion of the new logical level is done in two steps:
During the first click of the mouse
the signal, the new logical level and t1 are selected.
During the second click t2 is selected and the new signal
description is drawn.
To insert an interval that has a free state 
one should hold the shift key pressed down
while making the first selection.
\item[{\it YANK : }]
Store a (part of a) signal description in the buffer.
\item[{\it PUT : }]
Insert a copy of the signal description that is in the buffer onto
a particular position.
The user has to select the signal to which the contents
of the buffer is added and the time from which on the
new signal description is valid.
The new signal description may (partly) override the existing
signal description for the selected signal.
Furthermore,
the user is asked to type the number of repetitions for the signal
description that is added.
\item [ {\it SPEED : }]
Speed up the signals by some factor.  The value of
          'sigunit' in the command file will be divided by the
          value that is specified with this command.  If a value
         $<$ 1 is specified, the signals will be slowed down.

\item[{\it T\_END : }]
Update the end time of the input signals (= end time of simulation).
\item[{\it READY : }]
Disable the edit menu and
update the command file %
\it cell\rm%
\end{description}
\section{Command file for sls}
\index{sls!command file|bold}
In order to simulate a circuit,
{\it simeye}
uses a command file called %
\it cell\rm%
.cmd.
This file should contain a description
of the input signals, values for the simulation control variables,
and a listing of the terminals for which output should be generated.
This is explained in more detail in the user manuals of
the sls simulator and the spice simulator.
\par
The "set" commands in the command file
may be edited by enabling the edit menu of
{\it simeye.}
The new signal descriptions will then be written back
to the command file when updating
the command file.
However, when a set command is followed by the keyword "no\_edit",
between comment signs,
this set command will not be changed.
This is for example useful to define supply signals and
periodical signals like clock signals.
\par
When the command file contains on separate line,
between comment signs, the keyword "auto\_print"
("auto\_plot"),
{\it simeye}
will automatically add a print (plot) statement to the
command file for each terminal of the cell that is simulated, 
prior to simulation.
The new print (plot) commands will have a keyword "auto" between
comment signs following the keyword print (plot) to indicate
that the print (plot) command will be replaced each time
when a new simulation is started.
\par
An example of a default command file that can be used for
both sls and spice simulations is:
\par
% \nofill
/$\ast$ auto\_set $\ast$/\\
set /$\ast$ no\_edit $\ast$/ vdd = h$\ast$\~\\
set /$\ast$ no\_edit $\ast$/ vss = l$\ast$\~\\
set /$\ast$ no\_edit $\ast$/ phi1 = (l$\ast$110 h$\ast$80 l$\ast$10)$\ast$\~\\
set /$\ast$ no\_edit $\ast$/ phi2 = (l$\ast$10 h$\ast$80 l$\ast$110)$\ast$\~\\
option sigunit = 1n\\
option outacc = 10p\\
option simperiod = 4000\\
option level = 3\\
/$\ast$\\
$\ast$\%\\
tstep 0.1n\\
trise 0.5n\\
tfall 0.5n\\
$\ast$\%\\
$\ast$\%\\
$\ast$/\\
/$\ast$ auto\_print $\ast$/\\
/$\ast$ auto\_plot $\ast$/\\
% \fill
\section{Startup file for simeye}
\index{simeyerc@\fname{.simeyerc} file}
At start-up of the program,
{\it simeye}
will read some information from a file called ".simeyerc".
First, it tries to find this file in the current directory.
Second, it tries to open this file in the process directory.
The start-up file may contain the following keywords, followed
by a specification on the same line if the keyword ends with ':'.
\begin{description}
\item[SLS:]
Specifies the command for running the
{\it sls}
simulator.
\item[SLS\_LOGIC\_LEVEL:]
Specifies the level of simulation when "sls-logic" is selected
(use 1 or 2).
\item[SLS\_LOGIC\_SIGNAL:]
Specifies the default signal representation for sls-logic simulations
(use A or D).
\item[SLS\_TIMING\_SIGNAL:]
Specifies the default signal representation for sls-timing simulations
(use A or D).
\item[SPICE:]
Specifies the command for running the
{\it spice}
simulator (use
{\it nspice}
or a derivative of it).
\item[XDUMP\_FILE:]
Specifies the name of the X Window dump file that is generated
when the print button is clicked.
\item[PRINT:]
Specifies the command that is executed when the print button is clicked
in order to process the X Window dump file (e.g. to convert the window dump to
PostScript and to send the output to a laser-printer).
\item[PRINT\_LABEL:]
Specifies an optional label that is placed in upper
left corner of the display window when an X Window dump
is generated.
\item[SETTINGS\_FILE:]
Specifies the name of the file in (from) which the window
settings are stored (loaded) when using the "S"
button ("L" button or -L option).
\item[DETAIL\_ZOOM\_IN]
          if this keyword is specified in the configuration file,
          the zoom-in function is also defined for the y-axis of
          a (single analog) signal.  In this case it is not
          necessary to hold the shift key down (see command IN).
\item[DETAIL\_ZOOM\_ON]
If this keyword is specified in the configuration file,
the zoom-in function is also defined for the
y-axis of a (single analog) signal.
In this case it is not necessary to hold the shift key down
\item[TRY\_NON\_CAPITAL\_ON]
If this keyword is specified in the configuration file
and simeye fails to open a command file that starts with a capital letter,
the program will try to open the same command file but now starting
with a non-capital letter.
If this succeeds and
when performing a simulation, simeye will first run the program
on a copy of the command file
to expand one dimensional array node names into single node names
(e.g. a[1..3] is converted into a\_1\_ a\_2\_ a\_3\_).
\end{description}
\par
In the above specifications, the string '\$cell' may be used
to refer to the current cell name, '\$date' may be used to
refer to the current date, and '\$time' may be used to refer
to the current time.
\par
Example of a start-up file (default values are shown):
\par
% \nofill
SLS: sls \$cell \$cell.cmd\\
SLS\_LOGIC\_LEVEL: 2\\
SPICE: nspice  nspice \$cell \$cell.cmd\
XDUMP\_FILE: simeye.wd\\   
PRINT: xtops -white 1 -in simeye.wd -out \$cell.ps; pspr \$cell.ps; rm simeye.wd\\
PRINT\_LABEL: \$cell  \$date  \$time
SETTINGS\_FILE: simeye.set
% \fill

\index{simeye@\tool{simeye}|)}

