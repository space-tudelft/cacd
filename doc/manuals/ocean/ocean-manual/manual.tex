\documentclass[a4wide,chapterhead,boatheadings]{book}

\thispagestyle{empty}

\newcommand{\smc}[1]{{\sc #1}}
\newcommand{\ii}{\"{\i}}

\newfont{\sbf}{cmssbx10} % sans font bold, point size = 10

\usepackage{epsfig}

% \input dvips/psfig
% \psdraft
% \input{psfig}

% Leave some vertical space between paragraphs; No paragraph indent:
\setlength{\parskip}{0.6\baselineskip} \setlength{\parindent}{0cm}
\addtolength{\topmargin}{-1cm}
% path to the directory of this manual
\newcommand{\topdir}{.}

% Directory where to look for figures:
\newcommand{\figdir}{\topdir/figures}

% Leave some vertical space between paragraphs; No paragraph indent:
\setlength{\parskip}{0.6\baselineskip} \setlength{\parindent}{0cm}

\newcommand{\thedir}{/myproject}
\newcommand{\cd}[1]{\renewcommand{\thedir}{#1}}
% this is how the computer displays command lines:
\newcommand{\type}[1]{\begin{quote}%
                         {\mbox{\tt /user/hillary\thedir~\%~ #1}}%
                      \end{quote}}
% ...and this is how the computer displays two command lines:
\newcommand{\typeb}[2]{\begin{quote}%
                         {\mbox{\tt /user/hillary\thedir~\%~ #1}%
                              \newline%
                              \mbox{\tt /user/hillary\thedir~\%~ #2}}%
                       \end{quote}}

% this is for reading figures from directory pictures
\newcommand{\callpsfig}[2]{\psfig{figure=\topdir/../library/plaatjes/#1,#2}}

% this is how to print terminal names:

% this is how to print terminal names:
% New 'term' defs made by Rene @ April 22, 1993 to keep compatible with 
% new \psfig (\term has been defined in psfig.sty).
\newcommand{\terminal}[1]{{\tt #1}}
%%% \newcommand{\term}[1]{{\tt #1}}

% this is how to print mask names:
\newcommand{\mask}[1]{{\sbf #1}}

% this is how to print names of nelsis tools:
\newcommand{\tool}[1]{{\sl #1\/}}

% this is how to print italic
\newcommand{\ital}[1]{{\sl #1\/}}

% this is how to print name of a button in an X window:
%\newcommand{\button}[1]{{\fbox{\small\rule[-0.5ex]{0mm}{2ex}$\!\!\!\!$\sf #1}}}
%%% \newcommand{\button}[1]{%
%%%         \fbox{\small\rule[-0.3ex]{0mm}{1.7ex}%
%%%         $\!\!\!\!$\sf #1}}
% New 'button' defs made by Rene @ April 22, 1993 to keep compatible with 
% the times font
\newlength{\spiff}
\newcommand{\button}[1]
{
  \settowidth{\spiff}{\hspace{2.0mm} {\small\sf #1}}
  \framebox[\spiff]{\small\sf #1}
}

% this is how to print file names and cell names:
\newcommand{\fname}[1]{{\tt #1}}

% this is how to print file names and cell names:
\newcommand{\file}[1]{{\tt #1}}

% To print an attention message:
\newcommand{\attention}[2]{\begin{description}\item[{\bf #1}] #2 \end{description}}

% to print a warning
\newcommand{\warning}[1]{\begin{description}\item[{\bf Warning}:] #1
\end{description}}

% set fancy headings style
%% \pagestyle{fancy}
%\addtolength{\headwidth}{\marginparsep}
%\addtolength{\headwidth}{\marginparwidth}
\renewcommand{\chaptermark}[1]{\markboth{#1}{#1}} % remember chapter title
\renewcommand{\sectionmark}[1]{\markright{\thesection\ #1}}
                                                % section number and title
%\lhead[\fancyplain{}{\bf\thepage}]{\fancyplain{}{\small\rightmark}}
%\rhead[\fancyplain{}{\small\leftmark}]{\fancyplain{}{\bf\thepage}}
%\cfoot{}

% make lines a little longer
\addtolength{\oddsidemargin}{-.75cm}
\addtolength{\evensidemargin}{-.75cm}
\addtolength{\textwidth}{1.5cm} 
\addtolength{\textheight}{3.8cm} 

% I have to add this macro for the index stuff
\newcommand{\see}[2]{$\rightarrow$ {\sl #1}}
\newcommand{\bold}[1]{{\bf #1}}
\makeindex

\sloppy
\begin{document}
\pagenumbering{roman}
\vspace*{1cm}
\centerline{\huge Ocean: the Sea-of-Gates Design System}
\vspace*{0.5cm}
\large
\centerline{Patrick Groeneveld and Paul Stravers}
\vspace*{0.6cm}
\centerline{Delft University of Technology, faculty of Electrical Engineering}
\centerline{Delft, the Netherlands}
\centerline{e-mail: space-support-ewi@tudelft.nl}
\centerline{\today}
\normalsize
\vspace*{2cm}
\centerline{\psfig{figure=\figdir/sog11.eps,width=0.9\textwidth}}
\newpage
\pagestyle{plain}
Picture on cover page:\\
The 11th chip which was made with the \smc{ocean} sea-of-gates design system.
The 200,000 transistor chip contains is configured as a 'multi-project chip',
containing a number of independent designs.
The top quarter contains a few smaller analog circuits and large on-chip
capacitor. The rest of the chip contains a DCF-77 radio clock receiver circuit. 

\vspace*{\fill}
%
%\centerline{\fbox{
%\psfig{figure=\figdir/tools.eps,width=\linewidth}
%}}
%\vspace{1cm}
%\centerline{The \smc{ocean} sea-of-gates design system}
\centerline{\psfig{figure=\figdir/patrick.eps,width=3cm,height=4.2cm}
\hspace{1cm}~\psfig{figure=\figdir/paul.eps,width=3cm,height=4.2cm}}
\vspace{1cm}
Copyright \copyright 1994 by P. Groeneveld and P. Stravers.\\

This document was processed with the use of latex.
\newpage
\pagestyle{plain} 
\tableofcontents
\newpage
\input \topdir/intro.tex

\input \topdir/image.tex

\input \topdir/tutorial.tex

\input \topdir/programs.tex

\input \topdir/seadali.tex

\input \topdir/madonna.tex

\input \topdir/trout.tex

\input \topdir/tricks.tex

\input \topdir/kissis.tex

\input \topdir/extract.tex

\input \topdir/simeye.tex

\input \topdir/library.tex

\input \topdir/ack.tex

\clearpage              % advance to the page where the index starts
\addcontentsline{toc}{chapter}{Index}

\input \topdir/index.tex

\end{document} 
