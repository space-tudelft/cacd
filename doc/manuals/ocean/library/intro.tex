% this is latex

\section{About this document}


This document describes the images and libraries supplied with the {\sc ocean}
system. The 'image' is the basic pattern on a semi-custom chip. We distribute
three types of images:
\begin{itemize}
\item[{\em fishbone}]
A gate-isolation image in a $1.6 \mu$ process with two layers of metal. 
\item[{\em octagon}]
A remarkable octagonal image in an imaginary $0.8 \mu$ process with three layers
of metal interconnect. 
\item[{\em gatearray}]
An old fashioned gate-array in a single metal layer process.
\end{itemize}
At Delft university we can only process the fishbone image.  Therefore most of
this document and the other manuals deal with this image. The other images are
supplied mainly to demonstrate the features of the placer and the router in the
{\sc ocean} system.  Therefore they have a small cell library and the SPICE and
SLS simulations will not give realistic results. 

In any case. the librar The library is quite small, since it doesn't contain
the intricate combinatorial logic cells. This was done deliberately to keep
it simple for the users. Moreover, on our sea-of-gates style complex
combinatorial cells can be constructed quite efficiently, so there is not
really a need for such cells in the library.


of the libary
This document describes the 'fishbone' Sea-of-Gates library which is used at
Delft university of technology, in the release of january 1993. The cells
are in the library called 'oplib1\_93'. The library contains 12 digital cells
and 5 analog cells.  The library is quite small, since it doesn't contain
the intricate combinatorial logic cells. This was done deliberately to keep
it simple for the users. Moreover, on our sea-of-gates style complex
combinatorial cells can be constructed quite efficiently, so there is not
really a need for such cells in the library.

\section{Additional information and related documents}
This is a library description. To get information about the design tools which
use these library cells, we refer to the following documents:
\begin{itemize}
\item
{\em The Seafood Menu for Layout: Seadali, Trout, Madonna and more}\\
This is the reference manual of the {\sc ocean} layout tools for Sea-of-Gates.
It is available in the ocean distribution or from the authors. 
\item
{\em A tutorial introduction to the {\sc ocean} tools}\\
Contains a short guided tour along the tools of the design system. We
think that this is the best way to get started with the system.
Available in the ocean distribution or from the authors.
\item
{\em The NELSIS IC design system manuals}\\ 
Especially interesting are the manuals on \tool{simeye}, \tool{sls} 
and \tool{space}. Available via the authors.
\item 
On-line documentation: icdman\\ 
Just type 'icdman simeye' to get the tool
description of \tool{simeye}.  This works similarly for any other nelsis tool.
To get on-line help on the ocean tools, type 'trout -h' (or 'madonna -h' to
get brief information.
\item
{\em Studentenhandleiding Ontwerppracticum}\\ 
This is the elaborate student's
manual (250 pages) for the second year chip design course at Delft universty of
technology.  Unifortunately its in Dutch. It contains all information the
students need. Available from the Diktatenverkoop electrical engineering, or
via the authors.
\end{itemize}

Finally a special word to you nasty American lawyers which might read this
(healthy people: skip this paragraph): We, some humble employees of Delft
University of Technology, and Delft University of Technology do not give
warranties of any kind, either express or
implied, as to any matter whatsoever, including but not limited to implied
warranties of merchantability or fitness for any particular purpose or that
the use of this library will not infringe any patent or copyright.

\section{Information of each library cell}

The names of the digital cells were chosen using the convention which was
used by Philips for its gate-array packages. The first three
characters describe the function of the cell. Sometimes the number of inputs
is included in the names: 'na210' is a 2-input nand, while 'na310' is a
3-input nand. The last 2 digits describe the fan-out of the output. But
sometimes I don't understand this convention myself.

The description of each cell consists of:
\begin{itemize}
\item
Function 
\item
Terminal connections
\item
IEC symbol
\item
Truth table
\item
Parameters to determine the delay
\item
Equivalent chip area
\end{itemize}

The parameters for the circuit delay consist of three parts:
\begin{tabbing}
xxxxxxxxxxxxxxxxxx\=\kill
$T_{PLH}$ en $T_{PHL}$\> the fixed delay times of the cell.\\
\\
${\Delta}T_{PLH}$ en ${\Delta}T_{PHL}$\> The delay coefficients with a
capacitative load.\\
\\
$C_{in}$\> The input capacitance.\\
\end{tabbing}

The delay times and delay coefficients are specified for the rising (LH) as
well as the falling (HL) edge of the output signal. The total delay can be 
calculated using the formula:
\begin{description}
\item
$T_{\it P_{tot}} = T_{\it P} + \Delta T_{\it P}.C_{\it load}$
\end{description}
in which $C_{load}$ is the load capacitance. This load capacitance is the sum
of the input capacitances of the cells which are driven plus the capacity of
the interconnection wires.
The unit of chip area is defined as the smallest piece of the fishbone image,
which consists of one nmos and one pmos transistor.
\clearpage
